\usepackage[utf8]{inputenc}
\usepackage[T1]{fontenc}
\usepackage{mathptmx}
\usepackage[scaled=.90]{helvet}
\usepackage{courier}
\usepackage{caption}
\captionsetup{labelformat=empty,labelsep=none}
\usepackage{verbatim}
\usepackage{hyperref}
\usepackage{listings}
% strikethrough (\sout)
\usepackage{ulem}
\lstset{language=Perl,basicstyle=\normalsize,tabsize=3,showstringspaces=false}

\title{Helpdesk::Integration}
\author[racke]{Stefan Hornburg (Racke)\\ \texttt{racke@linuxia.de}}
\date{Perl::Dancer Conference 2014, Hancock, 9th October 2014}

\begin{document}
\maketitle{}

\begin{frame}
  \titlepage
\end{frame}

\tableofcontents

\section{Helpdesk Workflow}

The the idea of a helpdesk system is to write an email to the helpdesk
or login to the helpdesk frontend.

\begin{frame}[fragile]{The Idea}
\begin{itemize}
\item mycustomer@support.linuxia.de
\item https://support.linuxia.de/
\end{itemize}
\end{frame}

The reality is though ...

% http://en.wikipedia.org/wiki/Telephone#mediaviewer/File:Alt_Telefon.jpg

\begin{frame}{The Reality}
\begin{center}
  \includegraphics[width=\textwidth,height=1\textheight,keepaspectratio]{pics/phone.jpg}
\end{center}
\end{frame}

\begin{frame}[fragile]{The Reality}
\begin{lstlisting}
From: Kuss Tommer <kuss@tommer.de>
To: Stefan Hornburg <racke@linuxia.de>
Subject: Fw: Fw: Foobar

nada nada
\end{lstlisting}
\end{frame}

\begin{frame}{The Reality}
\begin{itemize}
\item Copy \& Waste
\item Sender ?
\item Attachments
\end{itemize}
\end{frame}

\begin{frame}{The Reality}
\begin{itemize} 
\item Company A: Request Tracker
\item Company B: Teamwork
\end{itemize}
\end{frame}

\begin{frame}[fragile]{The Solution}
\begin{lstlisting}
helpdesk-integration --source=imap 
    --target=helpdesk 
    --subject="Foobar"
\end{lstlisting}
\end{frame}

\section{Supported Systems}

We are currently supporting the following helpdesk systems:

\begin{frame}{Helpdesk Systems}
\begin{itemize}
\item GitHub
\item Request Tracker
\item Teamwork
\end{itemize}
\end{frame}

\subsection{GitHub}

% screenshots ?

\begin{frame}{GitHub}
\begin{itemize}
\item Issues
\item Pull Requests
\item Labels
\end{itemize}
\end{frame}

\begin{frame}{RequestTracker}
\begin{itemize}
\item Queues
\item Tickets
\item RT 3 und RT 4
\end{itemize}
\end{frame}

\begin{frame}{Teamwork}
\begin{itemize}
\item Projects
\item TODO lists
\end{itemize}
\end{frame}

\section{Email and Calendar}
\subsection{Email}
\begin{frame}{Email}
\begin{itemize}
\item IMAP
\item PGP/GPG encryption
\end{itemize}
\end{frame}

\subsection{Calendar}
\begin{frame}{Calendar}
\begin{itemize}
\item Google Calendar
\end{itemize}
\end{frame}

\section{Examples}

\begin{frame}[fragile]{IMAP}
\begin{lstlisting}
inbox:
  type: imap
  server: ``support.linuxia.de''
  user: racke
  password: nevairbe
  ssl: 1
\end{lstlisting}
\end{frame}

\subsection{Google Calendar}

The calendar ID is actually for this conference.

\begin{frame}[fragile]{Configuration}
\begin{lstlisting}
  calendar:
    type: gcal
    secrets_file: data/secrets.json
    token_file: data/gcal-token.json
    calendar_id: q43d932vkfjq3lopuvgcr04p28@group.calendar.google.com
\end{lstlisting}
\end{frame}

\begin{frame}[fragile]{Google Calendar}
\begin{lstlisting}[language=bash]
helpdesk-integration --source gcal --target github
\end{lstlisting}
\end{frame}

\begin{frame}[fragile]{RT to Google Calendar}
\begin{lstlisting}[language=bash]
helpdesk-integration --source rt=1461 --target gcal
\end{lstlisting}
\end{frame}

\subsection{Email Encryption}
\begin{frame}[fragile]{Email Encryption}
\begin{lstlisting}
inbox:
  type: imap
  server: ``support.linuxia.de''
  user: racke
  password: nevairbe
  ssl: 1
  key: 9B726B71
\end{lstlisting}
\end{frame}

\section{Forecast and Contribution}

\subsection{Contribution}
\begin{frame}{Contribution}
\begin{itemize}
\item Documentation
\item Add support for other helpdesk systems
\end{itemize}
\end{frame}

\subsection{Slides}

\begin{frame}{Slides}
Slides:
\url{http://www.linuxia.de/talks/perldancer2014/helpdesk-en-beamer.pdf}
\end{frame}

\end{document}

%%% Local Variables: 
%%% mode: latex
%%% TeX-master: t
%%% End: 
