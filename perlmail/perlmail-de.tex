\usepackage{minted}

\date{GPW 2018, Gummersbach, 5. April}

\begin{document}
\maketitle

\begin{frame}
  \titlepage
\end{frame}

\cleardoublepage

\tableofcontents

\cleardoublepage

\begin{frame}{Übersicht}
  \begin{itemize}
  \item Aufbau einer Email
  \item Emailversand
  \item IMAP
  \item Use Cases  
  \end{itemize}
\end{frame}

\section{Einführung}

\subsection{Aufbau einer Email}

\begin{frame}{Aufbau einer Email}
  \begin{itemize}
  \item Header
  \item Body
  \end{itemize}
\end{frame}

\begin{frame}{Aufbau einer Email}
  \begin{itemize}
  \item Attachments
  \item HTML / Plaintext
  \end{itemize}
\end{frame}

\subsubsection{MIME::Entity}

Sowohl die gesamte Email als auch ihre Teile können als
\verb|MIME::Entity|-Objekt abgebildet werden:

\begin{frame}[fragile]{MIME::Entity}
  \begin{lstlisting}

$email = MIME::Entity->build(
    From    => '"Module Updates" <info@cpan.pm>',
    To      => '"Racke" <racke@racke.pm>',
    Subject => 'Changes in Module Foo::Bar',
    Data    => ["Added class Foo::Bar::Baz\n"],
);
  \end{lstlisting}
\end{frame}

\begin{frame}[fragile]{MIME::Entity Attachment}
  \begin{lstlisting}

$email->attach(
    Path => 'img/sunset-3193002_1920.jpg'
    Type => 'image/jpeg',
);

  \end{lstlisting}
\end{frame}

\section{Post verschicken}

\begin{frame}{Post verschicken}
  Too many ways to do it ...
\end{frame}

\begin{frame}{Legacy}
  \begin{itemize}
    \item Net::SMTP
    \item MIME::Lite
  \end{itemize}
\end{frame}

\begin{frame}{Module}
  \begin{itemize}
  \item Email::Send (deprecated)
  \item Email::Sender
  \item Email::Sender::Simple
  \end{itemize}
\end{frame}

\begin{lstlisting}
use Email::Sender::Simple qw(sendmail);
use Email::Simple;
use Email::Simple::Creator;
\end{lstlisting}

\begin{frame}[fragile]{Beispiel}
  \begin{lstlisting}
my $email = Email::Simple->create(
    header => [
        From    => '"Module Updates" <info@cpan.pm>',
        To      => '"Racke" <racke@racke.pm>',
        Subject => 'Changes in Module Foo::Bar',
    ],
    body => "Baz.\n",
);

sendmail($email);
  \end{lstlisting}
\end{frame}

\begin{frame}[fragile]{Beispiel mit Umlaut}
  \begin{lstlisting}
my $email = Email::Simple->create(
    header => [
        From    => '"Module Updates" <info@cpan.pm>',
        To      => '"Racke" <racke@racke.pm>',
        Subject =>  'Änderungen im Modul Foo::Bar',
    ],
    body => "Baz.\n",
);

sendmail($email);
  \end{lstlisting}
\end{frame}

\begin{frame}[fragile]{Mail Clients}

  \begin{itemize}
  \item[\goodsmile] Thunderbird
  \item[\sadsmile] K9 (Android)
  \end{itemize}

\end{frame}

\begin{frame}[fragile]{Korrektur: Beispiel mit Umlaut}
  \begin{lstlisting}
my $email = Email::Simple->create(
    header => [
        From    => '"Module Updates" <info@cpan.pm>',
        To      => '"Racke" <racke@racke.pm>',
        Subject => encode('MIME-Header', 
                          'Änderungen im Modul Foo::Bar',
                         ),
    ],
    body => "Baz.\n",
);

sendmail($email);
  \end{lstlisting}
\end{frame}

\begin{frame}[fragile]{Body der Email kodieren}
  \begin{lstlisting}
my $email = Email::Simple->create(
    header => [
        From    => '"Module Updates" <info@cpan.pm>',
        To      => '"Racke" <racke@racke.pm>',
        Subject => encode('MIME-Header',
                          'Änderungen im Modul Foo::Bar',
                         ),
    ],
    body => encode('UTF-8', "Überflüssig"),
);

sendmail($email);
  \end{lstlisting}
\end{frame}

\subsection{Transports}

\begin{frame}{Transports}
  % https://pixabay.com/en/sunset-plane-take-off-nature-3193002/
  \begin{figure}[!ht]
    \centering
    \includegraphics[width=0.9\linewidth]{img/sunset-3193002_1920.jpg}
  \end{figure}
\end{frame}

\begin{frame}{Transports}
  \begin{itemize}
    \item Sendmail
    \item SMTP
    \item Maildir
    \item Test
    \item Redirect
  \end{itemize}
\end{frame}

\subsubsection{Maildir-Transport}

Beim Maildir-Transport wird das Verzeichnis Maildir im aktuellen Verzeichnis
verwendet und angelegt soweit nicht vorhanden.

Deshalb verwenden wir hier das Verzeichnis Maildir im Homeverzeichnis.

\begin{lstlisting}
use Email::Sender::Transport::Maildir qw();
use File::HomeDir;
use Path::Tiny;
\end{lstlisting}

\begin{frame}[fragile]{Transport to Maildir}
  \begin{lstlisting}

my $maildir = path(File::HomeDir->my_home)
    ->child('Maildir');

my $transport = Email::Sender::Transport::Maildir->new(
   dir => $maildir
);

sendmail( $email , { transport => $transport, } );

\end{lstlisting}
\end{frame}

\subsection{Bilder einbinden}

\begin{frame}{Bilder einbinden}
  \begin{itemize}
  \item Bilder als Links
  \item Base64 kodiert
  \item CID Inline
  \end{itemize}
\end{frame}


\subsection{Emails aus Templates erzeugen}

\begin{frame}[fragile]{Emails aus Templates}
  \begin{minted}{perl}
    my $html = template $template, $tokens,
       { layout => 'email' };

    my $f    = HTML::FormatText::WithLinks->new;
    my $text = $f->parse($html);
  \end{minted}
\end{frame}

\section{IMAP}

\begin{frame}{IMAP}
%  \begin{itemize}
%  \end{itemize}
\end{frame}

\begin{frame}{IMAP Modules}
  \begin{itemize}
  \item Net::IMAP::Client
  \item Net::IMAP::Simple
  \end{itemize}
\end{frame}

\begin{frame}[fragile]{Ordner auswählen}
  Ordner auswählen:

\begin{lstlisting}
  $imap->select('Perl');
\end{lstlisting}
\end{frame}

\begin{frame}[fragile]{Emails suchen}
  Emails in ausgewählten Ordner suchen:

\begin{lstlisting}
  $ids = $imap->search( { subject => 'Perl' }, 'DATE' );
\end{lstlisting}
\end{frame}

\begin{frame}[fragile]{Verarbeitung der Emails}
  Komplette Email herunterladen:

\begin{lstlisting}
$imap->get_rfc822_body($id);
\end{lstlisting}

\end{frame}

\section{Post parsen}

\subsection{Module zum Parsen}

\begin{frame}[fragile]{Email::MIME}
  Email aus IMAP oder Datei einlesen:

  \begin{lstlisting}
    Email::MIME->new($message);
  \end{lstlisting}

\end{frame}

\begin{frame}[fragile]{Email::MIME}
MIME Parts durchlaufen:

\begin{lstlisting}
$self->mail->walk_parts(sub {
  my ($part) = @_;
  return if $part->subparts;

  if ($part->content_type =~ m{text/html}i) {
    push @out, $part->body_str;
  }
});
\end{lstlisting}
\end{frame}

\subsection{Verschlüsselte Emails}

\section{Tests und Entwicklung}

\begin{frame}{Tests und Entwicklung}
  \begin{itemize}
  \item Swaks
  \end{itemize}
\end{frame}

\section{Use Cases}

\begin{frame}{Use Cases}
  \begin{itemize}
  \item Dancer2::Plugin::Email (Email::Sender)
  \item Helpdesk::Integration
  \item Sympa
  \end{itemize}
\end{frame}

\section{Abschluß}

\subsection{Fragen}

\begin{frame}{Fragen}
% https://pixabay.com/de/fragezeichen-wissen-frage-anmelden-3255140/
  \begin{figure}[!ht]
     \centering
     \includegraphics[width=0.9\linewidth]{img/question-mark-3255140_1920.jpg}
  \end{figure}
\end{frame}

\subsection{The end}

\begin{frame}{The end}
% https://pixabay.com/de/ortsschild-ortstafel-schluss-aus-1158385/
  \begin{figure}[!ht]
     \centering
     \includegraphics[width=0.8\linewidth]{img/town-sign-1158385_1920.jpg}
  \end{figure}
\end{frame}

\end{document}

%%% Local Variables: 
%%% mode: latex
%%% TeX-master: t
%%% End: 
