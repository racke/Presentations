\begin{document}
\maketitle

\begin{frame}
  \titlepage
\end{frame}

\cleardoublepage

\tableofcontents

\cleardoublepage

\section{Einführung}

\section{Post verschicken}

\begin{frame}{Post verschicken}
  Too many ways to do it ...
\end{frame}

\begin{frame}{Legacy}
  \begin{itemize}
    \item Net::SMTP
    \item MIME::Lite
  \end{itemize}
\end{frame}

\begin{frame}{Module}
  \begin{itemize}
  \item Email::Send (deprecated)
  \item Email::Sender
  \item Email::Sender::Simple
  \end{itemize}
\end{frame}

\begin{frame}[fragile]{Beispiel}
  \begin{lstlisting}
  \end{lstlisting}
\end{frame}

\section{IMAP}

\subsection{IMAP Modules}

\begin{frame}{IMAP Modules}
  \begin{itemize}
  \item Net::IMAP::Client
  \item Net::IMAP::Simple
  \end{itemize}
\end{frame}

\begin{frame}[fragile]{Verarbeitung der Emails}
  Komplette Email herunterladen:

\begin{lstlisting}
$self->imap->get_rfc822_body($id);
\end{lstlisting}

\end{frame}

\section{Post parsen}

\subsection{Module zum Parsen}

\begin{frame}[fragile]{Email::MIME}
  Email aus IMAP oder Datei einlesen:

  \begin{lstlisting}
    Email::MIME->new($message);
  \end{lstlisting}

\end{frame}

\begin{frame}[fragile]{Email::MIME}
MIME Parts durchlaufen:

\begin{lstlisting}
$self->mail->walk_parts(sub {
  my ($part) = @_;
  return if $part->subparts;

  if ($part->content_type =~ m{text/html}i) {
    push @out, $part->body_str;
  }
});
\end{lstlisting}
\end{frame}

\subsection{Verschlüsselte Emails}

\section{Use Cases}

\begin{frame}{Use Cases}
  \begin{itemize}
  \item Dancer2::Plugin::Email (Email::Sender)
  \item Helpdesk::Integration
  \item Sympa
  \end{itemize}
\end{frame}

\section{Werkzeuge}

Swaks - !Modern perl

\section{Abschluß}

\subsection{Fragen}

\begin{frame}{Fragen}
\centering
Fragen ?
\end{frame}

\end{document}

%%% Local Variables: 
%%% mode: latex
%%% TeX-master: t
%%% End: 
