\usepackage[utf8]{inputenc}
\usepackage[T1]{fontenc}
\usepackage{mathptmx}
\usepackage[scaled=.90]{helvet}
\usepackage{courier}
\usepackage{caption}
\captionsetup{labelformat=empty,labelsep=none}
\usepackage{verbatim}
\usepackage{hyperref}
\usepackage{listings}
% strikethrough (\sout)
\usepackage{ulem}
\lstset{language=Perl,basicstyle=\normalsize,tabsize=3,showstringspaces=false}

\title{Monitoring with Nagios and check\_mk}
\author[racke]{Stefan Hornburg (Racke)\\ \texttt{racke@linuxia.de}}
\date{DORS/CLUC 2015, Zagreb}

\begin{document}
\maketitle{}

\begin{frame}
  \titlepage
\end{frame}

\tableofcontents

\section{Monitoring}

\subsection{Justification}

\begin{frame}[fragile]{Justification}
\begin{itemize}
\item Availability of server and services
\item SLA
\end{itemize}
\end{frame}

\subsection{Checks}

\begin{frame}[fragile]{Checks}
\begin{itemize}
\item diskspace
\item database
\item stuck jobs
\item products on Amazon
\item orders
\item crashes
\item import files
\item email blacklists
\item backups
\end{itemize}
\end{frame}

\section{Nagios}

\begin{frame}[fragile]{Check States}
\begin{itemize}
\item OK
\item WARINING 
\item CRITICAL
\item UNKNOWN
\end{itemize}
\end{frame}

\section{check\_mk}

Which benefits has \verb|check_mk| over a standard Nagios installation?

\begin{frame}[fragile]{Benefits}
\begin{itemize}
\item Automatic service detection
\item Rule based, hierarchical confirmation
\item High performance through passive checks
\item Creates Nagios configs for you
\end{itemize}
\end{frame}

\section{Slides}

\begin{frame}{Slides}
Slides:
\url{http://www.linuxia.de/talks/dorscluc2015/nagios-en-beamer.pdf}
\end{frame}

\end{document}

%%% Local Variables: 
%%% mode: latex
%%% TeX-master: t
%%% End: 
