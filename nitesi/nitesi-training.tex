\usepackage[utf8]{inputenc}
\usepackage[T1]{fontenc}
\usepackage{mathptmx}
\usepackage[scaled=.90]{helvet}
\usepackage{courier}
\usepackage{caption}
\captionsetup{labelformat=empty,labelsep=none}
\usepackage{beamerthemesplit}
\usepackage{verbatim}
\usepackage{hyperref}
\usepackage{listings}
\lstset{language=Perl,basicstyle=\normalsize,tabsize=3,showstringspaces=false}

\title{Nitesi Training}
\author[racke]{Stefan Hornburg (Racke)\\ \texttt{racke@linuxia.de}}
\date{eCommerce Innovation 2013, 8th October 2013}

\begin{document}
\maketitle{}

\begin{frame}
  \titlepage
\end{frame}

\tableofcontents

\section{Introduction}

\subsection{Targets}
\begin{frame}{Targets}
\begin{itemize}
\item Getting in touch with Nitesi
\item Implementing wishlists
\item Payment with PayPal
\item Product Filters with Solr
\end{itemize}
\end{frame}

\subsection{Separation}
\begin{frame}{Principle of Separation}
\begin{itemize}
\item Data
\item Template
\item Layout
\end{itemize}
\end{frame}

\begin{frame}[fragile]{ITL example}
\begin{lstlisting}
[fly-list code="[data session arg]"]
<div>
<p>[L]Description[/L]: [item-description]</p>
<p>[L]Price[/L]: [item-price]</p>
</div>
[query ...]
[/query]
[/fly-list]
\end{lstlisting}
\end{frame}

In this example we see a couple of problems.

\begin{frame}{ITL problems}
\begin{itemize}
\item Eierlegende Wollmilchsau
\item Code duplication
\item ITL deserts
\item Queries inside ITL
\item Interaction with designers
\end{itemize}
\end{frame}

\begin{frame}{Solutions}
\begin{itemize}
\item Fetch data inside code / embed with JSON
\item Better templating engine
\end{itemize}
\end{frame}

\begin{frame}{Solutions}
\begin{itemize}
\item Template::Flute
\item Data: Values, Iterators
\item Template: HTML, Specification
\end{itemize}
\end{frame}

\subsection{Template::Flute}
\subsubsection{Data}
\begin{frame}[fragile]{Data}
\begin{lstlisting}
$values{cart} = cart->items;
\end{lstlisting}
\end{frame}

\subsubsection{Specification}
\begin{frame}[fragile]{Specification}
\begin{lstlisting}
<specification>
<list name="cart" class="cartitem" iterator="cart">
<param name="name"/>
<param name="price" filter="currency"/>
<param name="qty" field="quantity"/>
</list>
</specification>
\end{lstlisting}
\end{frame}

\subsubsection{Layout}
\begin{frame}[fragile]{Layout}
\begin{lstlisting}
<tr class="cartitem">
   <td class="name">Simms replacement laces</td>
   <td>
       <input class="qty" value="2" />
   </td>
   <td class="td-right price">$24.55</td>
   <td class="td-right ">$49.10</td>
</tr>
\end{lstlisting}
\end{frame}

\section{Products}
\begin{frame}[fragile]{products Table}
\begin{itemize}
\item sku
\item name
\item price
\item uri
\item gtin
\item canonical\_sku
\item status
\item inactive
\end{itemize}
\end{frame}

There are more fields than that.

\begin{frame}[fragile]{Products}
\begin{lstlisting}
shop_product->create();
\end{lstlisting}
\end{frame}

\subsection{Variants}
\begin{frame}[fragile]{product\_attributes Table}
\begin{itemize}
\item code
\item sku
\item name
\item value
\item original\_sku
\end{itemize}
\end{frame}

\section{Project I: Implement wishlists}
\begin{frame}{Implement wishlists}
\begin{itemize}
\item Carts
\item Wishlists
\end{itemize}
\end{frame}

\subsection{Carts}
\begin{frame}{Cart}
\begin{itemize}
\item Multiple carts
\item Storage in session or database
\item Combines automatically
\item Price caching
\end{itemize}
\end{frame}

Interchange has a saved carts feature, but the way to store it is
really awkward.

\begin{frame}{Cart Validation}
\begin{itemize}
\item SKU, Name, Quantity, Price
\item Price > 0
\end{itemize}
\end{frame}

\begin{frame}[fragile]{Using cart keyword}
\begin{lstlisting}
use Dancer::Plugin::Nitesi;

cart->add(sku => 'POM253', name => 'Pomelo',
    price => 3.00, quantity => 10);

cart->remove(sku => 'POM253');

cart->count();

cart->clear();

cart->total();

cart->subtotal();
\end{lstlisting}
\end{frame}

\subsection{Wishlists}

\begin{frame}{Task I}
Add product to wishlist on flypage access
\end{frame}

\begin{frame}[fragile]{Wishlists}
\begin{lstlisting}
cart('wishlist')->add((sku => 'POM253', name => 'Pomelo',
    price => 3.00, quantity => 10);
\end{lstlisting}
\end{frame}

\begin{frame}{Shop break}
Slides:
\url{http://www.linuxia.de/talks/eic2013/nitesi-training-beamer.pdf}
\end{frame}

\end{document}

%%% Local Variables: 
%%% mode: latex
%%% TeX-master: t
%%% End: 
