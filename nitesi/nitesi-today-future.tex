\usepackage[utf8]{inputenc}
\usepackage[T1]{fontenc}
\usepackage{mathptmx}
\usepackage[scaled=.90]{helvet}
\usepackage{courier}
\usepackage{caption}
\captionsetup{labelformat=empty,labelsep=none}
\usepackage{beamerthemesplit}
\usepackage{verbatim}
\usepackage{hyperref}
\usepackage{listings}
\lstset{language=Perl,basicstyle=\normalsize,tabsize=3,showstringspaces=false}

\title{Nitesi Shop Machine Today and in the Future}
\author[racke]{Stefan Hornburg (Racke)\\ \texttt{racke@linuxia.de}}
\date{eCommerce Innovation 2013, Hancock, 9th October 2013}

\begin{document}
\maketitle{}

\begin{frame}
  \titlepage
\end{frame}

\tableofcontents

\begin{frame}{Nitesi Shop Machine}
\begin{itemize}
\item Motivation
\item Today
\item Future
\end{itemize}
\end{frame}

We grew more and more unsatisfied by with Interchange in the last years,
especially with the advent of Modern Perl and the slow development progress
compared to young and fast projects like Dancer.

We tried to alleviate these problems through WellWell, which introduced
hooks, plugins and proper database based carts.

\section{Motivation}
\begin{frame}{Motivation}
\begin{itemize}
\item Modern Perl
\item Plugins
\item Hooks
\item Engines
\end{itemize}
\end{frame}

\section{Today}
\begin{frame}{Today}
\begin{itemize}
\item Eco System
\item Navigation, Products, Carts
\item Routes
\item ...
\end{itemize}
\end{frame}
% sample shops

\subsection{Eco System}

\begin{frame}{Eco System}
\begin{itemize}
\item Modern Perl
\item Dancer
\item Template engine
\end{itemize}
\end{frame}

\begin{frame}{Dancer}
\begin{itemize}
\item PSGI
\item Keywords
\item Routes
\item Hooks
\item Engines
\item Plugins
\end{itemize}
\end{frame}


\begin{frame}{Template::Flute}
\end{frame}

\subsection{Navigation}
\begin{frame}{Navigation}
\begin{itemize}
\item Unlimited depth
\item Different types
\end{itemize}
\end{frame}

Unlimited depth: not recommended.
Different types: categories, menus

\subsection{Products}
\begin{frame}{Products}
\end{frame}

\subsubsection{Pricing}
\begin{frame}{Pricing}
\begin{itemize}
\item price field
\end{itemize}
\end{frame}

% Override with price method.

\subsubsection{Variants}
\begin{frame}[fragile]{Variants}
\begin{itemize}
\item Main product without attributes \\
      \verb|WBSF10043|
\item Each variant is a separate product \\
      \verb|WBSF10043-434-20| 
\begin{itemize}
\item size S 
\item color Sky Blue
\end{itemize}
\item GTIN (ISBN, EAN, UPC)
\item Data in product attributes
\end{itemize}
\end{frame}

For variants, we are usually creating one entry in products table per
product. This make sense as we will usually have a different GTIN/EAN/UPC
for each variant.

\subsection{Carts}
\begin{frame}{Cart}
\begin{itemize}
\item Multiple carts
\item Storage in session or database
\item Combines automatically
\item Price caching
\end{itemize}
\end{frame}

Interchange has a saved carts feature, but the way to store it is
really awkward.

\begin{frame}{Cart Validation}
\begin{itemize}
\item SKU, Name, Quantity, Price
\item Price > 0
\end{itemize}
\end{frame}

\begin{frame}[fragile]{Using cart keyword}
\begin{lstlisting}
use Dancer::Plugin::Nitesi;

cart->add(sku => 'POM253', name => 'Pomelo',
    price => 3.00, quantity => 10);

cart->remove(sku => 'POM253');

cart->count();

cart->clear();

cart->total();

cart->subtotal();
\end{lstlisting}
\end{frame}

\subsection{Routes}

Routes was an idea introduced by Ruby's Sinatra framework (which we often
credit) and takes a much cleaner and clearer approach to cut down on the
added hierarchy complexity that MVC can add.

Routes are basically paths the user takes which are attached to code that
will be triggered when a user reaches the specific route.

(from http://advent.perldancer.org/2010/2, SawyerX)

\begin{frame}{Routes}
\begin{itemize}
\item Routes in Web Frameworks
\item Interchange ``Routes''
\item Nitesi Routes
\end{itemize}
\end{frame}


\begin{frame}{Interchange ``Routes''}
\begin{itemize}
\item ActionMap
\item AutoPage
\item FlyPage
\item MissingPage
\end{itemize}
\end{frame}

ActionMap is a very limited route, as you can only
specify the first part of the path.

\begin{frame}[fragile]{Nitesi Routes}
\begin{description}
\item[Cart] \verb|/cart| 
\item[Checkout] \verb|/checkout|
\item[Navigation] \verb|/Citrus/Lemon|
\item[Product] \verb|/Citrus/Lemon/X1234| \verb|/X1234|
\end{description}
\end{frame}

Navigation and product are inside a catchall route.

\begin{frame}{Product/Flypage route}
\begin{itemize}
\item sku and uri fields
\item sku redirects to uri
\item 404 page for inactive products
\end{itemize}
\end{frame}

\begin{frame}[fragile]{Nitesi Routes}
\begin{lstlisting}
 package MyShopApp;

    use Dancer ':syntax';
    use Dancer::Plugin::Nitesi;
    use Dancer::Plugin::Nitesi::Routes;

    ...

    shop_setup_routes;
\end{lstlisting}
\end{frame}

\begin{frame}{Route Hooks}
\begin{itemize}
\item before\_product\_display
\item before\_cart\_display
\item before\_checkout\_display
\item before\_navigation\_display
\end{itemize}
\end{frame}

\subsection{Account}
\begin{frame}[fragile]{Accounts}
\begin{lstlisting}
post '/login' => sub {
    if (account->login(username => params('body')->{username},
                       password => params('body')->{password})) {
        redirect '/customerservice';
    }
    else {
        redirect '/login';
    }
};
\end{lstlisting}
\end{frame}

\begin{frame}[fragile]{Accounts}
\begin{lstlisting}
get '/mywishlist' => sub {
    if (account->acl(check => 'create_wishlists')) {
        return template 'mywishlist';
    }

    account->status(login_info => 'Please login to view wishlist.',
                    login_continue => 'mywishlist',
                   )

    redirect '/login';
};
\end{lstlisting}
\end{frame}

\subsubsection{Account Manager}
\begin{frame}{Account manager}
\begin{itemize}
\item Account Providers
\item Login/Logout
\item Account Information
\item Login status
\item Forgot password
\item Registration
\end{itemize}
\end{frame}

\subsubsection{Access Control}
\begin{frame}{Access Control}
\begin{itemize}
\item User
\item Roles
\item Permissions
\end{itemize}
\end{frame}

\subsection{Payment}

\begin{frame}{Payment}
\begin{itemize}
\item Business::OnlinePayment
\end{itemize}
\end{frame}

\begin{frame}[fragile]{Payment: charge Keyword}
\begin{lstlisting}
 $tx = charge(provider => 'Braintree',
                 amount => cart->total,
                 first_name => 'Test',
                 last_name => 'Tester',
                 card_number => '4111111111111111',
                 expiration => '0714',
                 cvc => '222');
\end{lstlisting}
\end{frame}

\begin{frame}[fragile]{Payment: Configuration}
\begin{lstlisting}
plugins:
  Nitesi:
    Payment:
      default_provider: Braintree
      providers:
        Braintree:
          merchant_id: sai8eicieki8aehe 
          public_key: vetheifaezaigh8u 
          private_key: oov8keiquiughie0
#         test_transaction: 1
\end{lstlisting}
\end{frame}

\subsection{Forms and Validation}
\begin{frame}{Forms and Validation}
\begin{itemize}
\item Dancer::Plugin::Form
\item Data::Transpose
\end{itemize}
\end{frame}

We use Dancer::Plugin::Form for form handling, which currently
only works with Template::Flute. The interesting feature is
that this can see the fields available in the HTML / specification.

\subsection{API}
\begin{frame}{API}
See presentation tomorrow!
\end{frame}

\subsection{Query}
\begin{frame}{Query}
\begin{itemize}
\item SQL::Abstract(::More)
\item reblessed \$dbh
\end{itemize}
\end{frame}

\section{Future}
\begin{frame}{Future}
\begin{itemize}
\item Features
\item Development
\item Demo
\item Admin
\item Dancer2
\end{itemize}
\end{frame}

\subsection{Features}
\begin{frame}{Features}
\begin{itemize}
\item Reviews
\item Gift Certificates, Coupons
\item Shipping
\end{itemize}
\end{frame}

\subsection{Development}
\begin{frame}{Development}
\begin{itemize}
\item Dancer::Plugin::Auth::Extensible
\item DBIx::Class
\end{itemize}
\end{frame}

\subsection{Demo}
\begin{frame}{Demo}
See talk tomorrow.
\end{frame}

\subsection{Admin}
\begin{frame}{Admin}

\end{frame}

\subsection{Dancer2}
\begin{frame}{Dancer2}
\begin{itemize}
\item Dancer2::Plugin::Nitesi
\end{itemize}
\end{frame}

\section{The End}

\begin{frame}{Shop break}
Slides:
\url{http://www.linuxia.de/talks/eic2013/nitesi-today-future-beamer.pdf}
\end{frame}

\end{document}

%%% Local Variables: 
%%% mode: latex
%%% TeX-master: t
%%% End: 
