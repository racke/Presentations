% TODO
% Wiki::Toolkit History, Author
% Functions
% Address for Slides

\usepackage{beamerthemesplit}
\usepackage{verbatim}
\usepackage{hyperref}
\usepackage{listings}
\lstset{language=Perl,basicstyle=\footnotesize}

% - Introduction
% - Motivation
% - Application

\title{Wikis with Wiki::Toolkit}
\author[racke]{Stefan Hornburg (Racke)\\ \texttt{racke@linuxia.de}}
\date[OPW2010]{Perl Oasis, 16th January 2010}

\begin{document}

\begin{frame}
  \titlepage
\end{frame}

\tableofcontents

\section{Preface}
Welcome to my presentation about Wiki::Toolkit, a toolkit for building
Wikis with Perl.



% Real World Applications
% - OpenGuides
% - ACT
\section{Architecture}

% Dispatch

\begin{frame}{Architecture}
 \begin{itemize}
  \item Storage
  \item Formatter(s)
  \item Plugins
  \item Search
  \item Glue
 \end{itemize}
\end{frame}

% \begin{frame}{Wiki Functions}
% \end{frame}

\section{Storage}

\subsection{Storage Backends}

All storage backends are derived from \verb+Wiki::Toolkit::Store::Database+.

\begin{frame}{Storage Backends}
 \begin{itemize}
  \item MySQL
  \item PostgreSQL
  \item SQLite
 \end{itemize}
\end{frame}

\subsection{Storage Setup}

% notes: cleardb, leaving tables alone
% cleardb doesn't erase tables for search backends

All storage backends can be easily initialized with the corresponding
\verb+Wiki::Toolkit::Setup+ classes.

\begin{frame}[fragile]
\frametitle{Storage Setup}
\lstinputlisting{scripts/setup_mysql}
\end{frame}

\subsection{Storage Tables}

\begin{frame}{Storage Tables}
 \begin{description}
  \item [content] Nodes
  \item [schema\_info] Schema Version
  \item [metadata] Metadata
  \item [internal\_links] Page Links
 \end{description}
\end{frame}

\subsection{Nodes and Metadata}

\subsubsection{Metadata}

% see also JSON plugin

Arbitrary metadata can be stored alongside with the node.

\begin{frame}[fragile]{Metadata}

\begin{lstlisting}
%metadata = {country => 'USA', 
    state => 'Florida',
    city => 'Orlando'};

$wiki->write_node($node, $content, $checksum, \%metadata, $requires_moderation);
\end{lstlisting}
\end{frame}

The metadata can be retrieved with the \verb+retrieve_node+ method.

\begin{lstlisting}
%node = $wiki->retrieve_node('Perl Oasis');

for (keys %{$node{metadata}}) {
    ...
}
\end{lstlisting}




\section{Formatters}

\subsection{Methods}

The interface for Wiki::Toolkit formatters is really simple, it consists
of a mandatory \verb+format+ and an optional \verb+find_internal_links+
method.

\verb+format+ takes the text in the Wiki format and returns HTML code
suitable for display in a web page.

\verb+find_internal_links+ returns internal links in the Wiki.

\begin{frame}[fragile]{Methods}
\begin{lstlisting}

my $html = $formatter->format($content);

my @links = $formatter->find_internal_links($content);

\end{lstlisting}
\end{frame}

\subsection{Constructor Options}

% see contructor for Wiki::Toolkit:Formatter::UseMod

\begin{frame}[fragile]{Constructor Options}
\begin{lstlisting}

my $formatter = Wiki::Toolkit::Formatter::Default->new(
                 extended_links  => 0,
                 implicit_links  => 1,
                 allowed_tags    => [qw(b i)],
                 macros          => {},
                 node_prefix     => 'wiki.cgi?node=' );

\end{lstlisting}
\end{frame}

Extended links are links marked by square brackets, e.g.
\verb+[[Page]|Link name]+.

Implicit links are links from CamelCase strings.

\subsection{Internal Links}

% Note: \verb+rename_node+ function uses internal_links table.

\subsection{Formatters in CPAN}

\begin{frame}{Formatters in CPAN}
 \begin{itemize}
  \item MediaWiki
  \item UseMod
  \item MarkDown
  \item POD
 \end{itemize}
\end{frame}

\subsection{Multiple Formatters}

\begin{frame}{Multiple Formatters}
\end{frame}

\subsection{Writing Formatters}

% HINT to Text::WikiFormat;

\begin{frame}{Writing Formatters}
\end{frame}

\section{Search}

\begin{frame}{Search}
\end{frame}

\section{Plugins}

% Wiki::Toolkit::Feed modules
% http://search.cpan.org/~dom/Wiki-Toolkit-0.78/lib/Wiki/Toolkit/Extending.pod

\begin{frame}[fragile]
\frametitle{Plugins}
\begin{lstlisting}
my $plugin = new Wiki::Toolkit::Plugin::Diff;
$wiki->register_plugin(plugin => $plugin);
%diff = $plugin->differences(node => 'FrontPage',
    left_version => 101,
    right_version => 105);
\end{lstlisting}
\end{frame}

\subsection{Plugins in CPAN}

\begin{frame}{Plugins in CPAN}
 \begin{description}
  \item[Diff] format differences between versions
  \item[Categorizer] category management
  \item[JSON] recent changes as JSON
  \item[Locator::Grid] manage co-ordinate data
  \item[Ping] ping various services on node updates
  \item[RSS::Reader] retrieve feeds for node inclusion
 \end{description}
\end{frame}

\section{Wiki Syntax}
\subsection{Markdown}
Homepage: \url{http://daringfireball.net/projects/markdown/syntax}

Wiki: \url{http://markdown.infogami.com/}

\subsubsection{Emphasis}

\begin{tabular}{ll}
\verb|*single asterisks*| & \textit{italic} \\
\verb|_single underscores_| & \textbf{bold} \\
\end{tabular}

% TODO: double asterisks / double underscores

\subsubsection{Links}
Internal links:

\begin{verbatim}
See my [About](/about/) page for details.
\end{verbatim}

\subsection{UseMod}
Homepage: \url{http://www.usemod.com/cgi-bin/wiki.pl}
Syntax: \url{http://www.usemod.com/cgi-bin/wiki.pl?TextFormattingRules}

\subsubsection{Emphasis}

\section{Resources}

Homepage: \url{http://www.wiki-toolkit.org/}

Mailing list: \url{http://www.earth.li/cgi-bin/mailman/listinfo/cgi-wiki-dev}

\subsection{Related Projects}
MojoMojo, a Catalyst \& DBIx::Class powered Wiki,
\url{http://search.cpan.org/perldoc?MojoMojo}.

\end{document}
    
%%% Local Variables: 
%%% mode: latex
%%% TeX-master: t
%%% End: 
