%% -*- mode: latex; -*-

\section{Spamassassin, Antivirus und Email via Perl}

\subsection*{Autor}
Stefan Hornburg (Racke) \verb/<racke@linuxia.de>/

\subsection*{Bio Stefan Hornburg}

\subsection*{Abstract}
Die Bedeutung von Email f�r die Kommunikation hat einen hohen
Stellenwert erreicht und ist oftmals nicht aus dem t�glichen
Leben und aus der Arbeitswelt wegzudenken. Attacken auf dieses
wichtige Instrument mittlels SPAM, Viren und Phishing nehmen
kontinuierlich zu. N�tzliche Emails drohen in einer Flut von unerw�nschter
und potentiell sch�dlichen Post unterzugehen, wenn Gegenma�nahmen
nicht oder nur unzureichend greifen.

F�r Perl sind eine Reihe von Modulen vorhanden, die die Erstellung
von Programmen zum Zugriff auf Email und die Bewertung der Email
erleichtern:

	Net::SMTP (Emailversand)
	Mail::IMAPClient (Zugriff auf Email mit IMAP)
	Mail::Spamassassin (Spamassassin)
	ClamAV::Client (Clam Antivirus)
	
Nach einer kurzen Einf�hrung in die Benutzung dieser Module besch�ftigt
sich der Hauptteil des Vortrags mit der Beschreibung von Anwendungen
zur Verbesserung der Abwehr von SPAM, Viren und Phishing.

Zur Demonstration werden Emails von einem IMAP-Server heruntergeladen 
und von ClamAV auf Viren und Phishing gepr�ft und anschlie�end durch
SpamAssassin bewertet. Anschlie�end wird gezeigt, wie diese Emails vom
Server gel�scht oder in bestimmte Verzeichnisse auf dem IMAP-Server verschoben
werden k�nnen. Die Behandlung von Fehlerkennungen (False Positives)
wird ebenso erl�utert.

Im Anschlu� wird die Erstellung von Statistiken �ber das Aufkommen
von Viren und SPAM besprochen und wie man daraus Schl�sse zur Optimierung
zur Abwehr treffen kann.

\subsection{Mailversand mit Net::SMTP}
\begin{verbatim}
use Net::SMTP;

my $smtp = new Net::SMTP;
\end{verbatim}

\subsection{Mailempfang mit Mail::IMAPClient}

Eine IMAP-Verbindung kann einfach durch die Erzeugung eines
\verb/Mail::IMAPClient/-Objektes
mit den geeigneten Parametern hergestellt werden. Daf�r m�ssen
zumindestens der Servername (Server), Benutzername (User) und Passwort
(Password) �bergeben werden. Dann erfolgt der Verbindungsaufbau und die
Anmeldung automatisch.

\begin{verbatim}
use Mail::IMAPClient;

my $params = (Server => $server,
              User => $login,
              Password => $password);

my $client = new Mail::IMAPClient (%params);
\end{verbatim}

Dies ist nicht moeglich, wenn man eine SSL-gesch�tzte Verbindung aufbauen
moechte. In diesem Fall erzeugt man die Verbindung mit dem
\verb/IO::Socket::SSL/-Modul: 

\begin{verbatim}
use IO::Socket::SSL;
use Mail::IMAPClient;

my $conn;

unless ($conn = new IO::Socket::SSL->new ("${server}:imaps")) {
    die "$0: imaps connection to $server failed: $!\n";
}

$imap = new Mail::IMAPClient (Socket => $conn,
							  User => $user,
							  Password => $password);
\end{verbatim}

Unbekannte Methoden werden direkt in IMAP-Befehle umgesetzt.

\subsection{Virenscanner mit ClamAV::Client}
Das Modul \WSIndex{ClamAV::Client} bietet eine Schnittstelle zum ClamAV
Virenscanner. Beim Anlegen des Objektes wird eine Verbindung zum 
\verb/clamd/-Daemon aufgebaut.
\begin{verbatim}
use ClamAV::Client;

my $scanner;

$scanner = new ClamAV::Client;

unless (defined $scanner && $scanner->ping()) {
	die "$0: connection to ClamAV daemon failed\n";
}
\end{verbatim}
Mit den Parametern \verb/socket_name/, \verb/socket_host/,
\verb/socket_port/ wird die Art der Verbindung (TCP oder Unixsocket)
und die Verbindungsparameter definiert.

Mit Hilfe der \verb/version/-Methode kann man bestimmen, ob die Virusengine
und die Virusdatenbank von ClamAV auf dem neuesten Stand ist:

\begin{verbatim}
\end{verbatim}

\subsection{Spamfilter mit Mail::Spamassassin}
Eine Email kann wie folgt mit \WSIndex{Mail::Spamassassin} untersucht werden:

\begin{verbatim}
# initiate Spamassassin object
$sa = new Mail::SpamAssassin;
# retrieve a message from the IMAP server
$message = $imap->message_string($msgno);
# let SpamAssassin parse the message
$samsg = $sa->parse($message);
$status = $sa->check($samsg);
# score and matched rules
$sascore = $status->get_hits();

# SPAM or HAM 
\end{verbatim}

\subsection{Scenarios}
\subsubsection{Bayes}
Update Bayes from FN/FP.
Die Emaildateien koennten auch direkt aus dem Dateisystem gelesen werden,
aber bei Zugriff mit IMAP brauchen wir keine Annahmen ueber die 
Verzeichnisstruktur zu machen bzw. koennen Mailserver und SA-Server auf
verschiedenen Rechnern sein.

Problem: verbraucht viel RAM

\subsubsection{Virus}
Mit einer lokalen Installation von ClamAV koennen Postfaecher auf anderen
Servern auf Malware untersucht werden.

Selbst bei einem geschuetzten 
Scan mails on remote server and delete them

\subsubsection{Spamassassin}
Scan mails on remote server and move them, idea: better SPAM fighting
from local machine, especially on new waves.

\subsection{Emails loeschen oder verschieben}

Das Loeschen von Emails auf einen IMAP-Server kann man nicht mit dem
Loeschen einer Datei auf einem Linux-System vergleichen. Emails werden
zunaechst nur zum Loeschen markiert. 

Die \verb/delete_message/-Methode fuehrt das folgende IMAP-Kommando aus:
\verb/UID store 3,4,5,6,7 +FLAGS.SILENT (\Deleted)/. Die Emails sind
weiterhin sichtbar. Erst durch das Schliessen des IMAP-Ordners oder
dem expliziten \verb/expunge/ werden die Emails physikalisch geloescht.

\begin{thebibliography}{99}

\end{thebibliography}

