\usepackage[utf8]{inputenc}
\usepackage[T1]{fontenc}
\usepackage{mathptmx}
\usepackage[scaled=.90]{helvet}
\usepackage{courier}
\usepackage{caption}
\captionsetup{labelformat=empty,labelsep=none}
\usepackage{beamerthemesplit}
\usepackage{verbatim}
\usepackage{hyperref}
\usepackage{listings}
\lstset{language=Perl,basicstyle=\normalsize,tabsize=3,showstringspaces=false}

\title{Dancer und DBIx::Class}
\author[racke]{Stefan Hornburg (Racke)\\ \texttt{racke@linuxia.de}}
\date{16. Deutscher Perl-Workshop, Hannover, 27. März 2013}

\begin{document}
\maketitle{}

\begin{frame}
  \titlepage
\end{frame}

\tableofcontents

\begin{frame}{Übersicht}
\begin{itemize}
\item Einführung
\item DBIC Schemas mit Dancer Plugin
\item DBIC session engine
\item Tabelleneditor
\end{itemize}
\end{frame}

Wir starten mit einer kurzen Einführung von DBIx::Class.

\section{Einführung}
\begin{frame}{Datenbank}
\begin{itemize}
\item Datenbank
\item Tabellen
\item Datensätze
\end{itemize}
\end{frame}

\begin{frame}{DBIx::Class}
\begin{itemize}
\item Schema
\item ResultSet
\item Row / Objekt
\end{itemize}
\end{frame}

\section{Dancer::Plugin::DBIC}
\begin{frame}[fragile]{DBIx::Class ohne Dancer Plugin}
\begin{lstlisting}
use Interchange6::Schema;

$schema = Interchange6::Schema->connect($testdb->connection_info);

$schema->resultset('User')->search({..});
\end{lstlisting}
\end{frame}

\begin{frame}[fragile]{DBIx::Class mit Dancer Plugin}
\begin{lstlisting}
use Dancer::Plugin::DBIC;

schema->resultset('User')->search({..});

resultset('User')->search({..});

rset('User')->search({..});
\end{lstlisting}
\end{frame}

\section{Dancer::Session::DBIC}

\section{Table Editor}

\section{Dancer2}

Was ist mit Dancer2 ?

Für Dancer2 existiert bereits ein Plugin:

\url{https://metacpan.org/pod/Dancer2::Plugin::DBIC}

Die Sessionengine und der Tabelleneditor wurde noch nicht auf Dancer2 portiert.

\begin{frame}{Tanzpause}
Slides:
\url{http://www.linuxia.de/talks/pws2014/dancer-dbic-de-beamer.pdf}
\end{frame}

\end{document}

%%% Local Variables: 
%%% mode: latex
%%% TeX-master: t
%%% End: 
