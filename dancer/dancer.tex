\usepackage[T1]{fontenc}
\usepackage{mathptmx}
\usepackage[scaled=.90]{helvet}
\usepackage{courier}

\usepackage{beamerthemesplit}
\usepackage{verbatim}
\usepackage{hyperref}
\usepackage{listings}
\lstset{language=Perl,basicstyle=\footnotesize,tabsize=3,showstringspaces=false}

\title{Dancer Academy - From Zero to Hero}
\author[racke]{Stefan Hornburg (Racke)\\ \texttt{racke@linuxia.de}}
\date{Swiss Perl Workshop 2013, Bern, 22nd March}

\begin{document}
\maketitle{}

\begin{frame}
  \titlepage
\end{frame}

\tableofcontents

%\section{Hintergrund und Projekte}

%\subsection{Tanzflur}
\begin{frame}{Tanzflur}
\begin{itemize}
\item einfach
\item Anwendung auf Knopfdruck
\end{itemize}
\end{frame}

\begin{frame}{Tanzflur}
\begin{itemize}
\item effektiv
\item Routes und Schl�sselw�rter
\end{itemize}
\end{frame}

\begin{frame}{Tanzflur}
\begin{itemize}
\item erweiterbar
\item Plugins
\end{itemize}
\end{frame}

\begin{frame}{Tanzflur}
\begin{itemize}
\item flexibel
\item Engines
\end{itemize}
\end{frame}

% \begin{frame}{Projekte}
% \begin{itemize}
% \item Dropbox-Klon
% \item LDAP Benutzerverwaltung
% \item Procurement f�r American Corners
% \end{itemize}
% \end{frame}

% \subsection{Dropbox}
% F�r die Firma Caithness in New York habe ich einen Dropbox-Klon
% entwickelt.

% Dropbox demonstriert die automatische Erkennung des MIME-Typs
% durch Dancer.

% \begin{frame}{Dropbox}
% \begin{itemize}
% \item Autoindex
% \item Upload/Download
% \item Benutzerverwaltung
% \end{itemize}
% \end{frame}

% \subsection{LDAP Benutzerverwaltung}
% \begin{frame}{LDAP Benutzerverwaltung}
% \begin{itemize}
% \item LDAP-Verzeichnis
% \item Benutzerverwaltung
% \begin{itemize}
% \item Adminstrator
% \item Referrer
% \item Patron
% \end{itemize}
% \end{itemize}
% \end{frame}

% \subsection{Procurement f�r American Spaces}
% \begin{frame}{Procurement f�r American Spaces}
% \begin{itemize}
% \item Warenkorb/Wunschliste
% \item Genehmigungsprozesse
% \item LibraryThing
% \item ISBNDB
% \end{itemize}

% \url{https://eshop.state.gov/}
% \end{frame}

\section{Quickstart}

\begin{frame}[fragile]{Quickstart}
\begin{itemize}
\item cpanm Dancer
\item cpanm YAML
\item dancer -a Locker
\item cd Locker
\item ./bin/app.pl
\item x-www-browser \url{http://localhost:3000/}
\end{itemize}
\end{frame}

\begin{frame}[fragile]{Programm}
\begin{lstlisting}
#!/usr/bin/env perl
use Dancer;
use Locker;
dance;
\end{lstlisting}
\end{frame}

\begin{frame}[fragile]{Modul}
\begin{lstlisting}
package Locker;
use Dancer ':syntax';

our $VERSION = '0.1';

get '/' => sub {
    template 'index';
};

true;
\end{lstlisting}
\end{frame}

\begin{frame}[fragile]{Template}
\begin{lstlisting}
views/index.tt
views/layouts/main.tt
\end{lstlisting}
\end{frame}

\subsection{Filebrowser}
\begin{frame}{Filebrowser}
\begin{itemize}
\item Liste der Dateien
\item Template
\item Route
\end{itemize}
\end{frame}

\begin{frame}[fragile]{Liste der Dateien}
\begin{lstlisting}
use Dancer::Plugin::Autoindex;

$files = autoindex('/');
\end{lstlisting}
\end{frame}

\begin{frame}[fragile]{Template}
\lstinputlisting{views/filebrowser.tt}
\end{frame}

\begin{frame}[fragile]{Route}
\begin{lstlisting}
get '/home' => sub {
    my $files = autoindex('/');

    template 'filebrowser', {directory => 'Home',
                             files => $files,
                            };
};
\end{lstlisting}
\end{frame}

\begin{frame}[fragile]{Konfiguration}
Standard config.yml:
\begin{lstlisting}
# template engine
# simple: default and very basic template engine
template: "simple"
\end{lstlisting}
Locker config.yml:
\begin{lstlisting}
template: "template_toolkit"
\end{lstlisting}
\end{frame}

\section{Templates}
\begin{frame}{Templates}
\begin{itemize}
\item Layout
\item Content
\end{itemize}
\end{frame}

Das Hash geht in Layout + Content.

\begin{frame}[fragile]{Templates}
\begin{lstlisting}
template 'index', {name => 'Test'}
\end{lstlisting}
\end{frame}

\section{Routes}

F�r eine Route ben�tigen wir

\begin{itemize}
\item HTTP-Methode
\item Pfad
\item Subroutine
\end{itemize}

Der Pfad f�r eine Route kann in einer der folgenden
Weisen angegeben werden.

\begin{frame}{Routes}
\begin{itemize}
\item String
\item String mit benannten Parametern
\item String mit Wildcards 
\begin{itemize}
\item Splat
\item Megasplat
\end{itemize}
\item Regul�rer Ausdruck
\item Regul�rer Ausdruck mit Captures
\end{itemize}
\end{frame}

\subsection{String}
\begin{frame}[fragile]{String}
\begin{lstlisting}
get '/home' => sub {
    my $files = autoindex('/');

    template 'filebrowser', {directory => 'Home',
                             files => $files,
                            };
};
\end{lstlisting}
\end{frame}

\subsection{Named parameters}
\begin{frame}[fragile]{Named parameters}
\begin{lstlisting}
get '/home/:file' => sub {
    my $files = autoindex(param('file'));

    template 'filebrowser', {directory => param('file'),
                             files => $files,
                            };
};
\end{lstlisting}
\end{frame}

\subsection{Splat}
\begin{frame}[fragile]{Splat}
\begin{lstlisting}
get '/images/covers/*.jpg' => sub {
    my ($isbn) = splat;

    if (-f "public/images/covers/$isbn.jpg") {
        return send_file "images/covers/$isbn.jpg";
    }

    status 'not_found';
    forward 404;
}
\end{lstlisting}
\end{frame}

\subsection{Megasplat}
Die einfache Wildcard matcht nur auf einen Teil des Pfads,
d.h. bis zum n�chsten Schr�gstrich (Slash).

Mit der doppelten Wildcard (Megasplat) wird einfach der 
Rest des Pfades gematcht und die \verb|splat|-Funktion
gibt eine Liste zur�ck.

\begin{frame}[fragile]{Megasplat}
\begin{lstlisting}
https://vsc.state.gov/lostpwd/biz@linuxia.de/e642bd543b9907bd2c06aa485261cb1a849a9f23fc7324bff45ebd35f4efe2cb

get '/lostpwd/**' => sub {
    my ($email, $hash) = splat;

    form->fill(email => $email,
               hash => $hash);
    
    template('lostpwd_confirm', form => $form);
}
\end{lstlisting}
\end{frame}

\subsection{Regular Expression}

\begin{frame}[fragile]{Regular Expression}

Catch-All (letzte Route!)

\begin{lstlisting}
any qr{.*} => sub {
    ...
};
\end{lstlisting}
\end{frame}

\subsection{Regular Expression with Captures}

\begin{frame}[fragile]{Regular Expression with Captures}
\begin{lstlisting}
https://dropbox.nite.si/~racke/talks/dancer-beamer.pdf

any qr{^/~(?<user>[^/]+)/(?<file>.*?)/?$} => sub {
    my ($capts, $user, $file);

    $capts = captures;
    $file = $capts->{file};
    $user = $capts->{user};

    ...
};
\end{lstlisting}
\end{frame}

\section{Download/Upload}
\subsection{Download}
\begin{frame}[fragile]{File Download}
\begin{lstlisting}
get '/home/**' => sub {
    my ($parts) = splat;
    my $path_relative = join('/', @$parts);
    my $path = autoindex_file_path($path_relative);

    if (-f $path) {
        return send_file $path, system_path => 1;
    }
    elsif (-d $path) {
        return template 'filebrowser', {directory => pop(@$parts),
                               files => autoindex($path_relative)};
    }
};
\end{lstlisting}
\end{frame}

\subsection{Upload HTML Form}
\begin{frame}[fragile]{Upload HTML-Form}
\begin{lstlisting}
<div class="upload">
<h2>Upload file</h2>
<form name="upload" id="upload_form" action="" method="post"
enctype="multipart/form-data">
<input type="file" name="upload_file"><br>
<input type="submit" value="OK">
</form>
</div>
\end{lstlisting}
\end{frame}

Mit dem upload-Schl�sselwort erhalten wir ein
Dancer::Request::Upload-Objekt.

\subsection{Upload Route}
\begin{frame}[fragile]{Upload Route}
\begin{lstlisting}
post '/home/**' => sub {
    my ($parts) = splat;
    my $path_relative = join('/', @$parts);
    my $path = autoindex_file_path($path_relative);

    if ($upload_file = upload('upload_file')) {
        # upload file to directory
        my $target_file = join('/', $path, $upload_file->{filename});
		
        unless ($upload_file->copy_to($target_file)) {
            die 'Upload failed';
        }
    }

    redirect "/home/$path_relative";
};
\end{lstlisting}
\end{frame}

% \subsection{Redirect/Forward}
% \begin{frame}[fragile]{Redirect/Forward}
% \begin{lstlisting}
% redirect "~$user->{username}/";

% \end{lstlisting}
% \end{frame}

\section{Hooks}

The before hook is called after a request has been accepted
and before routes are matched.

\begin{frame}[fragile]{before Hook}
Passwort protected site:
\begin{lstlisting}
hook 'before' => sub {
    unless (session('user')
        || request->path eq '/login'
        || request->path =~ m%^/lostpwd%
        ) {
        redirect '/login';
    }
};
\end{lstlisting}
Files in public aren't protected!
\end{frame}

\begin{frame}[fragile]{before\_template\_render Hook}
\begin{lstlisting}
hook before_template_render => sub {
    my $tokens = shift;
    my $menu;

    if (session('user')) {
        $menu = [{name => 'Logout', url => '/logout'}];
    }
    else {
        $menu = [{name => 'Login', url => '/login'}];
    }

    $tokens->{menu} = $menu;
};
\end{lstlisting}
\end{frame}

Die folgenden Hooks existieren in Dancer:

\begin{itemize}
\item before\_deserializer
\item before\_file\_render
\item before\_error\_init
\item before\_error\_render
\item before\_template\_render
\item before\_layout\_render
\item before\_serializer
\item after\_deserializer
\item after\_file\_render
\item after\_template\_render
\item after\_layout\_render
\item after\_error\_render
\end{itemize}

Der after\_file\_render-Hook wird ausgel�st, nachdem eine statische
Datei (Bild oder CSS-Datei) gesendet wurde.

% \section{Exceptions}
% Exceptions in Dancer sind nicht objektorientiert, um sie
% leichtgewichtig und schnell zu halten.
% \begin{frame}{Exceptions}
% \end{frame}

\section{Fehlerbehandlung}
\subsection{500}
\begin{frame}[fragile]{Fehlerbehandlung: 500}
\begin{itemize}
\item Datei: \verb|public/500.html|
\item Template: \verb|error_template|
\item Stacktrace: \verb|show_errors|
\end{itemize}
\end{frame}

\subsection{404}
\begin{frame}[fragile]{Fehlerbehandlung: 404}
\begin{itemize}
\item \verb|public/404.html|
\item 
\begin{lstlisting}
any qr{.*} => sub {
    status 'not_found';
    template 'my_404', { path => request->path };
};
\end{lstlisting}
\item XSS vermeiden
\end{itemize}
\end{frame}

\section{Plugins}
\begin{frame}{Plugins}
\begin{itemize}
\item Features
\item Keywords
\item Routes 
\item Configuration
\end{itemize}
\end{frame}

\subsection{Plugins on CPAN}
\begin{frame}[fragile]{Plugins on CPAN}
\begin{itemize}
\item Count: more than 100
\item Overview
\url{https://metacpan.org/module/Dancer::Plugins}
\item Search
\url{https://metacpan.org/search?q=dancer+plugin}
\end{itemize}
\end{frame}

\begin{frame}[fragile]{Plugins on CPAN}
\begin{itemize}
\item Dancer::Plugin::Database \\
\verb|database|
\item Dancer::Plugin::Email \\
\verb|email|
\end{itemize}
\end{frame}

\begin{frame}[fragile]{Dancer::Plugin::Database Beispiel}
\begin{lstlisting}
my $user = database->quick_select('users',
    {username => params->{'username'}});

if ($user) {
    session user => params->{'username'};
}
\end{lstlisting}
\end{frame} 

\begin{frame}[fragile]{Dancer::Plugin::Database Configuration}
\begin{lstlisting}
plugins:
  Database: 
    driver: Pg
    database: dropbox
    username: vienna
    password: nevairbe
\end{lstlisting}
\end{frame} 

\begin{frame}[fragile]{More SQL Plugins}
\begin{itemize}
\item Dancer::Plugin::DBIC
\item Dancer::Plugin::SimpleCRUD
\item ...
\end{itemize}
\end{frame} 

\begin{frame}[fragile]{Dancer::Plugin::Email Example}
\begin{lstlisting}
$message = template('lostpwd_email', 
       {password => $password, url => $url},
       {layout => undef});
	
email ({type => 'html',
        from => 'service@linuxia.de',
        to => param('email'),
        subject => 'Forgot password',
        message => $message});
\end{lstlisting}
\end{frame} 

\begin{frame}[fragile]{Dancer::Plugin::Email Redirect}
\begin{itemize}
\item Modul:

\verb|Email::Sender::Transport::Redirect|
\item Configuration:
\begin{lstlisting}
plugins:
  Email:
    transport:
      Sendmail:
        redirect_address: biz@icdev.de
\end{lstlisting}
\end{itemize}
\end{frame}

% thin wrapper around CPAN module
% add keywords (avoid overlaps)
% add configuration

% \subsection{Writing Plugins}
% \begin{frame}{Writing Plugins}
% \begin{itemize}
% \item register
% \item register\_plugin
% \item plugin\_setting
% \end{itemize}
% \end{frame}

% \subsection{Plugins and Hooks}
% \begin{frame}[fragile]{Plugins and Hooks}
% \begin{lstlisting}
% register_hook('before_cart_add');

% execute_hook('before_cart_add', ...);
% \end{lstlisting}
% \end{frame}

\section{Deployment}
\subsection{Sessions}
Die Session-Engine wird mit \verb|session| konfiguriert,
ansonsten werden keine Sessions verwendet. Cookies sind
Voraussetzung f�r Sessions mit Dancer.

YAML-Sessions und JSON-Sessions sind nur f�r die Entwicklung gedacht

Alternativen sind u.a.:
\begin{itemize} 
\item Storable
\item Cookie
\item Memcached
\item MongoDB
\end{itemize}

Eine gute Idee ist es die Session Engine f�r den Produktionsbetrieb
in \verb|config.yml| zu konfigurieren und f�r die Entwicklung in
\verb|environments/development.yml| zu �berschreiben.

Im Produktionsbetrieb ist es empfehlenswert die Ablauffrist f�r
Sessions zu setzen, sonst ist eine Session f�r immer g�ltig.

\begin{frame}[fragile]{Sessions}
\begin{itemize} 
\item Engine \\
\verb|session: Storable|
\item Directory \\
\verb|session_dir: /var/run/dancer-sessions|
\item Expiration \\
\verb|session_expires: 8 hours|
\end{itemize} 
\end{frame}

\subsection{Logs}
Die Logger-Engine wird mit \verb|logger| konfiguriert,
ansonsten werden Logs erzeugt.

Das Format f�r die Logs wird mit \verb|logger_format| definiert, siehe
\url{http://search.cpan.org/perldoc?Dancer::Logger::Abstract#logger_format}.

Verf�gbare Engines sind u.a.:
\begin{itemize} 
\item console
\item file
\item syslog *
\item log4perl *
\end{itemize}

Die Rotation der Logs f�r dateibasierte Engines darf nicht vergessen
werden im Produktionsbetrieb.

\begin{frame}[fragile]{Logs}
\begin{itemize} 
\item Engine \\
\verb|logger: file|
\item Directory \\
\verb|log_path: log|
\item Format \\
\verb|logger_format: "%t [%P] %L @%D> %m in %f l. %l"|
\end{itemize} 
\end{frame}

\subsection{Szenarien}
Die m�glichen Szenarien f�r ein Deployment von Dancer findet man
in der Dancer-Dokumentation:
\url{http://search.cpan.org/perldoc?Dancer::Deployment}.

\begin{frame}[fragile]{Szenarien}
\begin{itemize} 
\item Standalone \\
  \verb|./bin/app.pl|
\item CGI, FastCGI
\item plackup
\begin{itemize}
\item Starman
\item Twiggy
\item Corona
\end{itemize}
\item Reverse Proxy
\end{itemize} 
\end{frame}

\begin{frame}[fragile]{Szenarien: Reverse Proxy}
\begin{itemize}
\item Server
\begin{itemize} 
\item Apache
\item nginx
\item Perlbal
\end{itemize}
\item Dancer Configuration \\
\verb|behind_proxy: 1|
\end{itemize} 
\end{frame}

\subsection{Starman}
\begin{frame}[fragile]{plackup and Starman}
\begin{itemize} 
\item Command line
\begin{lstlisting}
plackup -E production 
        -s Starman \ 
        --workers=5 \
        --pid /home/racke/Dropbox/run/Dropbox.pid \
        -p 5000 \ 
        -a bin/app.pl \
        -D
\end{lstlisting}
\item Init script
\begin{itemize} 
\item \verb|Daemon::Control|
\end{itemize} 
\end{itemize} 
\end{frame}

\subsection{Perlbal}
Am Anfang der Konfigurationsdatei laden wir das ``vpaths''
Plugin.

Dann binden wir f�r bessere Performache
\verb|http://search.cpan.org/perldoc?Perlbal::XS::HTTPHeaders|
ein.

\begin{frame}[fragile]{Perlbal}
\begin{lstlisting}
LOAD vpaths
XS enable headers
\end{lstlisting}
\end{frame}

\subsubsection{Pools}
F�r unsere beiden Projekte erzeugen wir zun�chst Pools:

\begin{frame}[fragile]{Perlbal: Pools}
\begin{lstlisting}
CREATE POOL prod_starman_dosqua
 POOL prod_starman_dosqua ADD 127.0.0.1:5000

CREATE POOL prod_starman_vsc
 POOL prod_starman_vsc ADD 127.0.0.1:5001
\end{lstlisting}
\end{frame}

\begin{frame}[fragile]{Perlbal: Reverse Proxy}
\begin{lstlisting}
CREATE SERVICE vsc_prod
 SET role                 = reverse_proxy
 SET pool                 = prod_starman_vsc
 SET buffer_uploads       = on
ENABLE vsc_prod
\end{lstlisting}
\end{frame}

\begin{frame}[fragile]{Perlbal: Selector}
\begin{lstlisting}
CREATE SERVICE vsc_selector
 SET listen              = 86.59.13.238:80
 SET role                = selector
 SET plugins             = vpaths
 VPATH .*                = vsc_prod
ENABLE vsc_selector
\end{lstlisting}
\end{frame}

\begin{frame}[fragile]{Perlbal: SSL Selector}
\begin{lstlisting}
CREATE SERVICE vsc_ssl_selector
 SET listen              = 86.59.13.238:443
 SET role                = selector
 SET plugins             = vpaths
 SET enable_ssl          = on
 SET ssl_key_file        = /etc/ssl/private/vsc.state.gov.key
 SET ssl_cert_file       = /etc/ssl/certs/vsc.state.gov.pem
 VPATH .*                = vsc_prod
ENABLE vsc_ssl_selector
HEADER vsc_ssl_selector INSERT X-Forwarded-Proto: HTTPS
\end{lstlisting}
\end{frame}

\subsection{Skripts}
Dancer-Skripts sollten innerhalb der Dancer-Applikation
liegen, gut geeignet ist das bin/-Verzeichnis. Ein symbolischer
Link is ausreichend. Ansonsten wird die Konfiguration nicht gefunden.

Eine Ausgabe von Meldungen auf der Konsole ist w�nschenswert, mit
\verb|%m| wird lediglich die reine Meldung ausgegeben.
 
\begin{frame}[fragile]{Scripts}
\begin{lstlisting}
use Dancer ':script';

set logger => 'console';
set logger_format => '%m';
\end{lstlisting}
\end{frame}

% \subsection{Tests}

% \begin{frame}[fragile]{Tests}
% \begin{lstlisting}
% use Test::More tests => 42;
% use Dancer::Test;

% use Dancer ':tests';
% \end{lstlisting}
% \end{frame}

\section{Ausblick}

\subsection{Dancer 2}
\begin{frame}{Dancer2}

\begin{itemize} 
\item keine globalen Variablen
\item 100\% OO Backend (Moo)
\item Scoping for Sub-Applikationen
\item �berarbeitete Architektur
\item YAML => Config::Any 
\end{itemize}

\end{frame}

\subsection{Community}
\begin{frame}[fragile]{Community}
Web: \url{http://perldancer.org/} 

Github: \url{git://github.com/sukria/Dancer.git}

IRC: \#dancer @ irc.perl.org

Mitarbeit: Dancer::Development
\end{frame}

\begin{frame}{The End}
Slides:
\url{http://www.linuxia.de/talks/spw2013/dancer-beamer.pdf}
\end{frame}
\end{document}
