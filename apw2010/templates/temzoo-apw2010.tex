
\usepackage[T1]{fontenc}
\usepackage{mathptmx}
\usepackage[scaled=.90]{helvet}
\usepackage{courier}

\usepackage{beamerthemesplit}
\usepackage{verbatim}
\usepackage{hyperref}
\usepackage{listings}
\lstset{language=Perl,basicstyle=\footnotesize,tabsize=3,showstringspaces=false}

\title{Template::Zoom}
\author[racke]{Stefan Hornburg (Racke)\\ \texttt{racke@linuxia.de}}
\date[APW2010]{Austrian Perl Workshop 2010, 6th November 2010}

\begin{document}

\begin{frame}
  \titlepage
\end{frame}

\tableofcontents

\section{Templates}
Template::Zoom ist eine neue Templateengine, bei der im Gegensatz zu
Template::Toolkit das eigentliche Template aus reinem HTML besteht und somit
leicht von HTML-Designern bearbeitet werden kann.

Die Verkn�pfung zu Objekten, Listen und Datenbankabfragen erfolgt �ber eine
Spezifikationsdatei.

\subsection{Why}
\begin{frame}{Why Templates? Why Template::Zoom?}

 \begin{itemize}  
  \item Consistency of layout
  \item Reusability (components)
  \item Separation of layout and application logic
  \item Division of labor

%  \item Einarbeitungszeit f�r Designer reduzieren
%  \item Grenzf�lle, Fehler und Bedingungen (dynamische Seiten)
 \end{itemize}

\url{http://perl.apache.org/docs/tutorials/tmpl/comparison/comparison.html}
\end{frame}

\section{Example: Cart}
\subsection{Cart as Hash}
\begin{frame}[fragile]{Cart: Hash}
\begin{lstlisting}
$cart = [
         {isbn => '978-0-2016-1622-4', 
          title => 'The Pragmatic Programmer',
          quantity => 1, price => 49.95},

         {isbn => '978-1-4302-1833-3',
          title => 'Pro Git',
          quantity => 1, price => 34.99},
		];
\end{lstlisting}
\end{frame}
\subsection{HTML Template}
\begin{frame}[fragile]{Cart: HTML Template}
\begin{lstlisting}
<table class="cart">
<tr class="cartheader">
<th>Name</th>
<th>Quantity</th>
<th>Price</th>
</tr>
<tr class="cartitem">
<td class="name">Perl 6</td>
<td><input class="quantity" name="quantity" size="3" value="10"></td>
<td class="price">$$$</td>
</tr>
</table>
\end{lstlisting}
\end{frame}
\subsection{Cart with ITL}
\begin{frame}[fragile]{Cart: ITL}
\lstinputlisting{cart-example-basic/cart.itl}
\end{frame}
\subsection{Cart with Template::Toolkit}
\begin{frame}[fragile]{Cart: Template::Toolkit}
\lstinputlisting{cart-example-basic/cart.tt}
\end{frame}
\subsection{Cart with HTML::Zoom}
\begin{frame}[fragile]{Cart: HTML::Zoom}
\begin{lstlisting}
use HTML::Zoom;

$cart = ...
$zoom = HTML::Zoom->from_file('cart.html');

$zoom->select('.cartitem')->repeat_content([
  map { my $field = $_; sub {
    $_[0]->select('.name')->replace_content($field->{title});
    $_[0]->select('.quantity')->replace_content($field->{quantity});
    $_[0]->select('.price')->replace_content($field->{price});
    };
  } @$cart]);

print $zoom->to_html();
\end{lstlisting}
\end{frame}
\subsection{Template Problems}
\begin{frame}{Template Problems}
 \begin{itemize}
  \item Mini language in HTML template
  \item Dynamic pages (border cases, errors, ...)
 \end{itemize}
\end{frame}
\subsection{Cart with Template::Zoom}
\begin{frame}{Template::Zoom Concept}
 \begin{itemize}
  \item Specification
  \item Template
  \item Data or objects (iterator)
 \end{itemize}
\end{frame}
\subsubsection{Specification}
\begin{frame}[fragile]{Template::Zoom Specification}
\lstinputlisting{cart-example-basic/cart.xml}
\end{frame}
\subsubsection{Quellcode}
\begin{frame}[fragile]{Template::Zoom Script}
\begin{lstlisting}
use Template::Zoom;

my ($cart, $zoom);

$cart = ...

$zoom = new Template::Zoom(specification_file => 'cart.xml',
                           template_file => 'cart.html',
                           iterators => {cart => $cart});

print $zoom->process();
\end{lstlisting}
\end{frame}

\section{Lists}
\subsection{Iterators}
\begin{frame}{Iterators}
\begin{itemize}
  \item next method
  \item count method
  \item hash as return value
  \item Template::Zoom::Iterator
 \end{itemize}
\end{frame}
\subsection{Alternating rows}
\begin{frame}[fragile]{Lists with alternating rows}
\begin{lstlisting}
<table class="cart">
<tr class="cartheader">
<th>Name</th><th>Quantity</th><th>Price</th>
</tr>
<tr class="cartitem odd">
<td class="name">Perl 6</td>
<td><input class="quantity" name="quantity" size="3" value="10"></td>
<td class="price">$$$</td>
</tr>
<tr class="cartitem even">
<td class="name">Pro Git</td>
<td><input class="quantity" name="quantity" size="3" value="10"></td>
<td class="price">$$$</td>
</tr>
</table>
\end{lstlisting}
\end{frame}
\section{Example: Menus}
\subsection{Database table for menus}
\begin{frame}[fragile]{Menus: Database table}
\begin{lstlisting}
CREATE TABLE menus (
  code int NOT NULL auto_increment,
  name varchar(255) NOT NULL DEFAULT '',
  url  varchar(255) NOT NULL DEFAULT '',
  menu_name varchar(64) NOT NULL DEFAULT ''
  weight int NOT NULL DEFAULT 0,
  PRIMARY KEY(code),
  KEY(menu_name)
);
\end{lstlisting}
\end{frame}

\subsection{Specification}
\begin{frame}[fragile]{Menus: Specification}
\begin{lstlisting}
<specification name="menu" description="Menu">
<list name="menu" class="menu" table="menus">
<input name="name" required="1" field="menu_name"/>
<param name="label" field="name"/>
<param name="url"/>
</list>
</specification>
\end{lstlisting}
\end{frame}

\subsection{Template}
\begin{frame}[fragile]{Menus: Template}
\begin{lstlisting}
<ul class="menu">
<li><a href="" class="url"><span class="label"></span></li>
</ul>
\end{lstlisting}
\end{frame}

\subsection{Script}
\begin{frame}[fragile]{Menus: Script}
\begin{lstlisting}
use Template::Zoom;
use Template::Zoom::Database::Rose;

$db_object = new Template::Zoom::Database::Rose(dbname => 'temzoo',
    dbtype => 'Pg',
);

$zoom = new Template::Zoom(specification_file => 'menu.xml',
						   template_file => 'menu.html',
						   database => $db_object,
						  );
\end{lstlisting}
\end{frame}

\section{Filter and Sort}
\subsection{Specification with Filter}
\begin{frame}[fragile]{Filter: Specification}
\begin{lstlisting}
<specification name="menu" description="Menu">
<list name="menu" class="menu" table="menus">
<input name="name" required="1" field="menu_name"/>
<param name="label" field="name"/>
<param name="url" target="href" filter="link"/>
</list>
</specification>
\end{lstlisting}
\end{frame}

\subsection{Filter function}
\begin{frame}[fragile]{Filter: Function}
\begin{lstlisting}
sub link_filter {
    my $page = shift;
    my $url;

    $url = ...
    
    return $url;
}

$zoom = new Template::Zoom(specification_file => 'menu.xml',
						   template_file => 'menu.html',
						   database => $db_object,
						   filters => {link => \&link_filter},
						  );
\end{lstlisting}
\end{frame}

\subsection{Specification with sort}
\begin{frame}[fragile]{Sort: Specification}
\begin{lstlisting}
<specification name="menu" description="Menu">
<list name="menu" class="menu" table="menus">
<sort name="default">
<field name="weight" direction="desc"/>
<field name="code" direction="asc"/>
</sort>
<input name="name" required="1" field="menu_name"/>
<param name="label" field="name"/>
<param name="url" target="href" filter="link"/>
</list>
</specification>
\end{lstlisting}
\end{frame}

\section{Roadmap}
\begin{frame}{Roadmap}
 \begin{itemize}
   \item Conditions
   \item empty lists
   \item Selected items
   \item Paging
   \item Forms
   \item Trees 
   \item i18n
 \end{itemize}
\end{frame}

\begin{frame}{The End}
 \begin{description}
  \item[Git] http://git.linuxia.de/?p=temzoo.git;a=summary
  \item[CPAN] -
  \item[Website] -
  \item[Release] Christmas 2010
  \item[Talk]
    http://www.linuxia.de/talks/apw2010/temzoo-apw2010-beamer.pdf
   \item[Questions] ???
 \end{description}
\end{frame}

\end{document}

%%% Local Variables: 
%%% mode: latex
%%% TeX-master: t
%%% End: 
