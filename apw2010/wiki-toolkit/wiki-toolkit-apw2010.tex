% TODO
% Metadata slide
% Wiki::Toolkit History, Author
% Functions, add more of them
% Address for Slides
%   pointer to Wiki.pm source code
% Glue => %Storage example
% Plugins => Diff example

% found at http://www.mrunix.de/forums/archive/index.php/t-39599.html
\usepackage[T1]{fontenc}
\usepackage{mathptmx}
\usepackage[scaled=.90]{helvet}
\usepackage{courier}

\usepackage{beamerthemesplit}
\usepackage{verbatim}
\usepackage{hyperref}
\usepackage{listings}
\lstset{language=Perl,basicstyle=\footnotesize,tabsize=3}

% - Introduction
% - Motivation
% - Application

\title{Wikis mit Wiki::Toolkit}
\author[racke]{Stefan Hornburg (Racke)\\ \texttt{racke@linuxia.de}}
\date[APW2010]{�sterreichischer Perlworkshop 2010\\6. November 2010}

\begin{document}

\begin{frame}
  \titlepage
\end{frame}

\tableofcontents

% \section{Preface}
Herzlich willkommen zu meinem Vortrag �ber Wiki::Toolkit, ein Werkzeugkasten
zur Erstellung von Wikis mit Perl.

Meine Hauptbet�tigungsfeld ist Interchange, eine Open Source E-Commerce
Software, die in Perl geschrieben ist. F�r die Interchange-Website haben
wir vor zwei Jahren ein MoinMoin Wiki installiert, mit dem wir auch soweit
zufrieden waren. Leider funktionierte es nach einem Upgrade
nicht mehr und unsere Versuche, es wieder zum Laufen zu
bringen, waren leider erfolglos.

Daraufhin bin ich durch CPAN gestreift auf der Suche nach einem modernen
und gut zu programmierenden Perl Wiki. Wiki::Toolkit hat dabei durch seine
ausgereifte Architektur und eine m�gliche Integration in Interchange mein Interesse
geweckt.

Wiki::Toolkit ist kein fertiges Wiki, es ist eher dazugedacht in eine
vorhandene Anwendung eingebettet zu werden. Daf�r hat man freie Wahl
bei der Datenspeicherung, Formatierung und Suchmaschine. Entweder es kann
eines der vorhandenen Module eingesetzt werden oder ein eigenes Modul
geschrieben werden.

ACT
OpenGuides

% Real World Applications
% - OpenGuides
% - ACT
\section{Architektur}

% Dispatch

We start with taking a look at the architecture of a Wiki::Toolkit wiki.

Wiki::Toolkit is written in a modern style, so all components of
it can be exchanged and subclassed.

\begin{frame}{Architektur}
 \begin{itemize}
  \item Storage
  \item Formatierer
  \item Plugins
  \item Suche
  \item Glue
 \end{itemize}
\end{frame}

% \begin{frame}{Wiki Functions}
% \end{frame}

\section{Glue}

Before going into detail with the different tools that Wiki::Toolkit
provides, I show you how the glue looks like.

After selecting the tools you want to use in your Wiki::Toolkit wiki,
you create an instance of each of these tools and pass them to
the constructor for the Wiki::Toolkit class.

\subsection{Wiki::Toolkit Setup}

\begin{frame}[fragile]{Wiki::Toolkit Setup}
\begin{lstlisting}
my $store = new Wiki::Toolkit::Store::MySQL(dbname => 'wiki',
    dbuser => 'wiki',
    dbpass => 'wikisecret');

my $formatter = new Wiki::Toolkit::Formatter::UseMod;

my $search = new Wiki::Toolkit::Search::DBIxFTS(dbh => $store->dbh);

my $wiki = new Wiki::Toolkit(store => $store,
    formatter => $formatter,
    search => $search,
);
\end{lstlisting}
\end{frame}

\subsection{Grundlegende Methoden}

A Wiki node consists of:

\begin{itemize}
  \item text
  \item version
  \item metadata
  \item checksum
\end{itemize}

% handling conflicts

\begin{frame}[fragile]{Grundlegende Methoden}
\begin{lstlisting}
# create node
$wiki->write_node($name, $text, undef, \%metadata);

# retrieve and display node
my %node = $wiki->retrieve_node({name => $name, version => $version});
my $output = $wiki->format($node{content}, $node{metadata});
 
# update node
$wiki->write_node($name, $text, $node{checksum}, \%metadata);
\end{lstlisting}
\end{frame}

\section{Storage}

\subsection{Storage Backends}

All storage backends are derived from \verb+Wiki::Toolkit::Store::Database+.

\begin{frame}{Storage Backends}
 \begin{itemize}
  \item MySQL
  \item PostgreSQL
  \item SQLite
 \end{itemize}
\end{frame}

\subsection{Storage Setup}

% notes: cleardb, leaving tables alone
% cleardb doesn't erase tables for search backends

All storage backends can be easily initialized with the corresponding
\verb+Wiki::Toolkit::Setup+ classes.

\begin{frame}[fragile]
\frametitle{Storage Setup}
\lstinputlisting{scripts/setup_mysql}
\end{frame}

\subsection{Storage Tabellen}

\begin{frame}{Storage Tabellen}
 \begin{description}
  \item [content] Knoten
  \item [metadata] Metadata
  \item [internal\_links] Links der Knoten
  \item [schema\_info] Schema-Version
 \end{description}
\end{frame}

Tabellenpr�fix fehlt.

What I miss a little bit here, is a chance to provide a table prefix,
so these tables can happily live with tables from other applications
in the same database.

\subsection{Metadata}

% see also JSON plugin

Arbitrary metadata can be stored alongside with the node.

The metadata can be retrieved with the \verb+retrieve_node+ method.

\begin{frame}[fragile]{Metadata}
\begin{lstlisting}
# store metadata
my %metadata = {country => '�sterreich', 
    city => 'Wien'};

$wiki->write_node($node, $content, $checksum, \%metadata);

# retrieve metadata
my %node = $wiki->retrieve_node('�sterreichischer Perlworkshop');

for (keys %{$node{metadata}}) {
    print "Metadata $_: $node{$metadata}->{$_}->[0]\n";
}
\end{lstlisting}
\end{frame}

\section{Formatierer}

Formatters are classes that providing methods for turning text in
Wiki format into HTML code.

\subsection{Methoden}

The interface for Wiki::Toolkit formatters is really simple, it consists
of a mandatory \verb+format+ and an optional \verb+find_internal_links+
method.

\verb+format+ takes the text in the Wiki format and returns HTML code
suitable for display in a web page.

\verb+find_internal_links+ returns internal links in the Wiki.

\begin{frame}[fragile]{Methoden}
\begin{lstlisting}

my $html = $formatter->format($content);

my @links = $formatter->find_internal_links($content);

\end{lstlisting}
\end{frame}

\subsection{Optionen f�r den Konstruktor}

% see contructor for Wiki::Toolkit:Formatter::UseMod

\begin{frame}[fragile]{Konstruktoroptionen}
\begin{lstlisting}

my $formatter = Wiki::Toolkit::Formatter::Default->new(
                 extended_links  => 0,
                 implicit_links  => 1,
                 allowed_tags    => [qw(b i)],
                 macros          => {},
                 node_prefix     => 'wiki.cgi?node=',
                 edit_prefix     => 'wiki.cgi?action=edit&node=');

\end{lstlisting}
\end{frame}

Erweiterte Links sind gekennzeichnet durch eckige Klammern und enthalten
z.B. Linkziel und Linknamen. Der Formatierer entscheidet �ber das
exakte Format.

Implizite Links sind Links aus W�rtern im CamelCase.

\subsection{Makros}

\begin{frame}[fragile]{Makros}
\begin{lstlisting}
macros => { qr/(^|\b)\@SEARCHBOX(\b|$)/ =>
            qq(<form action="wiki.cgi" method="get">
                <input type="hidden" name="action" value="search">
                <input type="text" size="20" name="terms">
                <input type="submit"></form>) }
\end{lstlisting}
\end{frame}

% \subsection{Internal Links}

% Note: \verb+rename_node+ function uses internal_links table.

\subsection{Formatierer im CPAN}

\begin{frame}{Formatierer im CPAN}
 \begin{itemize}
  \item Mediawiki
  \item UseMod
  \item MarkDown
  \item POD
 \end{itemize}
\end{frame}

Mediawiki arbeitet nur mit dem Storage-Backend
\verb/Wiki::Toolkit::Store::Mediawiki/ zusammen.

\subsection{Formatierer kombinieren}

It is possible to use multiple formatters for one Wiki with
the Wiki::Toolkit::Formatter::Multiple class.

\begin{frame}[fragile]{Formatierer kombinieren}
\begin{lstlisting}
my $fmt_usemod = new Wiki::Toolkit::Formatter::UseMod;
my $fmt_mediawiki = new Wiki::Toolkit::Formatter::Mediawiki;

my $fmt = new Wiki::Toolkit::Formatter::Multiple (
      usemod => $fmt_usemod,
      mediawiki => $fmt_mediawiki,
      _DEFAULT => $fmt_usemod,
  );

...

my $output = $wiki->format($text, {formatter => 'mediawiki'});
\end{lstlisting}
\end{frame}

\begin{frame}[fragile]{Formatierer kombinieren}
\begin{lstlisting}
# write node
$metadata{formatter} = 'mediawiki';
$wiki->write_node($name, $text, $checksum, \%metadata);

# retrieve node
%node = $wiki->retrieve_node($name);

# display node
$wiki->format($node{content}, $node{metadata});
\end{lstlisting}
\end{frame}

\subsection{Formatierer selbst erstellen}

% HINT to Text::WikiFormat;

\begin{frame}{Formatierer selbst erstellen}
 \begin{itemize}
   \item Wiki::Toolkit::Formatter::Foo
   \item format()-Methode
   \item Text::WikiFormat
 \end{itemize}
\end{frame}

\section{Suche}

\subsection{Backends f�r die Suche}

% Search::InvertedIndex
% DBIx::FullTextSearch

Plucene is a Perl port of the Lucene search engine.

\begin{frame}{Backends f�r die Suche}
 \begin{itemize}
  \item Wiki::Toolkit::Search::SII
  \item Wiki::Toolkit::Search::DBIxFTS
  \item Wiki::Toolkit::Search::Plucene
 \end{itemize}
\end{frame}

\subsection{Suche Setup}

\begin{frame}[fragile]{Suche Setup}
\lstinputlisting{scripts/setup_mysqlindex}
\end{frame}

\subsection{Beispiel f�r die Suche}

% note: setup DBIx::FullTextSearch indexes
% mention script

\begin{frame}[fragile]{Beispiel f�r die Suche}
\begin{lstlisting}
# first time setup
Wiki::Toolkit::Setup::DBIxFTSMySQL::setup($dbname, $dbuser, $dbpass, $dbhost);

my $store = new Wiki::Toolkit::Store::MySQL(dbname => 'wiki',
    dbuser => 'wiki',
    dbpass => 'wikisecret');

my $search = new Wiki::Toolkit::Search::DBIxFTS(dbh => $store->dbh);
my %dpw_nodes = $search->search_nodes('wien');
\end{lstlisting}
\end{frame}

\section{Plugins}

% Wiki::Toolkit::Feed modules
% http://search.cpan.org/~dom/Wiki-Toolkit-0.78/lib/Wiki/Toolkit/Extending.pod

\begin{frame}[fragile]
\frametitle{Plugins}
\begin{lstlisting}
my $plugin = new Wiki::Toolkit::Plugin::Diff;
$wiki->register_plugin(plugin => $plugin);
%diff = $plugin->differences(node => 'FrontPage',
    left_version => 101,
    right_version => 105);
\end{lstlisting}
\end{frame}

\subsection{Plugins im CPAN}

\begin{frame}{Plugins im CPAN}
 \begin{description}
  \item[Diff] Anzeige der Unterschiede zwischen den Versionen
  \item[Categorizer] Kategorieverwaltung
  \item[JSON] letzte �nderung als JSON
  \item[Locator::Grid] Verwaltung von Koordinaten
  \item[Ping] verschiedene Dienste anpingen
  \item[RSS::Reader] retrieve feeds for node inclusion
 \end{description}
\end{frame}


\begin{frame}{The End}
 \begin{description}
  \item[CPAN] http://search.cpan.org/perldoc?Wiki::Toolkit
  \item[Website] http://www.wiki-toolkit.org/
  \item[Talk]
    http://www.linuxia.de/talks/apw2010/wiki-toolkit-apw2010-beamer.pdf
   \item[Fragen] ???
 \end{description}
\end{frame}

\section{Wiki Syntax}
\subsection{Markdown}
Homepage: \url{http://daringfireball.net/projects/markdown/syntax}

Wiki: \url{http://markdown.infogami.com/}

\subsubsection{Emphasis}

\begin{tabular}{ll}
\verb|*single asterisks*| & \textit{italic} \\
\verb|_single underscores_| & \textbf{bold} \\
\end{tabular}

% TODO: double asterisks / double underscores

\subsubsection{Links}
Internal links:

\begin{verbatim}
See my [About](/about/) page for details.
\end{verbatim}

\subsection{UseMod}
Homepage: \url{http://www.usemod.com/cgi-bin/wiki.pl}
Syntax: \url{http://www.usemod.com/cgi-bin/wiki.pl?TextFormattingRules}

\subsubsection{Emphasis}

\section{Resources}

Homepage: \url{http://www.wiki-toolkit.org/}

Mailing list: \url{http://www.earth.li/cgi-bin/mailman/listinfo/cgi-wiki-dev}

\subsection{Related Projects}
MojoMojo, a Catalyst \& DBIx::Class powered Wiki,
\url{http://search.cpan.org/perldoc?MojoMojo}.

\end{document}
    
%%% Local Variables: 
%%% mode: latex
%%% TeX-master: t
%%% End: 
