% CLONE TICKETS !!
% Konformit�t
% Anwendung

%% -*- mode: latex; -*-

\section{Automatisierung und Integration von Request Tracker Systemen mittels REST-Schnittstelle}

\subsection*{Autor}
Stefan Hornburg (Racke) \verb/<racke@linuxia.de>/

\subsection*{Kurzbiographie}
Stefan Hornburg arbeitet seit 1998 als Open Source Consultant mit den
Schwerpunkten Linux, Perl und Interchange. Als Captain der ICDEVGROUP leitet
er die Entwicklergruppe von Interchange und ist als Debian Maintainer f�r
verschiedene Serverpakete verantwortlich (u.a. Courier, Pure-FTPd und
Sympa).

\subsection*{Einf�hrung}
Request Tracker (RT) ist ein in Perl programmiertes
Trouble-Ticket-System. Neben gro�en Organisationen wie die NASA und dem MIT
wird RT auch f�r das Bugtracking von CPAN und Perl selbst verwendet. 

Tickets k�nnen interaktiv im Browser oder durch Emails bearbeitet werden. Um
typische Aufgaben automatisieren zu k�nnen, bietet RT sowohl eine Perl API
und als auch eine REST-Schnittstelle an. W�hrend f�r die API der gr��ere
Funktionsumfang spricht, erlaubt die REST-Schnittstelle Kommunikation mit
Installationen auf anderen Rechnern und ben�tigt keine Zugriffsrechte auf
die Konfigurationsdatei, die sensible Informationen wie den
Datenbankbenutzer und das Datenbankpasswort enth�lt. 

F�r meine Projekte (Bugtracking-System f�r Interchange und Zusammenfassung
der Informationen von mehreren RT-Systeme meiner Kunden) habe ich den Weg
�ber die REST-Schnittstelle gew�hlt. 

Die REST-Schnittstelle erlaubt den Zugriff auf die Tickets, deren
Transaktionen und alle zugeh�rigen Dateien (Attachments). Neben 
dem Abruf aller Informationen zu einem Ticket und der Suche k�nnen
verschiedene Operationen durchgef�hrt werden, um Tickets zu
bearbeiten. Darunter f�llt das Hinzuf�gen von Kommentaren, Zusammenfassen,
Referenzierung und die �bernahme von Tickets. 

Der Vortrag erl�utert die Grundlagen von REST, die verf�gbaren Funktionen
der Schnittstelle von RT und die Programmierung mit Hilfe von LWP (libwww-perl).

\subsection{Definition von \WSIndex{REST}}
Der Begriff Representational State Transfer (REST) wurde durch die
Dissertation von Roy Fielding \cite{racke:fielding} gepr�gt. REST bezeichnet
einen Softwarearchitekturstil f�r verteilte
Hypermedia-Informationssysteme wie das World Wide Web.


Die vier Grundprinzipien von REST sind:

\begin{itemize}
% wiki 1.
\item Funktionalit�t und Status der Anwendung ist in Ressourcen aufgeteilt
% wiki 2.
\item Adressierbarkeit, universelle Syntax zur Identifikation von
  Ressourcen, jede Ressource ist eindeutig durch ihre URI addressierbar
\item Zustandslosigkeit
\item Menge von wohldefinierten Operationen, die auf alle Ressourcen
  angewandt werden k�nnen (f�r HTTP u.a. GET, POST, PUT und DELETE)
\end{itemize}

Systeme, die Fieldings Prinzipien entsprechend, werden oft als REST-konform
(Englisch ``RESTful'') bezeichnet.

%\subsection{Request Tracker}
%
%Jedes Objekt in RT hat einen Type (\verb/ticket/, \verb/queue/) und eine
%numerische ID.

\subsection{REST-Schnittstelle des Request Trackers}

Die REST-Schnittstelle von RT erlaubt die Abfrage und Manipulation von
verschiedenen Objekten: Queues, Benutzer, Benutzergruppen und Tickets. 

Jede Anfrage an die Schnittstelle beinhaltet eine Aktion:

\begin{description}
\item [list] Auflistung/Suche von Objekten
\item [show] Anzeige von Objekteigenschaften
\item [create] Anlegen von Objekten
\item [edit] Bearbeitung von Objekten
\end{description}

Weitere Parameter dienen zur genaueren Spezifikation der Aktion bzw.
der Antwort:

\begin{description}
\item query
\item orderby
\item format
\end{description}

\subsubsection{Adressierung}
Die REST-Schnittstelle wird �ber die URI der RT-Instanz mit der
Pfadangabe /REST/1.0/ angesprochen, also z.B.:

\begin{verbatim}
http://support.linuxia.de/rt/REST/1.0/
\end{verbatim}

An diese URI wird der Name der jeweiligen Aktion angeh�ngt:

\begin{verbatim}
http://support.linuxia.de/rt/REST/1.0/show
\end{verbatim}

Damit ist die REST-Konformit�t aber schon am Ende. Es wird f�r alle
HTTP-Anfragen die POST-Methode verwendet und weitere Parameter werden
im Body platziert. Dies verletzt die Prinzipien der 
Adressierbarkeit und wohldefinierter Ressourcen.

\subsubsection{Nutzung von der Kommandozeile}
Mit dem Kommandozeilentool /usr/bin/rt kann man sich recht gut einen
�berblick �ber die Funktionsweise der REST-Schnittstelle verschaffen.

Queue anlegen:
\begin{verbatim}
$ rt create -t queue set name='Perl-Workshop'
# Queue 4 created.
\end{verbatim}

Ticket anlegen:
\begin{verbatim}
$ rt create -t ticket set subject='Ausarbeitung des Vortrags' \
  set queue='Perl-Workshop' 
# Ticket 11 created.
\end{verbatim}

Die Kommunikation mit Request Tracker ist nicht genauer spezifiziert.
Deshalb habe ich den Aufbau der Anfragen und die Auswertung der Antworten
nach dem Motto RTSL (Read the Source, Luke) aus dem Quellcode des
Kommandozeilentools abgeleitet.

\subsection{Kommunikation �ber die REST-Schnittstelle}
\subsubsection{Aufbau der HTTP-Anfrage}
An die URI der REST-Schnittstelle wird die gew�nschte Funktion angeh�ngt,
alle weiteren Parameter werden im Body der HTTP-Anfrage �bergeben.

\begin{verbatim}
http://support.linuxia.de/rt/REST/1.0/show
\end{verbatim}

\subsection{Implementierung}
Die Implementierung der HTTP-Kommunikation wird mit dem bekannten Perlmodul
\WSIndex{LWP::UserAgent} durchgef�hrt.

\begin{verbatim}
my ($ua, $req, $res);

$ua = new LWP::UserAgent(agent => "Vend::RT/1.0", env_proxy => 1);
$req = POST($uri, $data, Content_Type => 'form-data');
$res = $ua->request($req);
\end{verbatim}

\subsection{Authentifizierung}
Die REST-Schnittstelle unterst�tzt keine HTTP-Authentifizierung.

\subsection{Fehlerbehandlung}
Fehlerhafte Eingabeparameter werden von RT im \emph{Body} der HTTP-Antwort
quittiert:
\begin{verbatim}
RT/3.6.4 400 Bad Request
 
No objects specified.
\end{verbatim}

% liste der m�glichen Fehler

\begin{thebibliography}{99}
\bibitem{racke:fielding} Fielding, Roy Thomas. 
\emph{Architectural Styles and Design of Network-Based Software Architectures}.
University of California, Irvine, 2000.
\bibitem{racke:restped} Representational State Transfer
%\texttt{http://de.wikipedia.org/wiki/Representational_State_Transfer}
\bibitem{racke:restfulws} Leonard Richardson und Sam Ruby.
\emph{RESTful Web Services}.
O'Reilly, Sebastopol, California, 2007.
\end{thebibliography}


