% CLONE TICKETS !!
% Konformit�t
% Anwendung

%% -*- mode: latex; -*-

\section{Automatisierung und Integration von Request Tracker Systemen mittels REST-Schnittstelle}

\subsection*{Autor}
Stefan Hornburg (Racke) \verb/<racke@linuxia.de>/

\subsection*{Kurzbiographie}
Stefan Hornburg arbeitet seit 1998 als Open Source Consultant mit den
Schwerpunkten Linux, Perl und Interchange. Als Captain der ICDEVGROUP leitet
er die Entwicklergruppe von Interchange und ist als Debian Maintainer f�r
verschiedene Serverpakete verantwortlich (u.a. Courier, Pure-FTPd und
Sympa).

\subsection*{Einf�hrung}
Request Tracker (RT) ist ein in Perl programmiertes
Trouble-Ticket-System. Neben gro�en Organisationen wie die NASA und dem MIT
wird RT auch f�r das Bugtracking von CPAN und Perl selbst verwendet. 

Tickets k�nnen interaktiv im Browser oder durch Emails bearbeitet werden. Um
typische Aufgaben automatisieren zu k�nnen, bietet RT sowohl eine Perl API
und als auch eine REST-Schnittstelle an. W�hrend f�r die API der gr��ere
Funktionsumfang spricht, erlaubt die REST-Schnittstelle Kommunikation mit
Installationen auf anderen Rechnern und ben�tigt keine Zugriffsrechte auf
die Konfigurationsdatei, die sensible Informationen wie den
Datenbankbenutzer und das Datenbankpasswort enth�lt. 

F�r meine Projekte (Bugtracking-System f�r Interchange und Zusammenfassung
der Informationen von mehreren RT-Systeme meiner Kunden) habe ich den Weg
�ber die REST-Schnittstelle gew�hlt. 

Die REST-Schnittstelle erlaubt den Zugriff auf die Tickets, deren
Transaktionen und alle zugeh�rigen Dateien (Attachments). Neben 
dem Abruf aller Informationen zu einem Ticket und der Suche k�nnen
verschiedene Operationen durchgef�hrt werden, um Tickets zu
bearbeiten. Darunter f�llt das Hinzuf�gen von Kommentaren, Zusammenfassen,
Referenzierung und die �bernahme von Tickets. 

Der Vortrag erl�utert die Grundlagen von REST, die verf�gbaren Funktionen
der Schnittstelle von RT und die Programmierung mit Hilfe von LWP (libwww-perl).

\subsection{Definition von \WSIndex{REST}}
Der Begriff Representational State Transfer (REST) wurde durch die
Dissertation von Roy Fielding \cite{racke:fielding} gepr�gt. REST bezeichnet
einen Softwarearchitekturstil f�r verteilte
Hypermedia-Informationssysteme wie das World Wide Web.


Die vier Grundprinzipien von REST sind:

\begin{itemize}
% wiki 1.
\item Funktionalit�t und Status der Anwendung ist in Ressourcen aufgeteilt
% wiki 2.
\item Adressierbarkeit, universelle Syntax zur Identifikation von
  Ressourcen, jede Ressource ist eindeutig durch ihre URI addressierbar
\item Zustandslosigkeit
\item Menge von wohldefinierten Operationen, die auf alle Ressourcen
  angewandt werden k�nnen (f�r HTTP u.a. GET, POST, PUT und DELETE)
\end{itemize}

Systeme, die Fieldings Prinzipien entsprechend, werden oft als REST-konform
(Englisch ``RESTful'') bezeichnet.

%\subsection{Request Tracker}
%
%Jedes Objekt in RT hat einen Type (\verb/ticket/, \verb/queue/) und eine
%numerische ID.

\subsection{REST-Schnittstelle des Request Trackers}

Die REST-Schnittstelle von RT erlaubt die Abfrage und Manipulation von
verschiedenen Objekten: Queues, Benutzer, Benutzergruppen und Tickets. 

Jede Anfrage an die Schnittstelle beinhaltet eine Aktion:

\begin{description}
\item [list] Auflistung/Suche von Objekten
\item [show] Anzeige von Objekteigenschaften
\item [create] Anlegen von Objekten
\item [edit] Bearbeitung von Objekten
\end{description}

Die Suche ist zur Zeit leider nur f�r Tickets implementiert, bei allen
anderen Objekttypen ist die Antwort ein \verb/Server Error/ mit der
Erl�uterung \verb/Unsupported object type/.

Weitere Parameter dienen zur genaueren Spezifikation der Aktion bzw.
der Antwort:

\begin{description}
\item query
\item orderby
\item format
\end{description}

\subsubsection{Adressierung}
Die REST-Schnittstelle wird �ber die URI der RT-Instanz mit der
Pfadangabe /REST/1.0/ angesprochen, also z.B.:

\begin{verbatim}
http://support.linuxia.de/rt/REST/1.0/
\end{verbatim}

An diese URI wird der Name der jeweiligen Aktion angeh�ngt:

\begin{verbatim}
http://support.linuxia.de/rt/REST/1.0/show
\end{verbatim}

Damit ist die REST-Konformit�t aber schon am Ende. Es wird f�r alle
HTTP-Anfragen die POST-Methode verwendet und weitere Parameter werden
im Body platziert. Dies verletzt die Prinzipien der 
Adressierbarkeit und wohldefinierter Ressourcen.

\subsubsection{Nutzung von der Kommandozeile}
Mit dem Kommandozeilentool /usr/bin/rt kann man sich recht gut einen
�berblick �ber die Funktionsweise der REST-Schnittstelle verschaffen.

Queue anlegen:
\begin{verbatim}
$ rt create -t queue set name='Perl-Workshop'
# Queue 4 created.
\end{verbatim}

Ticket anlegen:
\begin{verbatim}
$ rt create -t ticket set subject='Ausarbeitung des Vortrags' \
  set queue='Perl-Workshop' 
# Ticket 11 created.
\end{verbatim}

Ticket anzeigen:
\begin{verbatim}
$ rt show ticket/11 -f id,subject,queue,requestors,owner
id: ticket/11
Subject: Ausarbeitung des Vortrags
Queue: Perl-Workshop
Requestors: racke@linuxia.de
Owner: Nobody
\end{verbatim}

Die HTTP-Abfrage und Antwort 
kann mit der Umgebungsvariable \WSIndex{RTDEBUG} sichtbar gemacht werden:

\begin{verbatim}
$ RTDEBUG=3 rt show -t user racke -f id,name,emailaddress,realname,nickname
POST http://support.linuxia.de/rt//REST/1.0/show
Content-Length: 175
Content-Type: multipart/form-data; boundary=xYzZY
Cookie: RT_SID_support.linuxia.de.80=0df6e85dc77d5ca70908c0bad8a038cd

--xYzZY
Content-Disposition: form-data; name="fields"

id,name,emailaddress,realname,nickname
--xYzZY
Content-Disposition: form-data; name="id"

user/racke
--xYzZY--
HTTP/1.1 200 OK
Connection: close
Date: Thu, 17 Jan 2008 11:02:05 GMT
Server: Apache/2.2.3 (Debian) PHP/5.2.0-8+etch9 mod_ssl/2.2.3 OpenSSL/0.9.8c mod_perl/2.0.2 Perl/v5.8.8
Content-Type: text/plain; charset=utf-8
Client-Date: Thu, 17 Jan 2008 11:02:29 GMT
Client-Peer: 85.10.244.99:80
Client-Response-Num: 1
Client-Transfer-Encoding: chunked

RT/3.6.1 200 Ok

id: user/22
Name: racke
EmailAddress: racke@linuxia.de
RealName: Stefan Hornburg
NickName: Racke
\end{verbatim}

\subsection{Programmierung}
Die Kommunikation mit Request Tracker ist nicht genauer spezifiziert.
Deshalb habe ich den Aufbau der Anfragen und die Auswertung der Antworten
nach dem Motto RTSL (Read the Source, Luke) aus dem Quellcode des
Kommandozeilentools abgeleitet.

Die Implementierung der HTTP-Kommunikation wird mit dem bekannten Perlmodul
\WSIndex{LWP::UserAgent} durchgef�hrt.

\begin{verbatim}
my ($ua, $req, $res);

$ua = new LWP::UserAgent(agent => "Vend::RT/1.0", env_proxy => 1);
$req = POST($uri, $data, Content_Type => 'form-data');
$res = $ua->request($req);
\end{verbatim}

Bei fehlerloser HTTP-Kommunikation liefert die REST-Schnittstelle 
die gew�nschten Daten zur�ck. Vor dem eigentlichen Inhalt befindet
sich eine Statuszeile, die �hnlich aufgebaut ist wie die HTTP-Statuszeile:

\begin{verbatim}
RT/3.6.1 200 Ok
\end{verbatim}

Ein einzelnes Objekt in der Antwort wird dann wie folgt geparst:

\begin{verbatim}
sub parse_object {
    my ($text) = @_;
    my (@lines, $count, %attr);
    
    @lines = split(/\n/, $text);

    # skip empty lines / comments at the beginning
    $count = @lines;
    while ($count-- && ($lines[0] !~ /\S/ || $lines[0] =~ /^#/)) {
        shift(@lines);
    }

    my $lastname = '';
    
    for my $line (@lines) {
        next unless length($line);
        
        # line continuation ?
        if ($line =~ s/^(\s+)//) {
            $attr{$lastname} .= "\n$line";
            next;
        }
        my ($name, $value) = split(/:\s*/, $line, 2);
        $attr{$name} = $value;
        $lastname = $name;
    }

    # break up object/id
    if (exists $attr{id}) {
        my @frags = split('/', $attr{id});

        if (@frags == 2) {
            delete $attr{id};
            $attr{Object} = $frags[0];
            $attr{ID} = $frags[1];
        }
    }

    return \%attr;
}
\end{verbatim}

Auflistungen enthalten mehrere Objekte.

\begin{verbatim}
sub parse_object_list {
    my ($text) = @_;
    my (@frags, @list);

    @frags = split(/\n\n--\n\n/, $text);
    if (@frags == 1 && $frags[0] =~ /No matching results./) {
        # empty list
        return [];
    }
    for my $frag (@frags) {
        push (@list, parse_object($frag));
    }
    return \@list;
}
\end{verbatim}

\subsection{Authentifizierung}
Die REST-Schnittstelle unterst�tzt keine HTTP-Authentifizierung.

\subsection{Fehlerbehandlung}
Fehlerhafte Eingabeparameter werden von RT im \emph{Body} der HTTP-Antwort
quittiert:
\begin{verbatim}
RT/3.6.4 400 Bad Request
 
No objects specified.
\end{verbatim}

% liste der m�glichen Fehler

\begin{thebibliography}{99}
\bibitem{racke:fielding} Fielding, Roy Thomas. 
\emph{Architectural Styles and Design of Network-Based Software Architectures}.
University of California, Irvine, 2000.
%\bibitem{racke:restped} Representational State Transfer
%\texttt{http://de.wikipedia.org/wiki/Representational_State_Transfer}
\bibitem{racke:restfulws} Leonard Richardson und Sam Ruby.
\emph{RESTful Web Services}.
O'Reilly, Sebastopol, California, 2007.
\end{thebibliography}


