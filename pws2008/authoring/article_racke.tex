% CLONE TICKETS !!


%% -*- mode: latex; -*-

\section{Automatisierung und Integration von Request Tracker Systemen mittels REST-Schnittstelle}

\subsection*{Autor}
Stefan Hornburg (Racke) \verb/<racke@linuxia.de>/

\subsection*{Kurzbiographie}
Stefan Hornburg arbeitet seit 1998 als Open Source Consultant mit den
Schwerpunkten Linux, Perl und Interchange. Als Captain der ICDEVGROUP leitet
er die Entwicklergruppe von Interchange und ist als Debian Maintainer f�r
verschiedene Serverpakete verantwortlich (u.a. Courier, Pure-FTPd und
Sympa). Er betreut Mailserver f�r seine Kunden seit 1999.

\subsection*{Einf�hrung}
Request Tracker (RT) ist ein in Perl programmiertes
Trouble-Ticket-System. Neben gro�en Organisationen wie die NASA und dem MIT
wird RT auch f�r das Bugtracking von CPAN und Perl selbst verwendet. 

Tickets k�nnen interaktiv im Browser oder durch Emails bearbeitet werden. Um
typische Aufgaben automatisieren zu k�nnen, bietet RT sowohl eine Perl API
und als auch eine REST-Schnittstelle an. W�hrend f�r die API der gr��ere
Funktionsumfang spricht, erlaubt die REST-Schnittstelle Kommunikation mit
Installationen auf anderen Rechnern und ben�tigt keine Zugriffsrechte auf
die Konfigurationsdatei, die sensible Informationen wie den
Datenbankbenutzer und das Datenbankpasswort enth�lt. 

F�r meine Projekte (Bugtracking-System f�r Interchange und Zusammenfassung
der Informationen von mehreren RT-Systeme meiner Kunden) habe ich den Weg
�ber die REST-Schnittstelle gew�hlt. 

Die REST-Schnittstelle erlaubt den Zugriff auf die Tickets, deren
Transaktionen und alle zugeh�rigen Dateien (Attachments). Neben 
dem Abruf aller Informationen zu einem Ticket und der Suche k�nnen
verschiedene Operationen durchgef�hrt werden, um Tickets zu
bearbeiten. Darunter f�llt das Hinzuf�gen von Kommentaren, Zusammenfassen,
Referenzierung und die �bernahme von Tickets. 

Der Vortrag erl�utert die Grundlagen von REST, die verf�gbaren Funktionen
der Schnittstelle von RT und die Programmierung mit Hilfe von LWP (libwww-perl).

\subsection{\WSIndex{REST}}
% cut & waste from Wikipedia
Der Begriff Representational State Transfer (REST) bezeichnet einen Softwarearchitekturstil f�r verteilte Hypermedia-Informationssysteme. W�hrend die Architektur des World Wide Webs durch den URI Standard und das HTTP Protokoll beschrieben werden kann, diente der REST Architekturstil seit 1994 als Richtlinie f�r die Weiterentwicklung der bestehenden Standards.

REST stammt aus der Dissertation von Roy Fielding, in welcher der Erfolg des
World Wide Webs auf bestimmte Eigenschaften der verwendeten Mechanismen und
Protokolle (z.B. HTTP) zur�ckgef�hrt wird. Roy Fielding ist einer der
Hauptautoren der Spezifikation des Hypertext-Transfer-Protokolls (HTTP).
% cut & waste from Wikipedia

\subsection{Funktionen der REST-Schnittstelle}
Folgende Objekte k�nnen angesprochen werden:

\begin{itemize}
\item Tickets (ticket)
\item Benutzer (user)
\end{itemize}

Folgende Aktionen k�nnen ausgef�hrt werden:

\begin{itemize}
\item show
\end{itemize}

Beispiele:
\begin{verbatim}
erebus:~# rt show user/racke
URI: http://rt.icdevgroup.org/REST/1.0/show
id: user/22
Name: racke
Password: ********
EmailAddress: racke@linuxia.de
RealName: Stefan Hornburg
NickName: Racke
Lang: en
\end{verbatim}

Gleichwertig ist der Befehl rt show user/32

\subsection{Aufbau der URL}
\begin{verbatim}
http://support.linuxia.de/rt/REST/1.0/show
\end{verbatim}

\subsection{Authentifizierung}

\subsection{Fehlerbehandlung}
Fehlerhafte Eingabeparameter werden von RT im \emph{Body} der HTTP-Antwort
quittiert:
\begin{verbatim}
RT/3.6.4 400 Bad Request
 
No objects specified.
\end{verbatim}

% liste der m�glichen Fehler

\begin{thebibliography}{99}
\bibitem{racke:restped} Representational State Transfer
%\texttt{http://de.wikipedia.org/wiki/Representational_State_Transfer}
\bibitem{racke:restfulws} RESTful Web Services
\end{thebibliography}


