
\usepackage[T1]{fontenc}
\usepackage{mathptmx}
\usepackage[scaled=.90]{helvet}
\usepackage{courier}

\usepackage{beamerthemesplit}
\usepackage{verbatim}
\usepackage{hyperref}
\usepackage{listings}
\lstset{language=Perl,basicstyle=\footnotesize,tabsize=3,showstringspaces=false}

\title{Template::Flute - Modern designer-friendly HTML templating engine}
\author[racke]{Stefan Hornburg (Racke)\\ \texttt{racke@linuxia.de}}
\date[]{Perl Workshop in Israel, 28th February 2012, Ramat Gan}

\begin{document}
\maketitle{}

\begin{frame}
  \titlepage
\end{frame}

\tableofcontents

\section{Template::Flute}
Template::Flute enables you to completely separate web design and programming
tasks for dynamic web applications. 

Templates are plain HTML files without inline code or mini language, thus
making it easy to maintain them for web designers and to preview them with
a browser.

The CSS selectors in the template are tied to your data structures or
objects by a specification, which relieves the programmer from changing
his code for mere changes of class names.

In addition to HTML output, Template::Flute also supports generation of
PDF files on-the-fly based on the same template and specification.

\subsection{Why and Where}
\begin{frame}{Why}
  \begin{itemize}
  \item Separation of web design and programming
  \item How available template engines violate this principle
    \begin{itemize}
    \item Mini language (Template::Toolkit)
    \item Inline code
    \item CSS selectors (HTML::Zoom)
    \end{itemize}
  \item Solutions by Template::Flute
    \begin{itemize}
      \item Static HTML file
      \item Specification file
    \end{itemize}
  \end{itemize}
%  \item Einarbeitungszeit für Designer reduzieren
%  \item Grenzfälle, Fehler und Bedingungen (dynamische Seiten)
%\url{http://perl.apache.org/docs/tutorials/tmpl/comparison/comparison.html}
\end{frame}

% \begin{frame}{Where}
% \end{frame}

\subsection{Template Example}
\begin{frame}{Template Example}
\begin{tabular}[t]{rrrr}
\hfill Name & Quantity & Price \\
\hfill & & \\
\hfill & Total & \\
\hfill & & \\
\end{tabular}
\end{frame}

\begin{frame}[fragile]{Template Example: Data}
\begin{lstlisting}
@cart = (
         {isbn => '978-0-2016-1622-4', 
          title => 'The Pragmatic Programmer',
          quantity => 1, price => 49.95},

         {isbn => '978-1-4302-1833-3',
          title => 'Pro Git',
          quantity => 1, price => 34.99},
	);
\end{lstlisting}
\end{frame}

\begin{frame}{Template Example: Output}
\begin{tabular}[t]{rrrr}
\hfill Name & Quantity & Price \\
\hfill & & \\
\hfill The Pragmatic Programmer & 1 & \$ 49.95 \\
\hfill Pro Git & 1 & \$ 34.99 \\
\hfill & & \\
\hfill & Total & \$ 100
\end{tabular}
\end{frame}

\subsubsection{HTML Template}
\begin{frame}[fragile]{Cart: HTML Template}
\begin{lstlisting}
<table class="cart">
<tr class="cartheader">
<th>Name</th><th>Quantity</th><th>Price</th>
</tr>
<tr class="cartitem">
<td class="name">Perl 6</td>
<td><input class="quantity" name="quantity" size="3" value="10"></td>
<td class="price">$$$</td>
</tr>
<tr class="cartheader">
<th colspan="2"></th><th>Total</th></tr>
<tr>
<td colspan="2"></td><td class="cost">$$$</td></tr>
</table>
\end{lstlisting}
\end{frame}

\subsubsection{Cart with Template::Toolkit}
\begin{frame}[fragile]{Cart: Template::Toolkit}
\lstinputlisting{cart-example-basic/cart.tt}
\end{frame}

\subsection{Common Problems}
\begin{frame}{Common Problems}
 \begin{itemize}
  \item Mini language in HTML template
  \item Dynamic pages (border cases, errors, ...)
 \end{itemize}
\end{frame}

\subsection{Concept}
\begin{frame}{Concept}
 \begin{itemize}
  \item HTML Template
  \item Specification
  \item Data or objects (iterator)
 \end{itemize}
\end{frame}
% \subsubsection{Specification}
% \begin{frame}[fragile]{Template::Flute Specification (XML)}
% \lstinputlisting{cart-example-basic/cart.xml}
% \end{frame}
% \begin{frame}[fragile]{Template::Flute Specification (Config::Scoped)}
% \lstinputlisting{cart-example-basic/cart.conf}
% \end{frame}
% \subsubsection{Quellcode}
% \begin{frame}[fragile]{Template::Flute Script (XML)}
% \begin{lstlisting}
% use Template::Flute;

% my ($cart, $flute, %values);

% $cart = ...
% $values{cost} = ...

% $flute = new Template::Flute(specification_file => 'cart.xml',
%                            template_file => 'cart.html',
%                            iterators => {cart => $cart},
%                            values => \%values,
% );

% print $flute->process();
% \end{lstlisting}
% \end{frame}

% \begin{frame}[fragile]{Template::Flute Script (Config::Scoped)}
% \begin{lstlisting}
% use Template::Flute;

% my ($cart, $flute, %values);

% $cart = ...
% $values{cost} = ...

% $flute = new Template::Flute(specification_file => 'cart.conf',
%                            specification_parser => 'Scoped',
%                            template_file => 'cart.html',
%                            iterators => {cart => $cart},
%                            values => \%values,
% );

% print $flute->process();
% \end{lstlisting}
% \end{frame}

You are probably missing the \$ sign in the HTML output, we see
to that later.

\subsubsection{Template}
The HTML template for the menu is really simple, because the
styling can be done completely with CSS.

% \begin{frame}[fragile]{Menus: Template}
% \begin{lstlisting}
% <ul class="menu">
% <li><a href="" class="url"><span class="label"></span></li>
% </ul>
% \end{lstlisting}
% \end{frame}

% \subsubsection{Script}
% \begin{frame}[fragile]{Menus: Script}
% \begin{lstlisting}
% use Template::Flute;
% use Template::Flute::Database::Rose;

% $db_object = new Template::Flute::Database::Rose(dbname => 'temzoo',
%     dbtype => 'Pg',
% );

% $flute = new Template::Flute(specification_file => 'menu.xml',
% 						   template_file => 'menu.html',
% 						   database => $db_object,
% 						  );

% $flute->process({name => main});
% \end{lstlisting}
% \end{frame}

\section{Basic Elements}
\begin{frame}{Basic Elements}
\begin{itemize}
\item Variables
\item Conditions
\item Loops
\end{itemize}
\end{frame}

\subsection{Variables}
\begin{frame}[fragile]{Variables}
HTML
\begin{lstlisting}
<span class="email">foo@bar.com</span>
\end{lstlisting}
Specification
\begin{lstlisting}
<value name="email" class="email"/>
\end{lstlisting}
Code
\begin{lstlisting}
$flute->process(values => {email => 'racke@linuxia.de'});
\end{lstlisting}
Output
\begin{lstlisting}
<span class="email">racke@linuxia.de</span>
\end{lstlisting}
\end{frame}

\begin{frame}[fragile]{Variables}
HTML
\begin{lstlisting}
<span id="email">foo@bar.com</span>
\end{lstlisting}
Specification
\begin{lstlisting}
<value name="email" id="email"/>
\end{lstlisting}
Code
\begin{lstlisting}
$flute->process(values => {email => 'racke@linuxia.de'});
\end{lstlisting}
Output
\begin{lstlisting}
<span class="email">racke@linuxia.de</span>
\end{lstlisting}
\end{frame}

\subsection{Conditions}
\begin{frame}[fragile]{Conditions}
HTML
\begin{lstlisting}
<div class="warnings">What's up?</div>
\end{lstlisting}
Specification
\begin{lstlisting}
<container name="warnings" class="warnings" value="message">
<value name="message" class="warnings"/>
</container>
\end{lstlisting}
Code
\begin{lstlisting}
$flute->process(values => {message => 'No coffee available'});
\end{lstlisting}
Output
\begin{lstlisting}
<div class="warnings">No coffee available</div>
\end{lstlisting}
\end{frame}

\subsection{Lists}
\subsubsection{Example}
\begin{frame}[fragile]{List Example}
HTML
\begin{lstlisting}
<tr class="cartitem">
<td class="name">Perl 6</td>
<td><input class="quantity" name="quantity" size="3" value="10"></td>
<td class="price">$$$</td>
</tr>
\end{lstlisting}
Spezifikation
\begin{lstlisting}
<list name="cart" class="cartitem" iterator="cart">
<param name="name" field="title"/>
<param name="quantity"/>
<param name="price"/>
</list>
\end{lstlisting}
\end{frame}

\begin{frame}[fragile]{List Example}
Code
\begin{lstlisting}
$flute = Template::Flute->new(template => $template,
                              specification => $specification,
                              iterators => {cart => \@cart});

$output = $flute->process;
\end{lstlisting}
\end{frame}

\subsubsection{Iterators}
Template::Flute uses iterators to retrieve list elements and insert them into
the document tree. This abstraction relieves us from worrying about where
the data actually comes from. We basically just need an array of hash
references and an iterator class with a next and a count method. For your
convenience you can create an iterator from Template::Flute::Iterator
very easily.

\begin{frame}{Iterators}
\begin{itemize}
  \item next method
  \item count method
  \item hash reference as return value
 \end{itemize}
\end{frame}

\subsubsection{Template::Flute::Iterator}
\begin{frame}[fragile]{Template::Flute::Iterator}
\begin{lstlisting}
use Template::Flute::Iterator;

Template::Flute::Iterator->new($cart);
\end{lstlisting}
\end{frame}

\subsubsection{Subclassing Template::Flute::Iterator}
\begin{frame}[fragile]{Subclassing Template::Flute::Iterator}
\begin{lstlisting}
package MyApp::Iterator;

use base 'Template::Flute::Iterator';

sub new {
    ...
    $self->seed([...]);
    return $self;
}
\end{lstlisting}
\end{frame}

\subsubsection{List with alternating rows}
\begin{frame}[fragile]{Lists with alternating rows}
\begin{lstlisting}
<table class="cart">
<tr class="cartheader">
<th>Name</th><th>Quantity</th><th>Price</th>
</tr>
<tr class="cartitem odd">
<td class="name">Perl 6</td>
<td><input class="quantity" name="quantity" size="3" value="10"></td>
<td class="price">$$$</td>
</tr>
<tr class="cartitem even">
<td class="name">Pro Git</td>
<td><input class="quantity" name="quantity" size="3" value="10"></td>
<td class="price">$$$</td>
</tr>
</table>
\end{lstlisting}
\end{frame}

\section{Hints}
Because the specification is separate from the template itself,
Template::Flute relies on hints to insert and replace text
inside of HTML attributes.

\begin{frame}{Hints}
\begin{itemize}
\item Targets
\item Expressions
\item Operators
\end{itemize}
\end{frame}

\subsection{Targets}
\begin{frame}[fragile]{Explicit and implicit targets}
Link (Template::Toolkit)
\begin{lstlisting}
<a href="[% login_url %]" class="url">Login</li>
\end{lstlisting}
Link (Template::Flute Specification)
\begin{lstlisting}
<value name="url" target="href"/>
\end{lstlisting}
Hidden form field (Template::Toolkit)
\begin{lstlisting}
<input type="hidden" name="sku" class="sku" value="[% sku %]">
\end{lstlisting}
Hidden form field (Template::Flute Specification)
\begin{lstlisting}
<value name="sku" class="sku"/>
\end{lstlisting}
\end{frame}

\subsection{Expressions}
\begin{frame}[fragile]{Expressions}
\begin{lstlisting}
<container name="summary" value="wishlist_name&&!wishlist_page">
<value name="wishlist-name" field="wishlist_name"/>
<value name="total" class="wishlist-sum-total" filter="currency"/>
<list name="cart_collections" class="wishlist-collections-line" iterator="cart_collections">
<param name="sku" class="btn-remove" target="value"/>
<param name="name" class="wishlist-title"/>
<param name="quantity" class="wishlist-qty"/>
<param name="wishlist-title" field="sku" target="href" op="append"/>
</list>
</container>
\end{lstlisting}
\end{frame}

\subsection{Operators}
\begin{frame}{Operators}
\begin{itemize}
\item append
\item toggle
\item hook
\end{itemize}
\end{frame}

\section{More Elements}
\subsection{Filters}
There are two types of filters for lists: global filters and
parameter filters. Global filters are applied to the complete
record of a list element and can be used to skip list items.
Parameter filters are applied to a single value in a list
element record.

\begin{frame}[fragile]{Filters}
Values
\begin{lstlisting}
<value name="total_cost" filter="currency"/>
\end{lstlisting}
Lists
\begin{lstlisting}
<param name="price" filter="currency"/>
\end{lstlisting}

\end{frame}

\begin{frame}{Filter types}
\begin{itemize}
\item{Built-in filters}
\item{Custom filter classes}
\item{Custom filter functions}
\end{itemize}
\end{frame}

\subsubsection{Builtin filters}
\begin{frame}{Builtin filters}
\begin{itemize}
\item date
\item currency
\item country\_name
\item language\_name
\item upper
\item strip
\item eol
\item nobreak\_single
\end{itemize}
\end{frame}

\subsubsection{Custom filter class}
\begin{frame}[fragile]{Custom filter class}
\begin{lstlisting}
package MyApp::Filter;

sub filter {
    my ($self, $value) = @_;

    return $value;
}

1;
\end{lstlisting}
\end{frame}

\subsubsection{Custom filter function}
\begin{frame}[fragile]{Custom filter function}
\begin{lstlisting}
sub link_filter {
    my $page = shift;
    my $url;

    $url = ...
    
    return $url;
}

$flute = new Template::Flute(specification_file => 'menu.xml',
						   template_file => 'menu.html',
						   filters => {link => \&link_filter},
						  );
\end{lstlisting}
\end{frame}

\subsubsection{Chained Filters}
\begin{frame}[fragile]{Chained Filters}
\begin{lstlisting}
<value name="note" filter="upper eol"/>
\end{lstlisting}
\end{frame}

\subsubsection{Global Filter}
\begin{frame}[fragile]{Global Filter}
\begin{lstlisting}
<specification name="menu" description="Menu">
<filter name="acl" field="permission"/>
<list name="menu" class="menu" table="menus">
<input name="name" required="1" field="menu_name"/>
<param name="label" field="name"/>
<param name="url" target="href" filter="link"/>
</list>
</specification>
\end{lstlisting}
\end{frame}

\subsection{I18N}
I18N support is very basic right now. You write a function for
translating text inside the HTML template and instantiate an
Template::Flute::I18N with a reference to this function. 
\begin{frame}[fragile]{I18N}
\begin{lstlisting}
sub translate {
    my $text = shift;

    ...

    return $text;
}

$i18n = new Template::Flute::I18N (\&translate);

$flute = new Template::Flute(specification_file => 'menu.xml',
						   template_file => 'menu.html',
						   database => $db_object,
						   i18n => $i18n,
						  );
\end{lstlisting}
\end{frame}

\subsubsection{I18N: Lookup Keys}
You can override the text in the HTML template passed to the
translation function with a lookup key in the specification.
\begin{frame}[fragile]{I18N: Lookup Keys}
\begin{lstlisting}
<i18n name="returnurl" key="RETURN_URL"/>
\end{lstlisting}
\end{frame}

\subsection{Forms}
\begin{frame}[fragile]{Forms: Specification}
\begin{lstlisting}
<specification name='search' description=''>
<form name='search'>
<field name='searchterm'/>
<field name='searchsubmit'/>
</form>
</specification>
\end{lstlisting}
\end{frame}

\subsubsection{Manipulating Forms}
The Template::Flute::Form class provides a number of methods
to manipulate the output of forms in the resulting HTML:

\begin{frame}{Forms: Manipulating}
  \begin{description}
  \item[set\_action] Changing form action
  \item[set\_method] Changing form method (GET, POST)
  \item[fill] Fill form fields
  \end{description}
\end{frame}

\section{PDF}
PDF generation starts just the same way as HTML template
processing. In fact, it might make sense to use the same template
for display in the browser and for producing the PDF document.

The conversion is running through 3 passes. First the position
and sizes of the boxes are calculated. Second the boxes are
partitioned throughout the pages in the PDF document.

\subsection{HTML to PDF}
\begin{frame}{HTML to PDF}
  \begin{itemize}
  \item HTML template processing
  \item PDF conversion (PDF::API2)
    \begin{enumerate}
    \item calculate
    \item partition
    \item render
    \end{enumerate}
  \item Inline CSS 
  \end{itemize}
\end{frame}

\begin{frame}[fragile]{PDF: Code}
\begin{lstlisting}
$flute = new Template::Flute (specification_file => 'pdf.xml',
                            template_file => 'pdf.html',
                            values => \%values);
$flute->process();

$pdf = new Template::Flute::PDF (template => $flute->template(),
                                file => 'invoice.pdf');
$pdf->process();
\end{lstlisting}
\end{frame}

\subsection{Import}
\begin{frame}[fragile]{PDF: Import}
\begin{lstlisting}
$import{file} = 'shippinglabel.pdf';
$import{scale} = 0.8;
$import{margin} = {left => '3mm', top => '6mm'};

$pdf = new Template::Flute::PDF (template => $flute->template(),
                                file => 'invoice.pdf',
                                import => \%import);
\end{lstlisting}
\end{frame}

\section{Dancer}
\begin{frame}[fragile]{Dancer Example}
\begin{lstlisting}
use Dancer;

get '/' => sub {
    return 'Hello world';
};

dance;
\end{lstlisting}
\end{frame}

\subsection{Fruits Demo}
\begin{frame}[fragile]{Fruits Demo}
\begin{lstlisting}
dancer -a Fruits
$EDITOR Fruits/lib/Fruits.pm
\end{lstlisting}

(see next slide)

\begin{lstlisting}
Fruits/bin/app.pl
\end{lstlisting}
\end{frame}

\begin{frame}[fragile]{Fruits Demo}
\begin{lstlisting}
package Fruits;

prefix '/fruits';

# route for image files
get '/*.jpg' => sub {
    my ($name) = splat;

    send_file "images/$name.jpg";
};

# route for fruits page
get qr{/?(?<page>.*)}  => sub {
    template 'fruits';
};
\end{lstlisting}
\end{frame}

\subsection{Dancer \& Template::Flute}
\begin{frame}[fragile]{Dancer \& Template::Flute}
\begin{lstlisting}
template: "template_flute"

engines:
  template_flute:
    iterators:
      fruits:
        class: JSON
        file: fruits.json
\end{lstlisting}
\end{frame}

% \subsection{Restrictions}
\section{Conclusion}
\subsection{Use Cases}
\begin{frame}{Current and Future Use Cases}
  \begin{itemize}
  \item Very Large Product Lists
  \item Shop Backend
    \begin{itemize}
    \item Product Editor
    \item Product Search \& Replace
    \end{itemize}
  \item PDF Invoices
  \item Template Engine for Interchange
 \end{itemize}
\end{frame}

\subsection{Roadmap}
\begin{frame}{Roadmap}
 \begin{itemize}
   \item Documentation \& Tests
   \item Empty lists, number of results
   \item Paging
   \item Trees 
   \item CSS selectors
 \end{itemize}
\end{frame}

\subsection{The End}
\begin{frame}{The End}
 \begin{description}
  \item[Git] git://github.com/racke/Template-Flute.git
  \item[CPAN] \url{http://search.cpan.org/dist/Template-Flute/}
  \item[]   \url{http://search.cpan.org/dist/Template-Flute-PDF/}
  \item[]   \url{http://search.cpan.org/dist/Dancer-Template-TemplateFlute/}
  \item[Talk]
    \url{http://www.linuxia.de/talks/ilpw2012/tf-beamer.pdf}
   \item[Questions] ???
 \end{description}
\end{frame}

\end{document}

%%% Local Variables: 
%%% mode: latex
%%% TeX-master: t
%%% End: 
