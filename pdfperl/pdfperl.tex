\usepackage[utf8]{inputenc}
\usepackage[T1]{fontenc}
\usepackage{mathptmx}
\usepackage[scaled=.90]{helvet}
\usepackage{courier}

\usepackage{beamerthemesplit}
\usepackage{verbatim}
\usepackage{hyperref}
\usepackage{graphicx}

\usepackage{listings}
\lstset{language=Perl,basicstyle=\footnotesize,tabsize=3,showstringspaces=false}

\title{PDF mit Perl}
\author[racke]{Stefan Hornburg (Racke)\\ \texttt{racke@linuxia.de}}
\date[]{Perl Mongers Hamburg, 6. August 2012}

\begin{document}
\maketitle{}

\begin{frame}
  \titlepage
\end{frame}

\tableofcontents

\section{PDF::API2}

\subsection{Einführung}

Beim Thema PDF mit Perl stößt man meistens auf PDF::API2,
entweder weil es direkt eingesetzt wird oder als Abhängigkeit
des verwendeten Moduls auftaucht.

\begin{frame}{PDF::API2}
\begin{itemize}
\item API nah an den PDF-Objekten
\item Großer Funktionsumfang
\end{itemize}
\end{frame}

\begin{frame}[fragile]{Hallo Welt}
\begin{lstlisting}
$pdf = PDF::API2->new;

$font = $pdf->corefont('Verdana');
$text = $page->text;

$text->font($font, 50);
$text->translate(150, 105);

$text->text_center('Hallo Welt');

$pdf->saveas('hallo_welt.pdf');
\end{lstlisting}
\end{frame}

\begin{frame}{Hallo Welt}
\includegraphics[page=1,width=.6\textwidth]{hallo_welt.pdf}
\end{frame}

% Koordinatensystem

\begin{frame}{Übersicht}
\begin{itemize}
\item Text
\item Bilder
\item Figuren
\end{itemize}
\end{frame}

\begin{frame}{Text}
\begin{itemize}
\item Fonttyp
\item Fontgröße
\item Farbe
\end{itemize}
\end{frame}

\subsection{Fonts}

Die folgenden Fonts werden von PDF::API2
unterstützt:

% Core Fonts and UTF8?
% http://www.perlmonks.org/?node_id=954373

\begin{itemize}
\item Core Fonts, in PDF eingebaut
\item True Type und Open Type
\item Postscript Fonts
\item CJK-Fonts für asiatische Schriften
\item Unicode Map Fonts
\end{itemize}

\begin{frame}{Fonts}
\begin{itemize}
\item Core Fonts (Courier, Verdana, Helvetica, ...)
\item True Type und Open Type Fonts
\item Postscript Fonts
\item CJK-Fonts für asiatische Schriften
\item Unicode Map Fonts
\end{itemize}
\end{frame}

\begin{frame}[fragile]{Core Font}
\begin{lstlisting}
use PDF::API2;

$pdf = PDF::API2->new;
$font = $pdf->corefont('Verdana');
\end{lstlisting}
\end{frame}

% note: Template::Flute::PDF only supports core fonts

\begin{frame}[fragile]{True Type Font}
\begin{lstlisting}
use PDF::API2;

$pdf = PDF::API2->new;
$font = $pdf->ttfont('/usr/share/fonts/truetype/gentium/GenAI102.ttf');
\end{lstlisting}
\end{frame}

\begin{frame}[fragile]{Fonts zwischenspeichern}
\begin{itemize}
\item \begin{verbatim}$pdf->corefont, $pdf->ttfont, ... \end{verbatim} 
\item Einbetten des Fonts bei jedem Aufruf
\item Große PDF-Dateien
\end{itemize}
\end{frame}

\begin{frame}{Bilder}
\begin{itemize}
\item PNG
\item JPEG
\item TIFF
\item PNM
\item GIF
\end{itemize}
\end{frame}

\begin{frame}{Bilder}
\begin{itemize}
\item PNG
\item JPEG
\item TIFF
\item PNM
\item GIF
\end{itemize}
\end{frame}

\begin{frame}[fragile]{Beispiel Bild}
\begin{lstlisting}
$pdf = PDF::API2->new;
$page = $pdf->page;
$gfx = $page->gfx;

$pic = $pdf->image_png('nitesi.png');

# original size
$gfx->image($pic, 20, 100);

# scaled
$gfx->image($pic, 20, 50, 0.5);

# width and height
$gfx->image($pic, 20, 22, 156, 18);

$pdf->saveas('nitesi.pdf');
\end{lstlisting}
\end{frame}

\begin{frame}{Beispiel Bild}
\includegraphics[page=1,width=.6\textwidth]{nitesi.pdf}
\end{frame}

\begin{frame}{Auswahl Figuren}
\begin{itemize}
\item Linie
\item Rechteck
\end{itemize}
\end{frame}

\begin{frame}[fragile]{Farbiges Rechteck}
\begin{lstlisting}
$pdf = PDF::API2->new;
$page = $pdf->page;
$gfx = $page->gfx;

$gfx->fillcolor('lightgreen');
$gfx->rect(20, 100, 60, 40);
$gfx->fill;
\end{lstlisting}
\end{frame}

\begin{frame}{Beispiel Rechteck}
\includegraphics[page=1,width=.6\textwidth]{rechteck.pdf}
\end{frame}

\subsection{Farben}

\subsection{Import}


% \subsection{Specials}
% \begin{frame}{Specials}
%   \begin{itemize}
%   \item Rotation
%   \item Crossing lines
%   \end{itemize}
% \end{frame}

\section{Andere Module}

% \section{Quirks}

\subsection{CAM::PDF}
\begin{frame}{CAM::PDF}
  \begin{itemize}
  \item PDF-Bearbeitung
  \item Tests
  \end{itemize}
\end{frame}

\subsection{PDF::WebKit}
\begin{frame}{PDF::WebKit}
  \begin{itemize}
  \item Konvertierung mit der WebKit-Engine
  \item wkhtmltopdf Kommandozeile
  \item HTML als String, Datei oder URL
  \end{itemize}
\end{frame}

Based on wkhtmltopdf commandline utility.

\begin{frame}[fragile]{HTML als String}
\begin{lstlisting}
use PDF::WebKit;

my $kit = PDF::WebKit->new(\$html);

$output = $kit->to_pdf;
\end{lstlisting}
\end{frame}

\begin{frame}[fragile]{URL}
\begin{lstlisting}
use PDF::WebKit;

my $url = 'http://www.csszengarden.com/?cssfile=/207/207.css&page=0';
my $kit = PDF::WebKit->new($url);

$output = $kit->to_pdf;
\end{lstlisting}
\end{frame}

\begin{frame}
\includegraphics[page=1,width=.6\textwidth]{zengarden.pdf}
\end{frame}

\begin{frame}{PDF::WebKit Nachteile}
  \begin{itemize}
  \item ``Statisches'' Programm
  \item Seitenumbruch
  \end{itemize}
\end{frame}

See manual page for "Page breaking.".

%   This version of wkhtmltopdf has been compiled against a version of QT without the wkhtmltopdf patches.  There‐
%       fore some features are missing, if you need these features please use th%e static version.

\section{Ausklang}

\subsection{Anwendungsbeispiele}
\begin{frame}{Anwendungsbeispiele}
 \begin{itemize}
  \item Versandlabels
  \item Rechnungen
  \item Procurement-Dokument
  \end{itemize}
\end{frame}

\subsection{Zum Schluß}
\begin{frame}{Zum Schluß}
 \begin{description}
  \item[Folien]
    \url{http://www.linuxia.de/talks/hhmongers2012/pdfperl-beamer.pdf}
   \item[Questions] ???
 \end{description}
\end{frame}

\begin{itemize}
\item PDF::Table
\item PDF::Boxer
\item PDF::TextBlock
\end{itemize}

Module, die nicht PDF::API2 verwenden:

\begin{itemize}
\item PDF::Create
\item PDF::Reuse
\item PDF::FromHTML
\item CAM::PDF
\end{itemize}

\end{document}

%%% Local Variables: 
%%% mode: latex
%%% TeX-master: t
%%% End: 
