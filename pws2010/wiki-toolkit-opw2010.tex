% TODO
% Metadata slide
% Wiki::Toolkit History, Author
% Functions, add more of them
% Address for Slides
%   pointer to Wiki.pm source code
% Glue => %Storage example
% Plugins => Diff example

% found at http://www.mrunix.de/forums/archive/index.php/t-39599.html
\usepackage[T1]{fontenc}
\usepackage{mathptmx}
\usepackage[scaled=.90]{helvet}
\usepackage{courier}

\usepackage{beamerthemesplit}
\usepackage{verbatim}
\usepackage{hyperref}
\usepackage{listings}
\lstset{language=Perl,basicstyle=\footnotesize,tabsize=3}

% - Introduction
% - Motivation
% - Application

\title{Wikis with Wiki::Toolkit}
\author[racke]{Stefan Hornburg (Racke)\\ \texttt{racke@linuxia.de}}
\date[OPW2010]{Perl Oasis, 16th January 2010}

\begin{document}

\begin{frame}
  \titlepage
\end{frame}

\tableofcontents

% \section{Preface}
Welcome to my presentation about Wiki::Toolkit, a toolkit for building
Wikis with Perl.

My name is Stefan Hornburg and I'm full time perl hacker, involved for
the most part in Interchange, an Open Source E-Commerce software 
written in Perl.

We installed the MoinMoin wiki on our website two years ago. While I was
quite pleased about it, it is still Python and unfortunately it broke 
into pieces after a Debian upgrade. Due to my limited knowledge about
Python and MoinMoin I wasn't able to fix it quickly.
That motivated me to search for a modern and easy-to-use Perl Wiki.

Wiki::Toolkit isn't an instant Wiki, but it allows you to choose
your tools of the trade. You can pick your tools for storing, formatting,
searching and other purpose, or even design your own.

The work on the Interchange Wiki isn't yet completed, it'll take a few
months more to get it done due to all the other projects I have to care
for.

Before we continue with the look at the archictecture of Wiki::Toolkit
I want to mention the real world applications I found using Wiki::Toolkit,

% Real World Applications
% - OpenGuides
% - ACT
\section{Architecture}

% Dispatch

We start with taking a look at the architecture of a Wiki::Toolkit wiki.

Wiki::Toolkit is written in a modern style, so all components of
it can be exchanged and subclassed.

\begin{frame}{Architecture}
 \begin{itemize}
  \item Storage
  \item Formatter(s)
  \item Plugins
  \item Search
  \item Glue
 \end{itemize}
\end{frame}

% \begin{frame}{Wiki Functions}
% \end{frame}

\section{Glue}

Before going into detail with the different tools that Wiki::Toolkit
provides, I show you how the glue looks like.

After selecting the tools you want to use in your Wiki::Toolkit wiki,
you create an instance of each of these tools and pass them to
the constructor for the Wiki::Toolkit class.

\subsection{Wiki::Toolkit Setup}

\begin{frame}[fragile]{Wiki::Toolkit Setup}
\begin{lstlisting}
my $store = new Wiki::Toolkit::Store::MySQL(dbname => 'wiki',
    dbuser => 'wiki',
    dbpass => 'wikisecret');

my $formatter = new Wiki::Toolkit::Formatter::UseMod;

my $search = new Wiki::Toolkit::Search::DBIxFTS(dbh => $store->dbh);

my $wiki = new Wiki::Toolkit(store => $store,
    formatter => $formatter,
    search => $search,
);
\end{lstlisting}
\end{frame}

\subsection{Basic Methods}

A Wiki node consists of:

\begin{itemize}
  \item text
  \item version
  \item metadata
  \item checksum
\end{itemize}

% handling conflicts

\begin{frame}[fragile]{Basic Methods}
\begin{lstlisting}
# create node
$wiki->write_node($name, $text, undef, \%metadata);

# retrieve and display node
my %node = $wiki->retrieve_node({name => $name, version => $version});
my $output = $wiki->format($node{content}, $node{metadata});
 
# update node
$wiki->write_node($name, $text, $node{checksum}, \%metadata);
\end{lstlisting}
\end{frame}

\section{Storage}

\subsection{Storage Backends}

All storage backends are derived from \verb+Wiki::Toolkit::Store::Database+.

\begin{frame}{Storage Backends}
 \begin{itemize}
  \item MySQL
  \item PostgreSQL
  \item SQLite
 \end{itemize}
\end{frame}

\subsection{Storage Setup}

% notes: cleardb, leaving tables alone
% cleardb doesn't erase tables for search backends

All storage backends can be easily initialized with the corresponding
\verb+Wiki::Toolkit::Setup+ classes.

\begin{frame}[fragile]
\frametitle{Storage Setup}
\lstinputlisting{scripts/setup_mysql}
\end{frame}

\subsection{Storage Tables}

\begin{frame}{Storage Tables}
 \begin{description}
  \item [content] Nodes
  \item [metadata] Metadata
  \item [internal\_links] Page Links
  \item [schema\_info] Schema Version
 \end{description}
\end{frame}

What I miss a little bit here, is a chance to provide a table prefix,
so these tables can happily live with tables from other applications
in the same database.

\subsection{Metadata}

% see also JSON plugin

Arbitrary metadata can be stored alongside with the node.

The metadata can be retrieved with the \verb+retrieve_node+ method.

\begin{frame}[fragile]{Metadata}
\begin{lstlisting}
# store metadata
my %metadata = {country => 'USA', 
    state => 'Florida',
    city => 'Orlando'};

$wiki->write_node($node, $content, $checksum, \%metadata);

# retrieve metadata
my %node = $wiki->retrieve_node('Perl Oasis');

for (keys %{$node{metadata}}) {
    print "Metadata $_: $node{$metadata}->{$_}->[0]\n";
}
\end{lstlisting}
\end{frame}

\section{Formatters}

Formatters are classes that providing methods for turning text in
Wiki format into HTML code.

\subsection{Methods}

The interface for Wiki::Toolkit formatters is really simple, it consists
of a mandatory \verb+format+ and an optional \verb+find_internal_links+
method.

\verb+format+ takes the text in the Wiki format and returns HTML code
suitable for display in a web page.

\verb+find_internal_links+ returns internal links in the Wiki.

\begin{frame}[fragile]{Methods}
\begin{lstlisting}

my $html = $formatter->format($content);

my @links = $formatter->find_internal_links($content);

\end{lstlisting}
\end{frame}

\subsection{Constructor Options}

% see contructor for Wiki::Toolkit:Formatter::UseMod

\begin{frame}[fragile]{Constructor Options}
\begin{lstlisting}

my $formatter = Wiki::Toolkit::Formatter::Default->new(
                 extended_links  => 0,
                 implicit_links  => 1,
                 allowed_tags    => [qw(b i)],
                 macros          => {},
                 node_prefix     => 'wiki.cgi?node=' );

\end{lstlisting}
\end{frame}

Extended links are usually links marked by square brackets, e.g.
\verb+[[Page]|Link name]+. The formatter class can decide about
the exact format.

Implicit links are links from CamelCase strings.

\subsection{Macros}

\begin{frame}[fragile]{Macros}
\begin{lstlisting}
macros => { qr/(^|\b)\@SEARCHBOX(\b|$)/ =>
            qq(<form action="wiki.cgi" method="get">
                <input type="hidden" name="action" value="search">
                <input type="text" size="20" name="terms">
                <input type="submit"></form>) }
\end{lstlisting}
\end{frame}

% \subsection{Internal Links}

% Note: \verb+rename_node+ function uses internal_links table.

\subsection{Formatters in CPAN}

\begin{frame}{Formatters in CPAN}
 \begin{itemize}
  \item MediaWiki
  \item UseMod
  \item MarkDown
  \item POD
 \end{itemize}
\end{frame}

\subsection{Multiple Formatters}

It is possible to use multiple formatters for one Wiki with
the Wiki::Toolkit::Formatter::Multiple class.

\begin{frame}[fragile]{Multiple Formatters}
\begin{lstlisting}
my $fmt_usemod = new Wiki::Toolkit::Formatter::UseMod;
my $fmt_mediawiki = new Wiki::Toolkit::Formatter::MediaWiki;

my $fmt = new Wiki::Toolkit::Formatter::Multiple (
      usemod => $fmt_usemod,
      mediawiki => $fmt_mediawiki,
      _DEFAULT => $fmt_usemod,
  );

...

my $output = $wiki->format($text, {formatter => 'mediawiki'});
\end{lstlisting}
\end{frame}

\begin{frame}[fragile]{Multiple Formatters}
\begin{lstlisting}
my $fmt_usemod = new Wiki::Toolkit::Formatter::UseMod;
my $fmt_mediawiki = new Wiki::Toolkit::Formatter::MediaWiki;

my $fmt = new Wiki::Toolkit::Formatter::Multiple (
      usemod => $fmt_usemod,
      mediawiki => $fmt_mediawiki,
      _DEFAULT => $fmt_usemod,
  );

...

my $output = $wiki->format($text, {formatter => 'mediawiki'});
\end{lstlisting}
\end{frame}

\begin{frame}[fragile]{Multiple Formatters}
\begin{lstlisting}
# write node
$metadata{formatter} = 'mediawiki';
$wiki->write_node($name, $text, $checksum, \%metadata);

# retrieve node
%node = $wiki->retrieve_node($name);

# display node
$wiki->format($node{content}, $node{metadata});
\end{lstlisting}
\end{frame}

\subsection{Writing Formatters}

% HINT to Text::WikiFormat;

\begin{frame}{Writing Formatters}
 \begin{itemize}
   \item Wiki::Toolkit::Formatter::Foo
   \item format() method
   \item Text::WikiFormat
 \end{itemize}
\end{frame}

\section{Search}

\subsection{Search Backends}

% Search::InvertedIndex
% DBIx::FullTextSearch

\begin{frame}{Search Backends}
 \begin{itemize}
  \item Wiki::Toolkit::Search::SII
  \item Wiki::Toolkit::Search::DBIxFTS
 \end{itemize}
\end{frame}

\subsection{Search Setup}

\begin{frame}[fragile]{Search Setup}
\lstinputlisting{scripts/setup_mysqlindex}
\end{frame}

\subsection{Search Example}

% note: setup DBIx::FullTextSearch indexes
% mention script

\begin{frame}[fragile]{Search Example}
\begin{lstlisting}
# first time setup
Wiki::Toolkit::Setup::DBIxFTSMySQL::setup($dbname, $dbuser, $dbpass, $dbhost);

my $store = new Wiki::Toolkit::Store::MySQL(dbname => 'wiki',
    dbuser => 'wiki',
    dbpass => 'wikisecret');

my $search = new Wiki::Toolkit::Search::DBIxFTS(dbh => $store->dbh);
my %oasis_nodes = $search->search_nodes('oasis');
\end{lstlisting}
\end{frame}

\section{Plugins}

% Wiki::Toolkit::Feed modules
% http://search.cpan.org/~dom/Wiki-Toolkit-0.78/lib/Wiki/Toolkit/Extending.pod

\begin{frame}[fragile]
\frametitle{Plugins}
\begin{lstlisting}
my $plugin = new Wiki::Toolkit::Plugin::Diff;
$wiki->register_plugin(plugin => $plugin);
%diff = $plugin->differences(node => 'FrontPage',
    left_version => 101,
    right_version => 105);
\end{lstlisting}
\end{frame}

\subsection{Plugins in CPAN}

\begin{frame}{Plugins in CPAN}
 \begin{description}
  \item[Diff] format differences between versions
  \item[Categorizer] category management
  \item[JSON] recent changes as JSON
  \item[Locator::Grid] manage co-ordinate data
  \item[Ping] ping various services on node updates
  \item[RSS::Reader] retrieve feeds for node inclusion
 \end{description}
\end{frame}


\begin{frame}{The End}
 \begin{description}
  \item[CPAN] http://search.cpan.org/perldoc?Wiki::Toolkit
  \item[Website] http://www.wiki-toolkit.org/
  \item[Talk]
    http://www.linuxia.de/talks/opw2010/wiki-toolkit-opw2010-beamer.pdf
   \item[Questions] ???
 \end{description}
\end{frame}

\section{Wiki Syntax}
\subsection{Markdown}
Homepage: \url{http://daringfireball.net/projects/markdown/syntax}

Wiki: \url{http://markdown.infogami.com/}

\subsubsection{Emphasis}

\begin{tabular}{ll}
\verb|*single asterisks*| & \textit{italic} \\
\verb|_single underscores_| & \textbf{bold} \\
\end{tabular}

% TODO: double asterisks / double underscores

\subsubsection{Links}
Internal links:

\begin{verbatim}
See my [About](/about/) page for details.
\end{verbatim}

\subsection{UseMod}
Homepage: \url{http://www.usemod.com/cgi-bin/wiki.pl}
Syntax: \url{http://www.usemod.com/cgi-bin/wiki.pl?TextFormattingRules}

\subsubsection{Emphasis}

\section{Resources}

Homepage: \url{http://www.wiki-toolkit.org/}

Mailing list: \url{http://www.earth.li/cgi-bin/mailman/listinfo/cgi-wiki-dev}

\subsection{Related Projects}
MojoMojo, a Catalyst \& DBIx::Class powered Wiki,
\url{http://search.cpan.org/perldoc?MojoMojo}.

\end{document}
    
%%% Local Variables: 
%%% mode: latex
%%% TeX-master: t
%%% End: 
