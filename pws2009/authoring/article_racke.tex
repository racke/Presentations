% CLONE TICKETS !!
% Konformit�t
% Anwendung

%% -*- mode: latex; -*-

\section{Request Tracker 3.8 und Plugins}

\subsection*{Autor}
Stefan Hornburg (Racke) \verb/<racke@linuxia.de>/

\subsection*{Kurzbiographie}
Stefan Hornburg arbeitet seit 1998 als Open Source Consultant mit den
Schwerpunkten Linux, Perl und Interchange. Als Captain der ICDEVGROUP leitet
er die Entwicklergruppe von Interchange und ist als Debian Maintainer f�r
verschiedene Serverpakete verantwortlich (u.a. Courier, Pure-FTPd und
Sympa).

\subsection*{Einf�hrung}
Request Tracker (RT) ist ein in Perl programmiertes
Trouble-Ticket-System. Neben gro�en Organisationen wie die NASA, MIT und
DESY wird RT auch f�r das Bugtracking von CPAN und Perl selbst verwendet. 

Dieser Beitrag dokumentiert zwei Vortr�ge, ``Request Tracker 3.8'' und
``Request Tracker Plugins''.

% TODO
% add introduction
% applies to both presentations
% Mason

\subsection{Neue Features}
\subsubsection{\WSIndex{Bookmarks}}
Bookmarks (markierte Tickets) dienen zum schnellen Auffinden von Tickets.
In der Einzelansicht eines Tickets befindet sich rechts oben ein Stern,
mit dem dieses markiert werden kann.

\subsubsection{\WSIndex{Anzeigetafeln}}
Anzeigetafeln (\WSIndex{Dashboards}) finden sich im Men� Werkzeuge auf der 
linken Seite.

Jede Anzeigetafel ist eine Sammlung von verschiedene Dingen im
Ticketsystem, genannt Portlets. Das k�nnen z.B. sein:

\begin{itemize}
\item Suchabfragen
\item Graphen
\end{itemize}

\subsubsection{Richtext Email}
Sollte dieses Feature unerw�nscht sein, kann es sowohl global als
auch f�r jeden einzelnen Benutzer deaktiviert werden.

Global in \verb/RT_SiteConfig.pm/:

\verb/Set(\$MessageBoxRichText, 0);/.

Individuell kann dieses Feature in den Einstellungen (de)aktiviert
werden.

\subsubsection{Email-Signaturen und -Verschl�sselung}
Eingehende Emails k�nnen entschl�sselt und verifiziert werden.
Ausgehende Emails k�nnen verschl�sselt und signiert werden.

\subsubsection{Beziehungsgraphen}
Tickets k�nnen auf unterschiedliche Art und Weise verkn�pft werden
\cite{racke:wikirelations}. Der Graph zu einem Ticket wird anzeigt
wie folgt: Ticketanzeige => Beziehungen => Graph.

Die Eigenschaften des Graphen wie die Art der dargestellten Beziehungen
k�nnen dort auch konfiguriert werden. Dabei ist zu beachten, da� kein
Graph dargestellt wird wenn die gew�nschten Eigenschaften nicht 
zutreffen.

Die Graphen k�nnen gespeichert werden und in ein Dashboard eingebunden
werden.

\subsubsection{Emailversand und \WSIndex{Digests}}
F�r den Emailversand kann der Benutzer die folgenden Einstellungen
w�hlen:
\begin{itemize}
\item Einzeln (Individual messages)
\item T�gliche Zusammenfassungen (Daily digests)
\item W�chentliche Zusammenfassungen (Weekly digests)
\item Deaktiviert (Suspended)
\end{itemize}
Der Benutzer ben�tigt das \WSIndex{ModifySelf}-Recht.

F�r den Versand der Zusammenfassungen mu� ein entsprechender Cronjob
eingerichtet sein.

\subsubsection{Timeout f�r Sitzungen}
Bisher wurden Sitzungen erst bei expliziter Abmeldung oder Schlie�en des
Browsers beendet. Mit einem Cronjob kann nun auch ein Timeout zur Ende
einer Sitzung f�hren.

\subsubsection{Plugins}
%\subsubsection{Standalone Server}
\subsubsection{Visual Style}
\subsubsection{Bug Fixes}

\subsection{Installation}
\subsubsection{Installation aus den Quellen}
Zun�chst das Tararchiv der aktuellen Version herunterladen und 
auspacken. Dann in das Verzeichnis mit dem Archivinhalt wechseln:
\begin{verbatim}
wget http://download.bestpractical.com/pub/rt/release/rt.tar.gz
tar -xzvf rt.tar.gz
cd rt-3.8.2
\end{verbatim}
Anschlie�end empfiehlt es sich, die README-Datei zu lesen.

\begin{tabular}{|l|l|l|}
  \verb/--prefix/ & Installationsverzeichnis & \verb%/opt/rt3% \\
  \verb/--enable-gpg/ & GPG Support & Ja\\
  \verb/--enable-graphviz/ & GraphViz Charts & Nein
\end{tabular}
\begin{verbatim}

./configure
/opt/rt3
make testdeps
make fixdeps
make install (as root)
edit configuration
/opt/rt3/etc/RT_SiteConfig.pm
\end{verbatim}
\subsubsection{Debian}
Pakete f�r RT 3.8.2 befinden sich im experimentellen Zweig.
\subsection{Upgrade}
\subsection{Konfiguration}
\subsubsection{Cronjobs}
\begin{verbatim}
/etc/cron.daily/request-tracker3.8
# Send daily RT emails.
/usr/sbin/rt-email-digest -m daily

/etc/cron.weekly/request-tracker3.8
# Send weekly RT emails
/usr/sbin/rt-email-digest -m weekly
\end{verbatim}
\subsection{Schnittstellen}


\subsection{Plugins}
% TODO
% - Installationsverzeichnis
Wir beschr�nken uns auf die Beschreibung von Plugins f�r RT 3.8.

\subsubsection{Installation und Konfiguration}
Plugins f�r RT sind �blicherweise als Module vom CPAN erh�ltlich.
Die Installation wird also mit einem CPAN-Werkzeug durchgef�hrt:
\begin{verbatim}
cpan> install RT::Authen::ExternalAuth
\end{verbatim}
Alternativ kann man auch die Quellen herunterladen und wie gewohnt
installieren:
\begin{verbatim}
perl Makefile.PL
make
make install
\end{verbatim}
Damit die Installation des Plugins auch funktioniert, wenn RT sich nicht in
\verb%/opt/rt3% befindet, setzt man die Umgebungsvariable
%\WSIndex{\verb%RTHOME%}. 

\begin{verbatim}
RTHOME=/usr/share/request-tracker3.8 perl Makefile.PL
make
make install
\end{verbatim}

Nach der Installation des entsprechenden Perl-Moduls wird RT angewiesen,
dieses auch als Plugin zu laden:
\begin{verbatim}
Set(@Plugins,(qw(RTx::Calendar)));
\end{verbatim}
\subsubsection{\WSIndex{RTx::Email::Completion}}
Das Plugin \verb/RTx::Email::Completion/ erweitert RT um eine automatische
Vervollst�ndigung von Emailadressen.
\subsubsection{\WSIndex{RT::Authen::ExternalAuth}}
Weitere Dokumentation findet sich im RT Wiki \cite{racke:wikixauth}.
\subsubsection{\WSIndex{RTx::Calendar}}
Das Plugin \verb/RTx::Calendar/ erlaubt eine kalendarische Darstellung der
mit den Tickets verkn�pften Daten.

Voraussetzung f�r RTx::Calendar sind die folgenden Perlmodulen vom
CPAN: \WSIndex{Date::Ical}, \WSIndex{Data::ICal}, \WSIndex{DateTime::Set}.

Aktiviert wird das Plugin wie bekannt in \verb/RT_SiteConfig.pm/:

\begin{verbatim}
Set(@Plugins,(qw(RTx::Calendar)));
\end{verbatim}

Bei den Voreinstellungen gibt es dann einen weiteren Punkt \emph{Calendar}.

Um den Kalendar in ``RT auf einen Blick'' einbinden zu k�nnen, gilt es
\verb/MyCalendar/ zu den verf�gbaren Komponenten hinzuzuf�gen:

\begin{verbatim}
Set($HomepageComponents, [qw(QuickCreate Quicksearch MyCalendar
     MyAdminQueues MySupportQueues MyReminders
     RefreshHomepage Dashboards)]);
\end{verbatim}

Dann kann der Kalendar der pers�nlichen Homepage hinzugef�gt werden.

%  To enable private searches ICal feeds, you need to give
%  CreateSavedSearch and LoadSavedSearch rights to your users.

\begin{thebibliography}{99}
\bibitem{racke:wikirelations} \emph{Relationships}.\\
\texttt{http://wiki.bestpractical.com/view/Relationships}.
\bibitem{racke:wikixauth} \emph{External Authentication and Information for
    Request Tracker}.\\
\texttt{http://wiki.bestpractical.com/view/ExternalAuth}.
\bibitem{racke:rtessentials} Jesse Vincent, Robert Spier, Dave Rolsky,
  Darren Chamberlain \& Richard Foley.
\emph{RT Essentials}.
O'Reilly, Sebastopol, California, 2005.
\end{thebibliography}


