\usepackage[utf8]{inputenc}
\usepackage[T1]{fontenc}
\usepackage{mathptmx}
\usepackage[scaled=.90]{helvet}
\usepackage{courier}
\usepackage{caption}
\captionsetup{labelformat=empty,labelsep=none}
\usepackage{verbatim}
\usepackage{hyperref}
\usepackage{listings}
% strikethrough (\sout)
\usepackage{ulem}
\lstset{language=Perl,basicstyle=\normalsize,tabsize=3,showstringspaces=false}

\title{Mobile Shop}
\author[racke]{Stefan Hornburg (Racke)\\  \texttt{racke@linuxia.de}}
\date{Perl::Dancer Conference 2014, Hancock, 9th October 2014}

\begin{document}
\maketitle{}

\begin{frame}
  \titlepage
\end{frame}

\tableofcontents

This presentation is about a mobile shop project for Calevo.

\begin{frame}{Calevo}
\begin{center}
  \includegraphics{pics/calevo.jpg}
\end{center}
\begin{itemize}
\item Products for riders and horses
\item Since 15 years with Interchange 
\end{itemize}
\end{frame}

Instead of rewriting the old Interchange5 application, we started
to build a Dancer application from scratch.

We are using the same database as for the regular shop though.

Planned release at this end, but it doesn't work out.

\begin{frame}{Calevo Mobile}
\begin{itemize}
\item \url{mobile.calevo.com}
\item Coming soon
\end{itemize}
\end{frame}

\section{Architecture}
\begin{frame}{Architecture}
\begin{itemize}
\item Dancer
\item Template::Flute
\item DBIx::Class
\item Jquery Mobile
\end{itemize}
\end{frame}

\section{Features}

% Solr updates

\subsection{Search}
\begin{frame}{Search}
\begin{itemize}
\item Solr
\item Ajax Autocomplete
\end{itemize}
\end{frame}

\subsection{I18N}
\begin{frame}{I18N}
\begin{itemize}
\item Autodetect
\item Translations
\end{itemize}
\end{frame}

\begin{frame}[fragile]{I18N with Template::Flute}
\begin{lstlisting}
engines:
  template_flute:
    i18n:
      class: CalevoMobile::Lexicon
      method: try_to_translate
\end{lstlisting}
\end{frame}

\begin{frame}[fragile]{I18N with Template::Flute}
\begin{lstlisting}
package CalevoMobile::Lexicon;

use Moo;
use Dancer ':syntax';
use Dancer::Plugin::SimpleLexicon;

sub try_to_translate {
    my ($self, $string) = @_;
    my $translated = l($string);
    # debug(to_dumper($string));
    return $translated;
}
\end{lstlisting}
\end{frame}

\subsection{Payment}
\begin{frame}{Payment}
\begin{itemize}
\item PayPal
\item IPayment
\item Cash in Advance (Vorkasse)
\end{itemize}
\end{frame}

\subsection{PayPal}
\begin{frame}[fragile]{PayPal}
\begin{itemize}
\item Business::PayPal::API::ExpressCheckout
\item CalevoMobile::Routes::PayPal
\begin{lstlisting}
prefix '/paypal';
post '/setrequest' => sub { .. };
# Checkout response from PayPal
get '/getrequest' => sub {
prefix undef;
\end{lstlisting}
\end{itemize}
\end{frame}

\subsection{IPayment}
\begin{frame}[fragile]{IPayment}
\begin{itemize}
\item Business::OnlinePayment::IPayment
\item Silent CGI
\end{itemize}
\end{frame}


\section{Conclusion}

\subsection{Slides}

\begin{frame}{Slides}
Slides:
\url{http://www.linuxia.de/talks/perldancer2014/mobileshop-en-beamer.pdf}
\end{frame}


\end{document}

%%% Local Variables: 
%%% mode: latex
%%% TeX-master: t
%%% End: 
