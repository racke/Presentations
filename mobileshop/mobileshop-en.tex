\usepackage[utf8]{inputenc}
\usepackage[T1]{fontenc}
\usepackage{mathptmx}
\usepackage[scaled=.90]{helvet}
\usepackage{courier}
\usepackage{caption}
\captionsetup{labelformat=empty,labelsep=none}
\usepackage{verbatim}
\usepackage{hyperref}
\usepackage{listings}
% strikethrough (\sout)
\usepackage{ulem}
\lstset{language=Perl,basicstyle=\normalsize,tabsize=3,showstringspaces=false}

\title{Mobile Shop}
\author[racke]{Stefan Hornburg (Racke)\\  \texttt{racke@linuxia.de}}
\date{Perl::Dancer Conference 2014, Hancock, 9th October 2014}

\begin{document}
\maketitle{}

\begin{frame}
  \titlepage
\end{frame}

\tableofcontents

This presentation is about a mobile shop project for Calevo.

\begin{frame}{Calevo}
\begin{center}
  \includegraphics{pics/calevo.jpg}
\end{center}
\begin{itemize}
\item Products for riders and horses
\item Since 15 years with Interchange 
\end{itemize}
\end{frame}

Instead of rewriting the old Interchange5 application, we started
to build a Dancer application from scratch.

We are using the same database as for the regular shop though.

\section{Architecture}
\begin{frame}{Architecture}
\begin{itemize}
\item Dancer
\item DBIx::Class
\end{itemize}
\end{frame}

\section{I18N}
\begin{frame}{I18N}
\begin{itemize}
\item Autodetect
\end{itemize}
\end{frame}

\section{Conclusion}

\subsection{Slides}

\begin{frame}{Slides}
Slides:
\url{http://www.linuxia.de/talks/perldancer2014/mobileshop-en-beamer.pdf}
\end{frame}


\end{document}

%%% Local Variables: 
%%% mode: latex
%%% TeX-master: t
%%% End: 
