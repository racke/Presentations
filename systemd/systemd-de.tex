\begin{document}
\maketitle

\begin{frame}
  \titlepage
\end{frame}

\cleardoublepage

\tableofcontents

\cleardoublepage

\section{Einführung}

Systemd ist ein umfangreiches Management-Tool für System und Services mit
entsprechenden Vor- und Nachteilen. SysV Init ist nur für ein kleines Subset verantwortlich.

\begin{frame}{Systemd}
  \begin{itemize}
    \item Geschwindigkeit
    \item Parallele Ausführung
    \item Bereitstellung von Sockets
  \end{itemize}
\end{frame}

\section{Bestandteile}

\begin{frame}{Bestandteile}
  \begin{itemize}
\item Services
\item Logging
\item Hostname/Locale/Keyboard/Time/...
\item Temporäre Dateien
\item Timers (Cron-Ersatz)
\item ...
\end{itemize}
\end{frame}

\subsection{Unit-Typen}

\begin{frame}{Units}
  \begin{itemize}
  \item Services (.service)
  \item Gruppen (.target)
  \item Dateisysteme (.mount / .automount)
  \item Überwachung (.path)
  \item Sockets (.socket)
  \item Netzwerke (.network)
  \item Slice (.slice)
  \item ...
\end{itemize}
\end{frame}

\subsection{Targets}
Targets sind einfach eine Gruppe von Unitfiles.

Außerdem gibt es Targets, die beim Bootvorgang aufgerufen werden und grob
die Funktion von Runlevels ausüben:

\begin{frame}[fragile]{Targets}
  \begin{itemize}
  \item graphical.target
  \item multiuser.target
  \item rescue.target
  \item default.target
  \end{itemize}
\end{frame}

Dabei entspricht graphical grob dem Runlevel 1, multiuser dem Runlevel 3,
rescue dem Runlevel 1 und default ist ein Alias für das aktive Target
beim Booten.

\section{Kommandozeile}

\begin{frame}{Kommandozeile}
  \begin{itemize}
  \item systemctl
  \item journalctl
  \end{itemize}
\end{frame}

\subsection{systemctl}

\begin{frame}[fragile]{systemctl}
 \begin{lstlisting} 
    systemctl status nginx
    systemctl stop nginx
    systemctl start nginx
    systemctl enable nginx
    systemctl disable nginx
  \end{lstlisting}
\end{frame}

\subsection{Fehlermeldungen}

systemctl start gibt keine Fehlermeldungen aus. Im Default-Modus kann es sogar dazu kommen, daß gar keine Meldng kommt.

\begin{frame}[fragile]{systemctl}
\begin{lstlisting}
root@dev-debian9:~# systemctl start nginx
Job for nginx.service failed because the control process exited with error code.
See "systemctl status nginx.service" and "journalctl -xe" for details.
\end{lstlisting}
\end{frame}

\subsection{Übersicht}

Mit SysV Init finden sich die Skripte für die Services in \verb|/etc/init.d|.

\begin{frame}{Units}
Units auflisten:

    systemctl list-units

Unitdateien:

systemctl list-unit-files
\end{frame}

\subsection{Status Unitdateien}

\begin{frame}[fragile]{Status Unitdateien}
\begin{itemize}
\item enabled
\item disabled
\item generated
\item masked
\item static
\end{itemize}
\end{frame}

\subsection{Filter}

\begin{frame}{Filter}
Unit-Typ (Services):

    systemctl list-units --type=service

Inaktive und fehlende Services:

    systemctl list-units --type=service --all  
\end{frame}
  
\subsection{Targets}

\begin{frame}{Targets}
 systemctl set-default multi-user.target
\end{frame}

\section{Unitdateien}

\subsection{Dateisystem}

\begin{frame}[fragile]{Dateisystem}
  \begin{itemize}
\item[Standard] \verb|/lib/systemd/system|
\item[Custom] (Lokal, Ansible) \verb|/etc/systemd/system|
\end{itemize}
\end{frame}

\subsubsection{target.wants}

\subsubsection{Vorhandene Unitdateien bearbeiten}

% https://wiki.archlinux.org/index.php/systemd\#Editing_provided_units

 systemctl daemon-reload

 drop-in Unitdateien
 \subsection{Struktur Unitdatei}

\begin{frame}[fragile]{Struktur Unitdatei}
 \begin{lstlisting}
[Unit]
Description=The PHP FastCGI Process Manager
After=network.target

[Service]
Type=notify
PIDFile=/var/run/php5-fpm.pid
ExecStartPre=/usr/lib/php5/php5-fpm-checkconf
ExecStart=/usr/sbin/php5-fpm --nodaemonize --fpm-config /etc/php5/fpm/php-fpm.conf
ExecReload=/bin/kill -USR2 $MAINPID

[Install]
WantedBy=multi-user.target
\end{lstlisting}
\end{frame}

\section{Protokollierung}

journalctl ist für die Protokollierung zuständig. Die Daten werden in einem Binärformat gespeichert.

\begin{frame}{Protokollierung}
\begin{itemize}
\item Binärformat
\item Weiterleitung an Syslog
\end{itemize}
\end{frame}

\begin{frame}[fragile]{Protokollierung}
  \begin{lstlisting}
    journalctl -f
    \end{lstlisting}
\end{frame}
  
\subsection{Filter}

\begin{frame}[fragile]{Filter}
  \begin{itemize}
  \item[Service] \verb|journalctl -u nginx|
  \item[Zeit] \verb|journalctl --since 09:00 --until "1 hour ago"|
  \item[Priorität] \verb|journalctl -p err|
  \end{itemize}
\end{frame}

\section{Systemd \& Sysvinit}

\begin{frame}{Systemd \& SysV-Init}
  \begin{itemize}
  \item Initskripte
  \item Runlevel
  \item Inetd
  \end{itemize}
\end{frame}

\subsection{Einbindung alte Services}

\begin{frame}
systemd-sysv-generator

 systemctl show --property=UnitPath

 cat /run/systemd/generator.late/ntp.service
\end{frame}

 \subsubsection{Runlevel}

\begin{frame}{Runlevel}
 Obsolet in Systemd
\end{frame}

\section{Abschluß}
\subsection{Vor- und Nachteile}

\begin{frame}
\begin{itemize}
\item monolithish keine manapage keine help
\item widerspricht Unix-Philosophie
\item Abhängigkeiten
\end{itemize}
\end{frame}

\subsection{Hintergründe}

\begin{frame}
cgroups
\end{frame}

\subsection{Referenzen}

\begin{frame}
\href{https://wiki.archlinux.org/index.php/systemd}{ArchWiki}

\href{https://www.digitalocean.com/community/tutorials/understanding-systemd-units-and-unit-files}{https://www.digitalocean.com/community/tutorials/understanding-systemd-units-and-unit-files}

\href{https://www.digitalocean.com/community/tutorials/how-to-use-journalctl-to-view-and-manipulate-systemd-logs}{https://www.digitalocean.com/community/tutorials/how-to-use-journalctl-to-view-and-manipulate-systemd-logs}
\end{frame}

\end{document}

%%% Local Variables: 
%%% mode: latex
%%% TeX-master: t
%%% End: 
