\begin{document}
\maketitle

\begin{frame}
  \titlepage
\end{frame}

\cleardoublepage

\tableofcontents

\cleardoublepage

% \section{Einführung}

\note{
  A few words about myself:
}

\begin{frame}{/me}
\begin{itemize}
\item Nick: racke
\item Selbstständiger Programmierer seit 1998
\item Webanwendungen
\item Ecommerce
\item Datendanken
\item Provisionierung mit Ansible
\item Kunden in Deutschland, Österreich, Schweiz, USA, Australien, ...
\end{itemize}
\end{frame}

\note{
  One of my customers is the US Department of State.
  I'm responsible for a number of their web sites, for example
  the procurement solution eShop.
}

\begin{frame}{}
% https://pixabay.com/en/stress-man-hand-flame-burn-fire-864141/
\begin{figure}[!ht]
\centering
\includegraphics[width=1\linewidth]{img/stress.jpg}
\end{figure}
\end{frame}

\section{Einführung}

\note{
  Bevor es jedoch ans Eingemachte geht, starten wir erstmal mit einer kurzen Einführung in
  die Architektur von DBIx::Class und welche Vorteile DBIx::Class bietet.
}

\begin{frame}{Einführung}
  \begin{itemize}
  \item Architektur
  \item Vorteile
  \end{itemize}
\end{frame}

\subsection{DBIx::Class Architektur}

\begin{frame}{DBIx::Class Architektur}
  \begin{itemize}
  \item Datenbank \textit{perldance}
  \item Schemaobjekt \textit{PerlDance::Schema}
  \end{itemize}
\end{frame}

\begin{frame}[fragile]{DBIx::Class Architektur}
  \begin{itemize}
  \item Tabellen
    \begin{itemize}
    \item \textit{countries}
    \item \textit{users}
    \end{itemize}
  \item \textbf{ResultSet-Klassen}
    \begin{itemize}
      \item \textit{PerlDance::Schema::ResultSet::Country}
      \item \textit{PerlDance::Schema::ResultSet::User}
    \end{itemize}
  \end{itemize}
\end{frame}

\begin{frame}[fragile]{DBIx::Class Architektur}
  \begin{itemize}
  \item Datensätze
\verb|{"country_iso_code": "de", "name" : "Germany"} |

\verb|{"username": "racke", "email" : "racke@racke.pm"} |

  \item \textbf{Result-Klassen}
    \begin{itemize}
      \item \textit{PerlDance::Schema::Result::Country}
      \item \textit{PerlDance::Schema::Result::User}
    \end{itemize}
  \end{itemize}
\end{frame}

Die Interaktion mit einer SQL-Datenbank findet mittels SQL-Befehle statt.

\begin{frame}{DBIx::Class Architektur}
  \begin{itemize}
  \item SQL-Befehle
    \begin{itemize}
    \item INSERT
    \item UPDATE
    \item DELETE
    \item SELECT
    \end{itemize}
  \end{itemize}
\end{frame}

\begin{frame}[fragile]{DBIx::Class Architektur}
  INSERT
  \begin{lstlisting}
$country = $schema->resultset('Country')->create( { ... } );
  \end{lstlisting}
  UPDATE
  \begin{lstlisting}[language=SQL]
$country->update( { ... } );
  \end{lstlisting}
  DELETE
  \begin{lstlisting}
$country->delete;
  \end{lstlisting}
  SELECT
   \begin{lstlisting}
$country = $schema->resultset('Country')->search( { ... } );
  \end{lstlisting}
\end{frame}

% https://pixabay.com/en/steam-locomotive-blowing-axis-pivot-1899603/

Wir werden später noch feststellen, da die ResultSet-Klassen der Dreh- und Angelpunkt in einem DBIx::Class-Schema sind.

\begin{frame}{ResultSet-Klassen}
\begin{figure}[!ht]
\centering
\includegraphics[width=1\linewidth]{img/pivot.jpg}
\end{figure}
\end{frame}

\begin{frame}[fragile]{DBIx::Class Vorteile}
\begin{itemize}
\item OO anstatt von SQL
\item Abstrahierung von SQL-Implementierungen
\item Businesslogik
\item Performance
\item ResultSet Features
\item Ökosystem
\begin{itemize}
\item IRC / Mailing list
\item Komponenten
\item Helpers
\end{itemize}
\end{itemize}
\end{frame}

\subsection{Businesslogik}
\begin{frame}{Businesslogik}
% move business logic into schema
% https://pixabay.com/en/gear-gears-euro-forex-dollar-384743/
\begin{figure}[!ht]
\centering
\includegraphics[width=0.75\linewidth]{img/business-logic.jpg}
\end{figure}
\end{frame}

\begin{frame}{Vorteile Businesslogik}
\begin{itemize}
\item Mehrere Verbraucher
\begin{itemize}
\item Webanwendungen
\item Cronjobs, Skripte
\item Testumgebungen
\end{itemize}
\item Implementation verbergen
\begin{itemize}
\item SQL-Dialekte
\item Anwendungsansicht
\end{itemize}
\item Änderungen / DRY
\end{itemize}
\end{frame}

\begin{frame}{Beispiele Businesslogik}
\begin{itemize}
\item Preisberechnungen
\begin{itemize}
\item Angebote
\item Rabatt
\end{itemize}
\item Synchronisierung ERP
\end{itemize}
\end{frame}

\subsection{Performance}

All the experience, tests and different areas in which
DBIx::Class is applied makes it perform better than
handwritten SQL is most cases.

\begin{frame}{Performance}
\begin{itemize}
\item Erfahrung
\item Testsuiten
\item Use cases
\end{itemize}
\end{frame}

\section{Projekte}

\subsection{Interchange6::Schema}

\begin{frame}{Interchange}
\begin{figure}[!ht]
\centering
\includegraphics[width=1\linewidth]{img/interchange.jpg}
\end{figure}
\end{frame}

Interchange war schon lange nicht mehr auf dem neuesten Stand.

\begin{frame}{Interchange}
  \begin{itemize}
    \item CGI
    \item Embedded SQL
    \item \sout{Modern Perl}
  \end{itemize}
\end{frame}

Das war auch ein Grund, im Jahre 2013 eine Konferenz im beschaulichen
Hancock zu organisieren.


\begin{frame}{eCommerce Innovation 2013}
\begin{figure}[!ht]
\centering
\includegraphics[width=1\linewidth]{img/lodges-at-night.jpg}
\end{figure}
\end{frame}

An einem runden Tisch haben wir über das Datenbankschema diskutiert und nach
der Konferenz mit der Arbeit an \textit{Interchange6::Schema} begonnen.

% \begin{frame}[fragile]{Interchange}
%   \begin{itemize}
%   \item Interchange6::Schema
%   \item Interchange6
%   \item Dancer2::Plugin::Interchange6
%   \item DemoShop
%   \item IceCat
%   \end{itemize}
% \end{frame}


\begin{frame}{Interchange6::Schema}
\begin{itemize}
\item Candy
\item Individuelle ResultSet-Elternklasse
\item Komponenten/Helpers
\item Tests mit Test::Roo
\item ...
\end{itemize}
\end{frame}

% Picture: https://pixabay.com/en/cosmiques-ridge-granite-summit-2333885/

\begin{frame}{Interchange6::Schema}
\begin{figure}[!ht]
\centering
\includegraphics[width=1\linewidth]{img/rocky-ridge.jpg}
\end{figure}
\end{frame}

\begin{frame}{Fazit}
  \begin{itemize}
  \item Mehr als ein ORM
  \item ResultSet
  \item Ökosystem
  \item Dokumentation
%    \begin{itemize}
%      \item Gut für Core (Kochbuch)
%      \item ResultSet/Ökosystem nicht integriert
%    \end{itemize}
  \end{itemize}
\end{frame}

% \subsection{eShop}
% \begin{frame}{eShop}
% \begin{figure}[!ht]
% \centering
% \includegraphics[width=1\linewidth]{img/eshop.png}
% \end{figure}
% \end{frame}

% \subsection{Dancer}

% \begin{frame}{Dancer / DBIx::Class}
% \begin{itemize}
% \item Dancer
% \item DBIx::Class
% \item Interchange6
% \end{itemize}
% \end{frame}

% \note{We organized the second Perl Dancer conference October,}

% \begin{frame}{Perl Dancer Conference}
% \begin{figure}[!ht]
% \centering
% \includegraphics[width=1\linewidth]{img/perl-dancer-homepage-logo.png}
% \end{figure}
% \end{frame}

% \note{where he had also a one day DBIx::Class training.

% Here you see my co-trainers, ribasushi (mastermind behind
% DBIx::Class) and Peter Mottram doing the last preparations
% for the training:
% }

% \begin{frame}{DBIx::Class Training}
% \begin{figure}[!ht]
% \centering
% \includegraphics[width=1\linewidth]{img/training-preps.jpg}
% \end{figure}
% \end{frame}





\subsection{DBIx::Class Mengenlehre}

\note{The most important feature of DBIx::Class is the Resultset,
which we will examine through an example soon.}

\begin{frame}[fragile]{DBIx::Class Mengenlehre}
\begin{figure}[!ht]
\centering
\includegraphics[width=0.4\linewidth]{img/frew.png}
\caption{Arthur Axel "fREW" Schmidt}
\end{figure}
\centering
\href{http://www.perladvent.org/2012/2012-12-21.html}{Set Based DBIx::Class}
\end{frame}

\begin{frame}{DBIx::Class Mengenlehre}
\begin{itemize}
\item Verkettung
\item Relationship Traversal
\item Subqueries
  \begin{itemize}
    \item Korrelierte Subqueries
  \end{itemize}
\end{itemize}
\end{frame}

\begin{frame}{ResultSet anwenden}
\begin{itemize}
\item Einfache SQL-Abfrage
\item Entsprechende DBIx::Class-Suche
\end{itemize}
\end{frame}

\begin{frame}[fragile]{Einfache SQL-Abfrage}
Liste der Vorträge:
\begin{lstlisting}
SELECT talks_id, author_id, conferences_id, duration,
title, tags, abstract, url, comments, accepted, confirmed, 
lightning, scheduled, start_time, room, survey_id 
FROM talks 
WHERE accepted is TRUE 
AND conferences_id = 1 
AND room != '' 
AND start_time >= '2015-10-22 00:00:00'
AND start_time <= '2015-10-23 00:00:00'
\end{lstlisting}
\end{frame}

\begin{frame}[fragile]{Einfache SQL-Abfrage}
Mit DBIx::Class:
\begin{lstlisting}
my $talks = $schema->resultset('Talk')->search(
    {
        -bool          => 'accepted',
        conferences_id => 1,
        room           => { '!=' => '' },
        start_time     => {
            '>=' => '2015-10-22 00:00:00'
            '<=' => '2015-10-23 00:00:00'
            },
    },
);
\end{lstlisting}
\end{frame}

\subsection{Wahrheit über das ResultSet}
\note{
  If you hear the term ResultSet, you probably thing
  we are talking about a number of results aka
  table rows.
}

\begin{frame}{Wahrheit über das ResultSet}
\begin{figure}[!ht]
\centering
\includegraphics[width=0.5\linewidth]{img/pdc_users.jpg}
\end{figure}
\end{frame}

% https://pixabay.com/en/wrong-way-sign-road-caution-167535/

\begin{frame}{}
\begin{figure}[!ht]
\centering
\includegraphics[width=1\linewidth]{img/wrong-way.jpg}
\end{figure}
\end{frame}

\begin{frame}[fragile]{Wahrheit über das ResultSet}
% \centering
\sout{ResultSet}

\begin{lstlisting}
isa(Abfrageplan);
\end{lstlisting}

\end{frame}

\begin{frame}[fragile]{Wahrheit über das ResultSet}

Hier wird keine SQL-Abfrage ausgeführt:

\begin{lstlisting}
my $talks = $schema->resultset('Talk')->search(...);
\end{lstlisting}

Hier schon:

\begin{lstlisting}
my $first_talk = $schema->resultset('Talk')
                 ->search(...)->first;
\end{lstlisting}

\end{frame}

\section{ResultSet-Baukasten}

Composability means that you don't need to construct the
complete at once, but compose it together, e.g with
chaining.

The underlying mechanism is that \verb|->search| on a
ResultSet actually doesn't search.

% https://pixabay.com/en/regulation-screw-colorful-color-261927/

\begin{frame}{ResultSet-Baukasten}
\begin{figure}[!ht]
\centering
\includegraphics[width=0.70\linewidth]{img/baukasten.jpg}
\end{figure}
\end{frame}

\begin{frame}{ResultSet-Baukasten}
\begin{itemize}
\item Verkettung
\item Vordefinierte Suche
\item Korrelierte Subqueries
\end{itemize}
\end{frame}

\subsection{Verkettung}

\subsubsection{Suche ohne Verkettung}

\begin{frame}[fragile]{Suche ohne Verkettung}
\begin{lstlisting}
$schema->resultset('Product')->search({
    active => 1,
    canonical_sku => undef,
});
\end{lstlisting}
\end{frame}

\subsubsection{Suche mit Verkettung}

\begin{frame}[fragile]{Suche mit Verkettung}
\begin{lstlisting}
$schema->resultset('Product')->search({
    active => 1,
})->search({
    canonical_sku => undef,
});
\end{lstlisting}
\end{frame}

\subsubsection{Verkettung mit Update}

Die Verkettung kann auch genutzt werden, um
ein Suchergebnis mit einer anderen Operation zu
verknüpfen, z.B. ein \verb|UPDATE|.

\begin{frame}[fragile]{Verkettung mit Update}
\begin{lstlisting}
$schema->resultset('Product')->search({
    manufacturer => 'Out of Fashion',
})->update({
    active => 0,
});
\end{lstlisting}
\end{frame}

\subsection{Vordefinierte Suche}

\begin{frame}[fragile]{Vordefinierte Suche}
\begin{lstlisting}
package Interchange6::Schema::ResultSet::Product;

sub active {
    my $self = shift;

    return $self->search({ $self->me('active') => 1 });
}

sub canonical_only {
    my $self = shift;

    return $self->search({ 
       $self->me('canonical_sku') => undef });
}

\end{lstlisting}
\end{frame}

\subsubsection{Vordefinierte Suche anwenden}

\begin{frame}[fragile]{Vordefinierte Suche anwenden}
\begin{lstlisting}
$schema->resultset('Product')->active->canonical_only;
\end{lstlisting}
\end{frame}

\begin{frame}[fragile]{Vordefinierte Suche: Produktvarianten}
\begin{lstlisting}
sub with_variant_count {
 my $self = shift;
 return $self->search(
  undef,
  {
   '+columns' => {
     variant_count =>
      $self->correlate('variants')->count_rs->as_query
   }
  }
 );
}
\end{lstlisting}
\end{frame}

\subsection{Korrelierte Subqueries}

\begin{frame}[fragile]{Korrelierte Subqueries - Interchange6::Schema}
  \begin{lstlisting} 
sub with_status {
  my $self = shift;

  return $self->search(
    undef,
      {
        '+columns' => {
        status => $self->correlate('statuses')->rows(1)
          ->order_by('!created,!order_status_id')->get_column('status')
          ->as_query,
        },
      }
  );
}
\end{lstlisting}
\end{frame}

\begin{frame}[fragile]{Korrelierte Subqueries - Perl Dancer Conference}
  \begin{lstlisting}
get '/admin/tickets' => require_role admin => sub {
  my $tokens = {};

  $tokens->{title} = "Tickets Sold";

  my $orders = rset('Order');

  $tokens->{orders} = $orders->search(
    {
       payment_status => 'paid',
    },
    {
       prefetch => 'user',
       order_by => 'orders_id',
    }
  )->with_status;

  template 'admin/tickets', $tokens;
};
\end{lstlisting}
\end{frame}

1;

\section{Relationen}

\subsection{Spezifische JOIN-Bedingungen}

\begin{frame}[fragile]{Spezifische JOIN-Bedingungen}

Relation für \textbf{alle} Adressen eines Benutzers:

\begin{lstlisting}

__PACKAGE__->has_many(
  "addresses",
  "CalevoMobile::Schema::Result::Address",
  { "foreign.username" => "self.username" },
  { cascade_copy => 0, cascade_delete => 0 },
);

\end{lstlisting}
\end{frame}

Relation für alle \textbf{Versandanschriften} eines Benutzers:

\begin{frame}[fragile]{Spezifische JOIN-Bedingungen}
\begin{lstlisting}

__PACKAGE__->has_many(
    "shipping_addresses",
    "CalevoMobile::Schema::Result::Address",
    sub {
        my $args = shift;

        return {
            "$args->{foreign_alias}.username" =>
                { -ident => "$args->{self_alias}.username",},
            "$args->{foreign_alias}.type" => 'shipping',
            "$args->{foreign_alias}.archived" => 0,
        }
    },
    { cascade_copy => 0, cascade_delete => 0 },
);
\end{lstlisting}
\end{frame}

% \begin{frame}[fragile]{Subjects included}
% \begin{itemize}
% \item Me Helper
% \verb|$self->me('active') => 1|
% \end{itemize}
% \end{frame}

\section{Ausbaufähig}

Ein weiteres wichtiges Feature von DBIx::Class ist die Möglichkeit von
diversen Erweiterungen. Hier fehlt leider ein Überblick in der
Dokumentation.

Komponenten behandeln wir in den folgenden Kapiteln.

\begin{frame}{Ausbaufähig}
\begin{itemize}
  \note{Zucker für DBIx::Class}
\item Komponenten
  \begin{itemize}
  \item Helpers
  \end{itemize}
\item Zusatzattribute
\item Schema "vererben"
\item Candy
\end{itemize}
\end{frame}

\subsection{Zusatzattribute für das Schema}

Für Funktionen der Businesslogik benötigen wir z.B.
Informationen über die Konfiguration der Webanwendung
bzw. den angemeldeten Benutzer.

\begin{frame}{Beispiel für Zusatzattribute}
\begin{itemize}
\item Konfiguration
\item angemeldeter Benutzer
\end{itemize}
\end{frame}

Beim Erzeugen der Schemainstanz können keine zusätzlichen Parameter
übergeben werden.

\begin{frame}[fragile]{Schemainstanz erzeugen}
\begin{lstlisting}
use Interchange6::Schema;

my $schema = Interchange6::Schema->connect(
  'dbi:Pg:dbname=perldance', 'racke', 'nevairbe', 
);
\end{lstlisting}
\end{frame}

\begin{frame}[fragile]{Zusatzattribute}
\begin{lstlisting}
__PACKAGE__->mk_group_ro_accessors(
    inherited => (
        [ 'current_user' => '_ic6_current_user' ]
    )
);

__PACKAGE__->mk_group_wo_accessors(
    inherited => (
        [ 'set_current_user' => '_ic6_current_user' ]
    )
);
\end{lstlisting}

Class::C3
\end{frame}

\begin{frame}[fragile]{Zusatzattribut setzen}
\begin{lstlisting}
sub BUILD {
  my $plugin = shift;
   weaken ( my $weak_plugin = $plugin );
  $plugin->app->add_hook(
    Dancer2::Core::Hook->new(
      name => 'before',
      code => sub {
        my $user = $weak_plugin->logged_in_user || undef;
        if ( $user ) {
          $user = $weak_plugin->shop_user->find(
            {
              username => $user->{username}
            }
           );
        }
        $weak_plugin->shop_schema->set_current_user($user);
      },
    )
  );
}
\end{lstlisting}
\end{frame}

% https://pixabay.com/en/treasure-chest-fire-flame-money-619937/

\subsubsection{Beispiel: Preisberechnung}

In einem Onlineshop ist die Preisberechnung immer ein heißes Thema.

\begin{frame}{Beispiel: Preisberechnung}
\begin{figure}[!ht]
\centering
\includegraphics[width=1\linewidth]{img/treasure-chest-fire.jpg}
\end{figure}
\end{frame}

\begin{frame}{Beispiel: Preisberechnung}
  \begin{itemize}
  \item Preislisten
  \item Aktionen
  \item Rabatte
  \end{itemize}
\end{frame}

\begin{frame}[fragile]{Tabelle price\_modifiers}
\begin{lstlisting}
primary_column price_modifiers_id => {
    data_type         => "integer",
    is_auto_increment => 1,
};
column sku => { data_type => "varchar", size => 64 };
column quantity =>  { data_type => "integer", default_value => 0 };
column roles_id =>  { data_type => "integer", is_nullable => 1 };
column price => {
    data_type     => "numeric",
    size          => [ 21, 3 ],
};
column start_date => {
    data_type     => "date",
    is_nullable   => 1,
};
column end_date => {
    data_type     => "date",
    is_nullable   => 1,
};
\end{lstlisting}
\end{frame}

\begin{frame}[fragile]{Preisberechnung}
Suchbedingung für Einträge in \verb|price_modifiers|:
\begin{lstlisting}
    my $search_cond = {
        'start_date' => [ undef, { '<=', $today } ],
        'end_date'   => [ undef, { '>=', $today } ],
        'quantity'   => { '<=' => $args->{quantity} },
        'roles_id'   => undef,
    };
\end{lstlisting}
\end{frame}


\begin{frame}[fragile]{Preisberechnung}
Bedingung für Rollen des Benutzers einfügen:

\begin{lstlisting}
if ( my $user = $schema->current_user ) {
    $search_cond->{roles_id} = [
        undef,
        {
            -in => $schema->resultset('UserRole')
                ->search( { users_id => $user->id } )
                ->get_column('roles_id')->as_query
            }
        ];
    }

\end{lstlisting}
\end{frame}

\subsection{Schemas vererben}

\begin{frame}[fragile]{Schemas vererben: Anwendungsbeispiele}
\begin{itemize}
\item Generisches Schema, z.B. \verb|Interchange6::Schema|
  \begin{itemize}
  \item \verb|PerlDance::Schema|
  \item \verb|DanceShop::Schema|
  \end{itemize}
\item Anwendung mit ähnlichen Datenbanken
\end{itemize}
\end{frame}

\begin{frame}{Schemas vererben}
\begin{itemize}
\item Tabellen hinzufügen
\item Spalten hinzufügen
\item Beziehungen hinzufügen
\end{itemize}
\end{frame}

\begin{frame}[fragile]{Beispiel Vererbung}
\begin{lstlisting}
package PerlDance::Schema;
our $VERSION = 16;

use Interchange6::Schema::Result::User;
package Interchange6::Schema::Result::User;

__PACKAGE__->add_columns(
    bio => { data_type => "varchar", size => 2048, 
             default_value => '' },
    media_id =>
      { data_type => "integer", is_nullable => 1 },
    pause_id => { data_type => "varchar", size => 128, 
        default_value => '' },
    t_shirt_size => { data_type => "varchar", size => 8, 
      is_nullable => 1 },
);
\end{lstlisting}
\end{frame}

Inherit from \verb|Interchange6::Schema| and set result namespace to 
\verb|Interchange6::Schema::Result| plus \verb|PerlDance::Schema::Result|.

\begin{frame}[fragile]{Beispiel Vererbung}
\begin{lstlisting}
package PerlDance::Schema;

use base 'Interchange6::Schema';

Interchange6::Schema->load_namespaces(
    default_resultset_class => 'ResultSet',
    result_namespace        =>
        [ 'Result', '+PerlDance::Schema::Result' ],
    resultset_namespace     =>
        [ 'ResultSet', '+PerlDance::Schema::ResultSet' ],
);
\end{lstlisting}
\end{frame}


\subsection{Candy - Zucker für DBIx::Class}

\begin{frame}{Zucker für DBIx::Class}
% https://pixabay.com/en/candy-cane-candy-cane-winter-488009/
\begin{figure}[!ht]
\centering
\includegraphics[width=0.8\linewidth]{img/candy-cane.jpg}
\end{figure}
\end{frame}

\begin{frame}[fragile]{Vanilla Result Class}
\begin{lstlisting}
package TravelDance::Schema::Result::Country;
use warnings;
use strict;
use base 'DBIx::Class::Core';

__PACKAGE__->table('countries');

__PACKAGE__->add_columns(
    country_iso_code => {
        data_type => "char",
        size      => 2,
    },
    name => {
        data_type => "varchar",
        size      => 255,
    },
);
\end{lstlisting}
\end{frame}

\begin{frame}[fragile]{Candy Result Class}
\begin{lstlisting}
package TravelDance::Schema::Result::Country;
use TravelDance::Schema::Candy;

primary_column country_iso_code => {
    data_type => "char",
    size      => 2
};

column name => {
    data_type => "varchar",
    size      => 255
};
\end{lstlisting}
\end{frame}

\subsubsection{Umstellung auf Candy}

\begin{frame}[fragile]{Umstellung auf Candy}
  \begin{itemize}
  \item Umstellung des ganzen Schemas
  \item Neue Resultklassen
  \item Neue Spalten, Relationen, ...
  \end{itemize}
\end{frame}

\begin{frame}[fragile]{Umstellung auf Candy}
  \begin{lstlisting}
package TravelDance::Schema::Result::User;

- use base 'DBIx::Class::Core';
+ use TravelDance::Schema::Candy;

-__PACKAGE__->table('users');
+table 'users';

\end{lstlisting}
\end{frame}

\section{Komponenten}

Es gibt eine Vielzahl von Komponenten für DBIx::Class.

\subsection{Komponenttypen}

\begin{frame}[fragile]{Komponenttypen}
  \begin{itemize}
  \item Komponenten für \emph{Schema}klasse
  \item Komponenten für \emph{Result}klassen
  \item Komponenten für \emph{Resultset}klassen
  \end{itemize}
\end{frame}

\subsection{Schema-Komponenten}

Schema-Komponenten, wie zum Beispiel einige Helper, können in die
Hauptschemaklassen mit der Methode load\_components eingebunden werden:

\begin{frame}[fragile]{Schema-Komponenten}
  \begin{lstlisting}
package Interchange6::Schema;

use strict;
use warnings;

use base 'DBIx::Class::Schema::Config';

__PACKAGE__->load_components(
    'Helper::Schema::QuoteNames'
    'Helper::Schema::DateTime',
);
\end{lstlisting}
\end{frame}

\subsection{ResultSet-Komponenten}

Um eine Komponente in alle ResultSet-Klassen des Schemas
Interchange6::Schema einzubinden, sind folgende
Schritte durchzuführen:

\begin{frame}[fragile]{ResultSet-Komponenten}
\begin{itemize}
\item Elternklasse anlegen \\
  \verb|Interchange6::Schema::ResultSet|
\item Als voreingestellte ResultSet-Klasse eintragen in \\
  \verb|Interchange6::Schema|
\item Elternklasse in vorhandene ResultSet-Klassen eintragen \\
 \verb|Interchange6::Schema::ResultSet::User|
\end{itemize}
\end{frame}

\subsubsection{Elternklasse anlegen}

\begin{frame}[fragile]{Elternklasse anlegen}
\begin{lstlisting}
package Interchange6::Schema::ResultSet;

use base 'DBIx::Class::ResultSet';

__PACKAGE__->load_components(
    'Helper::ResultSet::CorrelateRelationship',
    'Helper::ResultSet::Me',
    'Helper::ResultSet::Random',
    'Helper::ResultSet::SetOperations',
    'Helper::ResultSet::Shortcut'
);

\end{lstlisting}
\end{frame}

\subsubsection{Als Voreinstellung konfigurieren}

\begin{frame}[fragile]{Als Voreinstellung konfigurieren}
\begin{lstlisting}
package Interchange6::Schema;

__PACKAGE__->load_namespaces(
    default_resultset_class => 'ResultSet'
);

\end{lstlisting}
\end{frame}

\subsubsection{Vererbung von der Elternklasse}
\begin{frame}[fragile]{Vererbung von der Elternklasse}

Bereits vorhandene ResultSet-Klassen müssen die eigene
ResultSet-Elternklasse übernehmen:

\begin{lstlisting}
package Interchange6::Schema::ResultSet::User;

use base 'Interchange6::Schema::ResultSet';

__PACKAGE__->load_components(
    qw(Helper::ResultSet::CorrelateRelationship)
);
\end{lstlisting}
\end{frame}

\subsection{Komponenten für Resultklassen}

\subsubsection{Tree::Adjacency Komponente}

Eine Baumstruktur, wie z.B. für die Navigation einer
Webseite, kann man mit Hilfe eines \verb|parent_id|-Feldes
darstellen.

\begin{frame}[fragile]{Tree::Adjacency Komponente}
\begin{lstlisting}

package Interchange6::Schema::Result::Navigation;

...

__PACKAGE__->load_components(qw( Tree::AdjacencyList ... ));

...

__PACKAGE__->add_columns(
   ...
 "parent_id",
  { data_type => "integer", default_value => 0, is_nullable => 0 },
   ...
);

...

__PACKAGE__->parent_column('parent_id');

\end{lstlisting}
\end{frame}

\begin{frame}{Tree::Adjacency Methoden}
\begin{itemize}
\item ancestors
\item children
\item siblings
\end{itemize}
\end{frame}

\section{DBIx::Class Helpers}

\begin{frame}{DBIx::Class Helpers}
Typische Anwendungsfälle für DBIx::Class vereinfachen.
\end{frame}

\begin{frame}{DBIx::Class Helpers}
\begin{figure}[!ht]
\centering
\includegraphics[width=0.4\linewidth]{img/frew.png}
\caption{Arthur Axel "fREW" Schmidt}
\end{figure}
\end{frame}

% list of helpers we show

\begin{frame}{DBIx::Class Helpers}
\begin{itemize}
\item Helper::Schema::QuoteNames
\item Helper::ResultSet::Me
\item Helper::Row::ProxyResultSetMethod
\item Helper::Row::OnColumnChange
\end{itemize}
\end{frame}

\subsection{Helper::QuoteNames}

\begin{frame}[fragile]{Helper::QuoteNames}
\begin{itemize}
\item Maskierung von reservierten Wörtern
\item Änderungen zwischen Engines und Versionen
\item e.g. MySQL
\begin{itemize}
\item \verb|select user from userdb| => crash
\item \verb|select `user` from `userdb`| => works
\end{itemize}
\item \verb|quote_names| in connection info
\end{itemize}
\end{frame}

\begin{frame}[fragile]{Helper::QuoteNames}
\begin{lstlisting}

package Interchange6::Schema;

use base 'DBIx::Class::Schema';

__PACKAGE__->load_components( 
    'Helper::Schema::QuoteNames' 
);

...

\end{lstlisting}
\end{frame}

\subsection{Helper::ResultSet::Me}

Bei vordefinierten Suchen wissen wir nicht, welcher Tabellen-Alias
gebraucht wird.

Deshalb verwenden wir die \verb|current_source_alias|-Methode.

\begin{frame}[fragile]{Helper::Resultset::Me}
\begin{lstlisting}
sub active {
    my $self = shift;

    return $self->search({ 
        $self->current_source_alias . ".active" => 1,
    });
}
\end{lstlisting}
\end{frame}

Mit diesen Helper wird der Quellcode übersichtlicher:

\begin{frame}[fragile]{Helper::Resultset::Me}
\begin{lstlisting}
sub active {
    my $self = shift;

    return $self->search({ 
        $self->me('active') => 1,
    });
}
\end{lstlisting}
\end{frame}

\subsection{Helper::ResultSet::Shortcut}



\note{A lot of shortcuts are provided, I'll show
you just a few examples.}

\begin{frame}{Shortcuts}
\begin{itemize}
\item columns
\item like
\item hri
\end{itemize}
\end{frame}

\begin{frame}[fragile]{columns Shortcut}

\verb|columns| shortcut:

\begin{lstlisting}
$rs->columns([qw/ first_name last_name /]);
\end{lstlisting}

gleichbedeutend mit:

\begin{lstlisting}
$rs->search( undef, { 
    columns => [qw/ first_name last_name /] 
} );
\end{lstlisting}
\end{frame}

\begin{frame}[fragile]{like Shortcut}

\verb|like| shortcut:

\begin{lstlisting}
$rs->like( 'city', 'region', '%York' );
\end{lstlisting}

oder:

\begin{lstlisting}
$rs->like([ 'city', 'region' ], '%York' );
\end{lstlisting}

gleichbedeutend mit:

\begin{lstlisting}
$rs->search(
    { city => { -like => '%York' },
    { region => { -like => '%York' },
);
\end{lstlisting}
\end{frame}

\subsection{HashRefInflator}

Using the HashRefInflator makes sense when you need to quickly retrieve
data from a massive resultset or you need a list of hash references anyway,
e.g. for input to a template in a web application.

\begin{frame}[fragile]{HashRefInflator}
\begin{lstlisting}
my $rs = $schema->resultset('Country')->search({}, {
   result_class
     => 'DBIx::Class::ResultClass::HashRefInflator',
 });
\end{lstlisting}
\end{frame}

\begin{frame}[fragile]{HashRefInflator mit HRI helper}
\begin{lstlisting}
my $rs = $schema->resultset('Country')->search({})->hri;

# since 'search' here is redundant we can just use:
my $rs = $schema->resultset('Country')->hri;
\end{lstlisting}
\end{frame}

\begin{frame}[fragile]{Tickets}
\begin{lstlisting}
 $tokens->{tickets} = [
    $schema->resultset('Conference')
        ->find( $conferences_id )
            ->tickets->active->prefetch('inventory')
                ->hri->all 
];
\end{lstlisting}
\end{frame}

\subsection{Helper::Row::ProxyResultSetMethod}

\begin{frame}[fragile]{Helper::Row::ProxyResultSetMethod}
\begin{lstlisting}
package Interchange6::Schema::Product;

use Interchange6::Schema::Candy -components => [
    qw( Helper::Row::ProxyResultSetMethod );
];

proxy_resultset_method 'variant_count';
\end{lstlisting}
\end{frame}

\subsection{Helper::Row::OnColumnChange}

\begin{frame}[fragile]{Helper::Row::OnColumnChange}

Führe Aktionen aus, wenn sich der Wert
einer Spalte ändert:

\begin{itemize}
\item before\_column\_change
\item around\_column\_change
\item after\_column\_change
\end{itemize}

\end{frame}

\begin{frame}[fragile]{Helper::Row::OnColumnChange}

\begin{lstlisting}
package TravelDance::Schema::Result::User;

use TravelDance::Schema::Candy -components =>
    [qw(Helper::Row::OnColumnChange 
        InflateColumn::DateTime)];

use DateTime;

column last_password_change => {
    data_type => timestamp,
};
\end{lstlisting}
\end{frame}

\begin{frame}[fragile]{Helper::Row::OnColumnChange}

Nach der Passwortänderung:

\begin{itemize}
\item neues Passwort mit dem alten vergleichen
\item Wert in \verb|last_password_change|-Spalte aktualisieren
\end{itemize}

\end{frame}

\begin{frame}[fragile]{Helper::Row::OnColumnChange}
\begin{lstlisting}
after_column_change password => {
    method   => 'change_password',
    txn_wrap => 1,  # wrap it all in a transaction
};
\end{lstlisting}
\end{frame}

\begin{frame}[fragile]{Helper::Row::OnColumnChange}
\begin{lstlisting}
sub change_password {
    my ( $self, $old_value, $new_value ) = @_;
    if ( $self->check_password($new_value) ) {
        $self->throw_exception("Password not changed")
    }
    else {
        $self->update(
            { last_password_change => DateTime->now() }
        );
    }
}
\end{lstlisting}
\end{frame}

\subsection{Andere nützliche Helper}
\begin{frame}{Andere nützliche Helper}
\begin{itemize}
\item Helper::Schema::DateTime
\note{Easily correlate your ResultSets}
\item Helper::ResultSet::CorrelateRelationship
\note{Remove columns from search}
\item Helper::ResultSet::RemoveColumns
\note{Random results, don't use Helper::Random}
\item Helper::ResultSet::Random
\note{Force numeric context on numeric columns:}
\item Helper::Row::NumifyGet
\end{itemize}
\end{frame}

% \section{Writing Tests}
% \begin{frame}{Writing Tests}
% \end{frame}

\section{Schema \& Deployment}

\begin{frame}[fragile]{Deployment}
\begin{itemize}
\item Generierung aus der Datenbank
\begin{itemize}
\item \verb|dbicdump|
\end{itemize}
\item Generierung aus dem Schema
\begin{itemize}
\item \verb|DBIx::Class::DeploymentHandler|
\end{itemize}
\end{itemize}
\end{frame}

Es gibt aber auch Situationen, wo die
Generierung aus der Datenbank sinnvoll ist,
z.B. wenn man mit DBIx::Class beginnt und
bereits ein Schema für eine vorhandene
Datenbank benötigt.


\subsection{Generierung aus der Datenbank}

\verb|dbicdump| wird verwendet um ein DBIx::Class-Schema für eine
existierende Datenbank zu erstellen..

\begin{frame}[fragile]{Generierung aus der Datenbank: dbicdump}
\begin{lstlisting}
dbicdump -o dump_directory=/home/dance/TravelDance/lib 
         TravelDance::Schema 
         dbi:Pg:dbname=perldance
\end{lstlisting}
\end{frame}

\verb|dbicdump| kann auch verwendet werden, um Änderungen in
der Datenbankstruktur in das Schema zu übernehmen.

\begin{frame}[fragile]{Generierung aus der Datenbank: dbicdump}
\begin{itemize}
\item \verb|dbicdump| mehrfach anwenden
\item zusätzlichen Methoden
\end{itemize}
\end{frame}

\begin{frame}[fragile]{Nachteile dbicdump}
\begin{itemize}
\item Komponenten, Helpers
\item Candy
\item generische Schemas
\end{itemize}
\end{frame}

\subsection{Deployment Handler}

\begin{frame}{Deployment Handler}
\begin{itemize}
\item DBIx::Class::DeploymentHandler
\item Datenbankänderungen einspielen
\item Datenbank upgraden
\item Datenbank downgraden
\end{itemize}
\end{frame}

\subsection{Skripte für Deployment Handler}
\begin{frame}{Skripte für Deployment Handler}
\begin{description}
\item[dh-prepare-version-storage] Installation Datenbanktabelle vorbereiten
\item[dh-install-version-storage] Installation Datenbanktabelle einspielen
\item[dh-prepare-upgrade] Upgrade vorbereiten
\item[dh-upgrade] Upgrade einspielen
\end{description}
\end{frame}

\subsection{Vorgehensweise für Updates}

\begin{frame}{Vorgehensweise für Updates}
\begin{itemize}
\item Backup anlegen
\item Schema ändern
\item Skripts hinzufügen
\item Versionsnummer erhöhen
\item Upgrade vorbereiten
\item SQL-Befehle kontrollieren
\item Upgrade einspielen
\end{itemize}
\end{frame}

Note: create backup can be done by DeploymentHandler itself.

\subsection{Schema ändern}

\begin{frame}{Schema ändern}
\begin{itemize}
\item Tabelle hinzufügen
\item Spalte hinzufügen
\item Spalte ändern
\end{itemize}
\end{frame}

\subsection{Versionsnummer erhöhen}
\begin{frame}{Versionsnummer erhöhen}
\begin{itemize}
\item Natürliche Zahlen: 1, 2, 3, ...
\item Erhöhen: 1 => 2
\end{itemize}
\end{frame}

\begin{frame}[fragile]{Version 1}
\begin{lstlisting}
package TravelDance::Schema;
use warnings;
use strict;
use base 'DBIx::Class::Schema';

our $VERSION = 1;

__PACKAGE__->load_namespaces();

1;
\end{lstlisting}
\end{frame}

\begin{frame}[fragile]{Version 2}
\begin{lstlisting}
package TravelDance::Schema;
use warnings;
use strict;
use base 'DBIx::Class::Schema';

our $VERSION = 2;

__PACKAGE__->load_namespaces();

1;
\end{lstlisting}
\end{frame}

\subsection{Upgrade vorbereiten}

\begin{frame}[fragile]{Upgrade vorbereiten}
\begin{lstlisting}
my $dh     = DBIx::Class::DeploymentHandler->new(
    {
        schema              => $schema,
        databases           => 'MySQL',
        sql_translator_args => { add_drop_table => 0 }
    }
);
$dh->prepare_deploy;
$dh->prepare_upgrade(
    {
        from_version => $dh->database_version,
        to_version => $dh->schema_version
    }
);
\end{lstlisting}
\end{frame}

\subsection{Eigene Skripte hinzufügen}

\begin{frame}{Eigene Skripte hinzufügen}
\begin{itemize}
\item Deployment
\item für alle Upgrades
\item bestimmte Upgrades
\end{itemize}
\end{frame}

\begin{frame}[fragile]{Guru update}

\verb|sql/_common/upgrade/13-14/010-gurus.pl|

\begin{lstlisting}
#!perl
sub {
    my $schema = shift;
    $schema->resultset('User')->search(
        {
            username => {
                -in => [
                    'xsawyerx@cpan.org',
                    'russell.jenkins@strategicdata.com.au',
                    'rabbit@rabbit.us',
                ]
            }
        }
    )->update( { guru_level => 100 } );
};

\end{lstlisting}
\end{frame}

\subsection{Verzeichnisse und Dateien}

Reference for directory layout:
\href{https://metacpan.org/pod/DBIx::Class::DeploymentHandler::DeployMethod::SQL::Translator}{DBIx::Class::DeploymentHandler::DeployMethod::SQL::Translator}

\begin{frame}[fragile]{Verzeichnisse und Dateien}
\begin{description}
\item[sql/PostgreSQL] Datenbankspezifische SQL-Skripte
\begin{description}
\item[sql/PostgreSQL/deploy/1/001-auto.sql] Deploy Version 1
\item[sql/PostgreSQL/upgrade/1-2/001-auto.sql] Upgrade 1 => 2
\end{description}
\item[sql/\_common] Eigene Skripte
\begin{description}
\item[sql/\_common/upgrade/\_any] Alle Upgrades
\item[sql/\_common/upgrade/1-2] Upgrade 1 => 2
\end{description}
\item[sql/\_deploy] Strukturdateien (YAML)
\end{description}
\end{frame}

\subsection{Deployment Handler CLI}

\begin{frame}[fragile]{Deployment Handler CLI}
\begin{lstlisting}
    #! /usr/bin/env perl

    use strict;
    use warnings;
    use PerlDance::Schema;
    use DBIx::Class::DeploymentHandler::CLI;

    my $schema = PerlDance::Schema->connect('perldance');

    my $dh_cli = DBIx::Class::DeploymentHandler::CLI->new(
        schema => $schema,
        databases => 'PostgreSQL',
        args => \@ARGV,
    );

    if (my $ret = $dh_cli->run) {
       print $ret, "\n";
    }
\end{lstlisting}
\end{frame}

\section{Best Practices}

\begin{itemize}
\item Benennung Primärschlüssel
\item Helper::QuoteNames verwenden
\end{itemize}

\section{Abschluß}

\begin{frame}{Abschluß}
\begin{itemize}
\item Resources
\item Slides
\item Fragen
\end{itemize}
\end{frame}

\subsection{Resources}
\begin{frame}[fragile]{Resources}
\begin{itemize}
\item Extensive Dokumentation
\begin{itemize}
\item \verb|DBIx::Class::Manual::*|
\item \verb|DBIx::Class::Manual::ResultClass|
\item \verb|DBIx::Class::ResultSet|
\item \verb|DBIx::Class::Relationship::*|
\end{itemize}
\item Helpers von fREW
\begin{itemize}
\item \href{https://metacpan.org/pod/DBIx::Class::Helpers}{https://metacpan.org/pod/DBIx::Class::Helpers}
\end{itemize}
\item \href{http://www.perladvent.org/2012/2012-12-21.html}
{Perl Advent Calendar 2012: Set-based DBIx::Class}
by fRew
% \item \href{https://wiki.linuxia.de/library/stefan-hornburg-racke-a-train-ride-through-dbix-class-and-its-ecosystem-en}{A train ride through DBIx::Class and its ecosystem}
\end{itemize}


\end{frame}

\subsubsection{Interchange Resources}

\begin{frame}{Interchange Resources}
\begin{figure}[!ht]
\centering
\includegraphics[width=0.4\linewidth]{img/interchange6-logo-v2.png}
\end{figure}
\begin{itemize}
\item Interchange6::Schema
\item Demo Shop
\item Perl Dancer Conference
\end{itemize}
\end{frame}

\subsection{Slides}

\begin{frame}{Slides}
Slides:
\url{http://www.linuxia.de/talks/hhmongers2071/dbic-pr-de-beamer.pdf}
\end{frame}

\subsection{Fragen}

\begin{frame}{Fragen}
\centering
Fragen ?
\end{frame}

\end{document}

%%% Local Variables: 
%%% mode: latex
%%% TeX-master: t
%%% End: 
